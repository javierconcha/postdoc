%%%%%%%%%%%%%%%%%%%%%%%%%%%%%%%%%%%%%%%%%%%%%%%%%%%%%%%
%                File: express_style.tex              %
%             Created: 2 September 2009               %
%              Updated: 29 August 2017                %
%                                                     %
%           LaTeX template file for use with          %
%           OSA's journals Optics Express,            %
%             Biomedical Optics Express,              %
%            and Optical Materials Express            %
%                                                     %
%  send comments to Theresa Miller, tmiller@osa.org   %
%                                                     %
%       (c) 2017 Optical Society of America           %
%%%%%%%%%%%%%%%%%%%%%%%%%%%%%%%%%%%%%%%%%%%%%%%%%%%%%%%

\documentclass[10pt]{article}
%% Specify the Express journal you are submitting to
%\usepackage[OME]{express}
\usepackage[OE]{express}
%\usepackage[BOE]{express}

%%% MY PACKAGES %%%%%%%%%%%%%%%%%%%%%%%%%%%%%%%%%%%%%%%%%%%%%%%
\usepackage{graphicx}
% \usepackage[outdir=./]{epstopdf}
\usepackage{epstopdf}
\epstopdfsetup{update} % only regenerate pdf files when eps file is newer
\usepackage{amsmath,epsfig}

% Select what to do with todonotes: 
\usepackage[disable]{todonotes} % notes not showed
% \usepackage[draft]{todonotes}   % notes showed
% \usepackage[textwidth=2.0cm]{todonotes}
\presetkeys{todonotes}{fancyline, size=\scriptsize}{}
\setlength{\marginparwidth}{3cm}

\usepackage{tikz} % for flow charts
  \usetikzlibrary{shapes,arrows,positioning,shadows,calc}
  % \usetikzlibrary{external}
  % \tikzexternalize[prefix=Figures/]

% \usepackage[nostamp]{draftwatermark}
% \SetWatermarkLightness{0.8}
% \SetWatermarkScale{4}

\usepackage[percent]{overpic}
\usepackage{morefloats} % for the error "Too many unprocessed floats"

\usepackage{multirow}

% \renewcommand*{\bibfont}{\normalsize}

\usepackage{float}
\usepackage{hyperref}
\usepackage{pdflscape}
%%% END MY PACKAGES %%%%%%%%%%%%%%%%%%%%%%%%%%%%%%%%%%%%%%%%%%%

\begin{document}
\title{GOCI's Vicarious Calibration in SeaDAS/L2GEN}

\author{Javier Concha\authormark{1,2,*}, Antonio Mannino\authormark{1}, Bryan Franz\authormark{1}, Sean Bailey\authormark{1}, and Wonkook Kim\authormark{3}}

\address{\authormark{1}Ocean Ecology Lab,
NASA Goddard Space Flight Center,
8800 Greenbelt Rd, Greenbelt, MD 20771, USA\\
\authormark{2}Universities Space Research Association, Columbia, MD, USA\\
\authormark{3}Korea Institute of Ocean Science and Technology, 787 Haean-ro, Ansan, Republic of Korea}

\email{\authormark{*}javier.concha@nasa.gov} %% email address is required

% \homepage{http:...} %% author's URL, if desired

%%%%%%%%%%%%%%%%%%% abstract and OCIS codes %%%%%%%%%%%%%%%%
%% [use \begin{abstract*}...\end{abstract*} if exempt from copyright]

\begin{abstract}
% 100 words
Vicarious calibration for the Geostationary Ocean Color Instrument.
% Background/motivation/context
%
% Aim/objectives(s)/problem statement
% 
% Methods
%
% Results
%
% Conclusions
%
\end{abstract}

\ocis{(010.4450) Oceanic optics; (280.4991) Passive remote sensing.} % REPLACE WITH CORRECT OCIS CODES FOR YOUR ARTICLE, MINIMUM OF TWO; Avoid using the OCIS codes for “General” or “General science” whenever possible.
%For a complete list of OCIS codes, visit: https://www.osapublishing.org/oe/submit/ocis/

%%%%%%%%%%%%%%%%%%%%%%% References %%%%%%%%%%%%%%%%%%%%%%%%%
% \begin{thebibliography}{99}
 
\bibliographystyle{osajnl}
\bibliography{/Users/jconchas/Documents/Latex/bib/javier_NASA.bib}

% \end{thebibliography}

\small
\tableofcontents
\normalsize
%%%%%%%%%%%%%%%%%%% SECTION %%%%%%%%%%%%%%%%%%%%%%%%%%%%%%%%
\section{Introduction}
%%%%%%%%%%%%%%%%%%% SECTION %%%%%%%%%%%%%%%%%%%%%%%%%%%%%%%%
\section{Approach}
%-------------------sub-section-----------------------------
\subsection{Calibration of the Near-Infrared Bands}
The aerosol model used was r95f10v01 ($\alpha=0.852786$ from the 80 standard models) for an angstrom coefficient ($\alpha$) equal to 0.9.
The angstrom coefficient was calculated from MODIS-Aqua data over the same region of study.
%-------------------sub-section-----------------------------
\subsection{Calibration of the Visible Bands}
% jconchas:~/Documents/Research/GOCI/GOCI_ViCal/test$ cat l2gen_test.param 
% ifile=COMS_GOCI_L1B_GA_20121025011640.he5
% ofile1=G20121025011640.OCCAL_valregion 
% sline=4144 
% eline=5142 
% spixl=3169 
% epixl=5167 
% l2prod=default,ag_412_mlrc,poc,angstrom,aot_nnn,sena,senz,sola,solz,brdf,Lw_nnn,nLw_nnn,vgain_vvv
% gain=[1.0,1.0,1.0,1.0,1.0,1.0,1.0,1.0] 
% vcal_nLw=2.17247,1.84507,1.21365,0.30201,0.01827,0.01969,0.00000,0.00000
% vcal_opt=2



% jconchas:~/Documents/Research/GOCI/GOCI_ViCal/test$ cat val_extract_test.param 
% ifile=G20121025011640.OCCAL_valregion 
% ofile=G20121025011640.OCCAL_valregion.o 
% elat=29.4736 
% slat=24.2842 
% slon=131.9067 
% elon=142.3193
% valid_ranges=vgain_=0.5:1.5
% ignore_flags=LAND CLDICE STRAYLIGHT ATMFAIL


% GOCI slot edge area were removed

%%%%%%%%%%%%%%%%%%% SECTION %%%%%%%%%%%%%%%%%%%%%%%%%%%%%%%%
\section{Verification of the calibraion}
%-------------------sub-section-----------------------------
\subsection{AERONET-OC}
The atmospheric correction was validated using {\it in situ} observations from the AErosol RObotic NETwork-Ocean Color (AERONET-OC) as ground truth \cite{Zibordi2009}. The quality-assurance (QA) level used was level 2.0, which is the highest quality for the AERONET-OC data. The dataset from two stations were used for the analysis: Gageocho (N=20; PI: Jae-Seol Shim and Joo-Hyung Ryu) and Ieodo (N=25; PI: Young-Je Park and Hak-Yeol You). \autoref{fig:GOCI_AERO} shows scatter plots for satellite derived $R_{rs}(\lambda)$ versus {\it in situ} measurements. 
% Antonio: since we focus on diurnal variability, the time window for the AERONET validation should be reduced to +/- 1 hour because it doesn't make sense to use a 6 hour time window if our premise is that coastal waters express diurnal variability.  AERONET-OC sites are located in coastal waters, and we don't have knowledge as to their diurnal variability.   

The selection of matchups followed the satellite validation protocols described in \cite{Bailey2006}. GOCI data acquired within a three hours windows of the {\it in situ} sampling were considered as potential validation matchups. A $7\times7$ GOCI pixel array is extracted centered in the {\it in situ} sampling location.
% Antonio: why 7x7?  Why not 5x5?  These sites may have spatial variability.
A filtered mean is calculated from these $7\times7$ arrays using \autoref{eq:filtered_value} and \autoref{eq:filtered_mean} \cite{Bailey2006}. A minimum of at least half the total of pixels in the $7\times7$ pixel array, i.e. $49/2\approx25$ pixels, were required to be valid (unflagged) for inclusion of the matchup in the validation analysis. Additionally, a coefficient of variation (CV, filtered mean divided by the filtered standard deviation) for the visible bands 412 to 555 nm and the aerosol optical thickness (AOT) at 864 nm was calculated for each pixel array that passes the exclusion criteria described above, and then, the median value of these coefficients of variation ($\text{Median}[CV]$) was calculated. Finally, the pixel arrays whose $\text{Median}[CV]>0.15$, as suggested by \cite{Bailey2006}, are excluded from the validation analysis.

A validation analysis was performed by comparing the satellite-derived retrievals of products with the {\it in situ} observations based on different statistical parameters (\autoref{tab:val_stats}). These statistical parameters are: the slope and offset of the fitted Reduced Major Axis (RMA) regression line of the form $y=m*x+b$, and its coefficient of determination $R^2$, the root mean squared error (RMSE), the mean, standard deviation, and median of the absolute percentage difference (APD)(MAPD, $\pm$sd APD, and Median APD, respectively), the percentage bias, median ratio of computed filtered mean satellite value ($R_{rs:ret}$) to {\it in situ} measurement ($R_{rs:in}$), and the semi-interquartile range (SIQR) \cite{Bailey2006}. These parameters are defined as:
% \begin{linenomath*}
\begin{equation}
  \text{APD$_n$(\%)}=\left[\frac{\displaystyle \left|R_{rs:in}^1-R_{rs:ret}^1 \right|}{R_{rs:in}^1},\dots,\frac{\displaystyle \left|R_{rs:in}^n-R_{rs:ret}^n \right|}{R_{rs:in}^n},\dots,\frac{\displaystyle \left|R_{rs:in}^N-R_{rs:ret}^N \right|}{R_{rs:in}^N}\right]*100,\ n=1,\dots,N
\end{equation}
% \end{linenomath*}
\noindent where N is the total number of matchups, and $R_{rs:in}^n$ is the $n^{th}$ {\it in situ} $R_{rs}$ and $R_{rs:ret}^n$ is the $n^{th}$ satellite-derived $R_{rs}$.
% \begin{linenomath*}
\begin{equation}
  \text{MAPD(\%)} = \frac{1}{N} \sum_{n=1}^{N} \text{APD$_n$(\%)}
\end{equation}
% \end{linenomath*}
% \begin{linenomath*}
\begin{equation}
  \text{$\pm$ \text{sd} APD(\%)} =  SD\left[\text{APD$_n$(\%)}\right]
\end{equation}
% \end{linenomath*}
\noindent with $SD$ the standard deviation.
% \begin{linenomath*}
\begin{equation}
  \text{\text{Median} APD(\%)} =  Median\left[\text{APD$_n$(\%)}\right]
\end{equation}
% \end{linenomath*}
% \begin{linenomath*}
\begin{equation}
   \text{RMSE} = \sqrt{\frac{\displaystyle \sum_{n=1}^{N} \left(R_{rs:in}^n-R_{rs:ret}^n\right)^2}{N}}
\end{equation}
% \end{linenomath*} 
% \begin{linenomath*}
\begin{equation}
    \text{\% Bias} = \frac{\displaystyle \frac{1}{N}*\sum_{n=1}^N(R_{rs:ret}^n-R_{rs:in}^n)}{\text{Mean}[R_{rs:in}^n]}*100
\end{equation}
% \end{linenomath*}
% \begin{linenomath*}
\begin{equation}
  \text{Median ratio} =  Median\left[\frac{R_{rs:ret}^1}{R_{rs:in}^1},\dots,\frac{R_{rs:ret}^n}{R_{rs:in}^n},\dots,,\frac{R_{rs:ret}^N}{R_{rs:in}^N}\right],\ n=1,\dots,N
\end{equation}
% \end{linenomath*}
% \begin{linenomath*}
\begin{equation}
    \text{SIQR} = \frac{Q_3-Q_1}{2}
\end{equation}
% \end{linenomath*}
\noindent where $Q_3$ and $Q_1$ are the $75^{th}$ and $25^{th}$ percentiles for the ratios of the satellite-derived values to the {\it in situ} measurements.  

% Antonio: results show some near-zero values at 412 and 443nm?  clearly, there is a bias where GOCI Rrs is generally lower than AERONET.
% \autoref{fig:GOCI_AERO} shows scatter plots for AERONET-OC data versus GOCI matchups. 
% The data were separated by stations and color coded by the times of the day in order to evaluate the influence of the solar zenith angle in the validation matchups. 
% The statistics calculated for each time of the day and for all matchups (highlighted in bold cases) are shown in \autoref{tab:val_stats}.
The matchups of coincident AERONET-OC and GOCI $R_{rs}$ demonstrate generally good agreement (\autoref{fig:GOCI_AERO}; \autoref{tab:val_stats}). 
The AERONET-OC stations are located in two different kind of waters, which is reflected in the scatter plots, with the Ieodo having greater $R_{rs}$ values than Gaeocho. Overall, the MAPD value is large in the first hours of the day (0h and 1h), decreases towards the midday, becoming the smallest value at 4h, and the increases for last time of the day, for all band. This fact suggests there is an influence of the solar zenith angle. However, the number of matchups decreases throughout the day, having only 1 matchup for the last two hours of the day (6h and 7h), and therefore, there are not enough data to be conclusive. 

When all the data are included in the analyses, a good agreement was found between the retrieved $R_{rs}$ and {\it in situ} observations, with $R^2$ values varying from 0.81 to 0.98 for the 412 to 660 nm GOCI bands. 
The validation statistics indicate that our results show worse agreement than \cite{Ahn2015}, which is reflected in a consistently larger MAPD and RMSE for all bands in this study, even though the $R^2$ values are slightly larger for this study. \cite{Ahn2015} included a vicarious calibration in the atmospheric correction algorithm using {\it in situ} data. This suggests that an improved vicarious calibration based on in situ data could reduce the errors in GOCI data products derived from l2gen. Also, GOCI $R_{rs}$ is generally lower than the AERONET-OC values, overall.
 
% , exceeding the values previously reported by \cite{Ahn2015}
% From Antonio: it's not worthwhile to compare individual statistics metrics from another paper, but rather a group a statistics; our results show better (or worse) agreement than Ahn et al. 2015 based a combination of metrics (R2, RSME, APD, etc.).
%-%-%-%-%-%-%-%-%-%-%=END FIGURE=%-%-%-%-%-%-%-%-%-%-%-%-%
% - - - - - - - - - - - - - - - - - - - - - - - - - - - - - - - -
% \subsubsection{Cruises Matchups?}
% wavelength  APD    APD         RMSE    RMSE       R^2     R^2
%             (ours) (Ahn's)     (ours)  (Ahn's)    (ours)  (Ahn's)
% 412         40.1  > 22.3        0.0021 > 0.0015     0.81  >  0.78
% 443         28.0  > 22.0        0.0017 > 0.0013     0.91  >  0.89
% 490         25.7  > 12.7        0.0029 > 0.0013     0.95  >  0.93
% 555         22.8  > 10.4        0.0029 > 0.0015     0.98  >  0.94
% 660         39.3  > 34.7        0.0007 < 0.0008     0.97  >  0.87

\begin{table}[htbp!]
%\internallinenumbers
\caption{Satellite validation statistics of the atmospheric correction algorithm for GOCI. Satellite-derived values were compared with {\it in situ} observations from two AERONET-OC stations. Regression line of the form $y=m*x+b$ using the Reduced Major Axis (RMA). The statistics for all the matchups are highlighted in bold cases. \label{tab:val_stats} }

  \centering
    \includegraphics[height=11cm]{./Figures/val_stats.pdf}

\end{table}

\begin{figure}[htbp!]
  \centering
    \includegraphics[height=9cm]{./Figures/GOCI_AERO.pdf}

    %\internallinenumbers
    \caption{Scatter plots showing the comparison between the satellite-derived GOCI values and AERONET-OC {\it in situ} observations (Gaeocho: circles; Ieodo: triangles). The dashed black line is the 1:1 line, and the Reduced Major Axis (RMA) regression line is drawn in red. The data were color coded by the times of the day in order to evaluate the influence of the solar zenith angle in the validation matchups (red: 0h, green: 1h, blue: 2h, black: 3h, cyan: 4h, magenta: 5h, orange: 6h, and purple: 7h). \label{fig:GOCI_AERO} } 
\end{figure}

%-------------------sub-section-----------------------------
\subsection{Sensor Cross-comparison}

We computed the time series of monthly means for the Visible Infrared Imaging Radiometer Suite (VIIRS)\todo{cite}~on board the Suomi National Polar-orbiting Partnership (Suomi NPP) weather satellite and the Moderate Resolution Imaging Spectroradiometer (MODIS)\todo{cite}~on board the Aqua satellite (MODISA) over the same GCW region for cross-comparison with GOCI (\autoref{fig:CrossCompAllRrs}). 
These data were filtered following the same previous exclusion criteria and then averaged by month. 
For GOCI, the mean of the three midday values were used.

\autoref{fig:CrossCompAllRrs}.(a) shows the cross-comparisonfor the monthly time series for GOCI, MODISA and VIIRS for all wavelengths. 
Overall, some differences in $R_{rs}$ among the three mission that vary by season are observed, with a larger discrepancy in the red bands. 
GOCI follows a similar trend as MODISA, with GOCI slightly lower, while VIIRS is higher overall when compared to GOCI and MODISA, especially for the 660 and 680 nm bands.
\autoref{fig:CrossCompAllRrs}.(b-d) show the time series for ratios by spectral bands for GOCI/MODISA, GOCI/VIIRS, and MODIS Aqua/VIIRS, respectively. For the GOCI/MODISA ratio, the mean value fluctuates around one except for the 680 nm band, which fluctuates around 0.75 suggesting that the MODISA 678 nm band is brighter than the GOCI 680 nm band overall. Also, for the 660 nm band, the ratio is slightly smaller for some periods with an exception in the first months of 2013, where the ratio almost reaches three. For the GOCI/VIIRS, the ratio fluctuates 0.9, except the 660 nm band. For the MODISA/VIIRS, the ratio varies close to one as well, except for the 660 nm band, which fluctuates around 0.6. GOCI displays a consistent behavior from year to year and a no evident relative drift was found.
%-%-%-%-%-%-%-%-%-%-%=FIGURE=%-%-%-%-%-%-%-%-%-%-%-%-%
\begin{figure}[H]
  \centering
  \includegraphics[width=17cm]{./Figures/CrossCompAllRrs.pdf}
    %\internallinenumbers
    \caption{Cross-comparison with MODIS and VIIRS for all wavelengths. (a) Rrs, (b) GOCI/MODIS Aqua ratio, (c) GOCI/VIIRS ratio, and (d) MODIS Aqua/VIIRS ratio. \label{fig:CrossCompAllRrs} } 
\end{figure}
%-%-%-%-%-%-%-%-%-%-%=END FIGURE=%-%-%-%-%-%-%-%-%-%-%-%-%
\autoref{fig:GOCI_TimeSeriesComp_par}\todo{change axis for $a_g$}~shows the monthly time series comparison for GOCI, MODISA and VIIRS for the Chlorophyll-{\it a}, $a_g(412)$ and POC\todo{create ratio for products}. A good consistency in range of the retrieved values was found, for the most part, for the three missions and for the three different products, and a good consistency in phasing of seasonal cycles.
%-%-%-%-%-%-%-%-%-%-%=FIGURE=%-%-%-%-%-%-%-%-%-%-%-%-%
\begin{figure}[H]
  \centering
  \includegraphics[width=14cm]{./Figures/GOCI_TimeSeriesComp_par.pdf}
    %\internallinenumbers
    \caption{Time Series comparison for GOCI (blue solid line), MODISA (red solid line) and VIIRS (black solid line) for (a) chlor-{\it a}, (b) $a_g(412)$ and (c) POC. Overall, all the products follow a similar pattern. \label{fig:GOCI_TimeSeriesComp_par}} 
\end{figure}
%-%-%-%-%-%-%-%-%-%-%=END FIGURE=%-%-%-%-%-%-%-%-%-%-%-%-%

\autoref{fig:scatterRrs}\todo{change axis to start from zero}~shows the scatter plots for the $R_{rs}(\lambda)$ cross-comparison for GOCI, MODISA and VIIRS\todo{create table}. These data are daily values and only values greater than zero are shown. The selection of the data over the GCW region followed similar procedures to the ones described by \cite{Bailey2006} and only data that passed the exclusion criteria are used. As it can be seen in the plots, there are few coincidental data among all missions and especially between MODISA and VIIRS for all bands. This could be caused by the failure of the atmospheric correction or presence of cloud in the scenes. The $R^2$ values are high for the 412 and 443 nm bands, and start to decrease for 490 nm and beyond. There are only a handful of coincidental data for MODISA and GOCI for the 680 nm band.
%-%-%-%-%-%-%-%-%-%-%=FIGURE=%-%-%-%-%-%-%-%-%-%-%-%-%
\begin{figure}[H]
  \centering
  \includegraphics[width=15cm]{./Figures/scatterRrs.pdf}
    %\internallinenumbers
    \caption{Scatter plots for the $R_{rs}(\lambda)$ cross-comparison for GOCI, MODIS Aqua and VIIRS. Linear regression in solid red line. \label{fig:scatterRrs} } 
\end{figure}
%-%-%-%-%-%-%-%-%-%-%=END FIGURE=%-%-%-%-%-%-%-%-%-%-%-%-%

%%%%%%%%%%%%%%%%%%% SECTION %%%%%%%%%%%%%%%%%%%%%%%%%%%%%%%%
\section{Conclusions}
% Practical applications
% Disadvantages and Advantages
% Limitations
% Challenges

% General
% final conclusion
% 4.1 Validation of the Atmospheric Correction ....................... 8 
%     4.1.1 AERONET-OC.................................. 8 
First, a validation using in situ data from the AERONET-OC was performed.
A good agreement was found between the GOCI $R_{rs}$ retrievals and {\it in situ} data, reflected in $R^2$ values varying from 0.81 to 0.98 for the 412 to 660 nm bands.
% The validation with {\it in situ} data exhibit results comparable to heritage sensors.
When the atmospheric correction scheme is compared in a statistical sense with the scheme used by \cite{Ahn2015}, our results agree less with the AERONET-OC dataset. 
This suggest that an improvement to the vicarious calibration included in l2gen is needed.

% 4.6 Sensor Cross-comparison................................. 25
The monthly time series for GOCI was cross-compared with the time series for the heritage sensors MODIS Aqua and VIIRS (\autoref{fig:CrossCompAllRrs} and \autoref{fig:GOCI_TimeSeriesComp_par}).
% from Bryan's ppt
Some seasonally varying differences in $R_{rs}$ are observed, with a larger discrepancy for the red bands. 
The mission present a consistent behavior from year to year without a evident relative drift.
The time series for the mission to mission ratios vary seasonally, fluctuating around a value close to one, with the exception of the red bands.
Based on scatter plots (\autoref{fig:scatterRrs}) for the cross-comparison with the heritage missions, the differences vary spectrally, being smaller in the blue and increasing towards the red. 

%     4.1.2 Cruises Matchups?................................ 11

The GOCI clear water region is covered by two slots of the GOCI instrument, which is visually noticeable from the images. This introduces uncertainties in the data products. A partial solution from the Korea Institute of Ocean Science and Technology (KIOST)\todo{correct?}~is underway.\todo{check!}
%%%%%%%%%%%%%%%%%%% SECTION %%%%%%%%%%%%%%%%%%%%%%%%%%%%%%%%
\section*{Funding}
NASA Project ROSES Earth Science U.S. Participating Investigator (NNH12ZDA001N-ESUSPI)
%%%%%%%%%%%%%%%%%%% SECTION %%%%%%%%%%%%%%%%%%%%%%%%%%%%%%%%
\section*{Acknowledgments}
We want to acknowledge the Korea Ocean Satellite Center for providing the GOCI L1B data to OBPG, and the Ocean Biology Processing Group at the Goddard Space Flight Center, NASA. Also, the principal investigator for the AERONET-OC data: Jae-Seol Shim and Joo-Hyung Ryu (Gageocho station), Young-Je Park and Hak-Yeol You (Ieodo station).

\end{document}

% Sean's comments:
% Javier,

% The 'trick' is to find a reasonably stable body of water that can be assumed to
% have zero water-leaving radiance in the NIR (i.e. clear, deep ocean) and
% preferably one where an assumption of the aerosol type can be made.  It is best
% if the type is primarily maritime, but whatever it is should be non-absorbing
% and within our available model suite.

% If you have a time series of the angstrom exponent for the region you've been
% using for the visible gain calculation (from SeaWiFS or MODIS, NOT GOCI - as
% that is to be considered suspect until verified), you can use that to choose
% the model.  You *could* let the model vary based on a climatology of angstrom,
% but that might complicate the process too much...

% With the know model, you set aermodels=<my favorite model> and aer_opt=1,
% vcal_opt=1 and voila!  vgain_745 can be achieved.

% Sean

% aer_opt	Option for aerosol calculation mode.  (Default=-3) 
 
%   1	Multi-scattering with fixed model (Oceanic, 99% humidity)
%   2	Multi-scattering with fixed model (Maritime, 50% humidity)
%   3 	Multi-scattering with fixed model (Maritime, 70% humidity)
%   4	Multi-scattering with fixed model (Maritime, 90% humidity)
%   5	Multi-scattering with fixed model (Maritime, 99% humidity)
%   6	Multi-scattering with fixed model (Coastal, 50% humidity)
%   7	Multi-scattering with fixed model (Coastal, 70% humidity)
%   8 	Multi-scattering with fixed model (Coastal, 90% humidity)
%   9	Multi-scattering with fixed model (Coastal, 99% humidity)
% 10	Multi-scattering with fixed model (Tropospheric, 50% humidity)
% 11	Multi-scattering with fixed model (Tropospheric, 90% humidity)
% 12	Multi-scattering with fixed model (Tropospheric, 99% humidity)
%   0	Single-scattering white aerosols
% -1	Multi-scattering with 2-band model selection
% -3	Multi-scattering with 2-band model selection and NIR correction
% -9	Multi-scattering with 2-band model selection and SWIR correction(Hi-res MODIS only)

% Calibration control options:
% vcal_opt	Vicarious calibration option controls whether gain and offset sensor defaults or input parameters are used: 
 
% 0 - sensor defaults 
% 1 - default offset, parameter gain 
% 2 - default gain, parameter offset 
% 3 - parameter gain and offset
% gain	Calibration gain factors to multiply TOA radiance in each band; the default gain values are read from the $SDSDATA/sensor/sensor_table.dat file. 
% offset	Calibration offset adjustment to TOA radiance; the default gain values are read from the $SDSDATA/sensor/sensor_table.dat file. 

% aer_opt (int) (default=99) = aerosol mode option
%       -99: No aerosol subtraction
%       >0: Multi-scattering with fixed model (provide model number, 1-N,
%            relative to aermodels list)
%         0: White aerosol extrapolation.
%        -1: Multi-scattering with 2-band model selection
%        -2: Multi-scattering with 2-band, RH-based model selection and
%            iterative NIR correction

% Wonkook's comments:
% Hi Concha,

% Here I attached the first review of the manuscript.

% Overall impression of the manuscript to me is that a great amount of work
% has been done, but the way it is presented can be improved. It would be
% greatly helpful to readers, if you can summarize what you're going to do
% afterward, in the beginning of each paragraph. To me, it was difficult to
% follow the details, because I couldn't catch the
% direction/approach/intention of each paragraph at the beginning.

% Many comments have been made in the attached file, but here I present 4
% important issues that I'd like to share with the other co-authors.

% (1) Weak logic in showing the temporal homogeneity of GCW
% I'm not totally convinced by the overall concept/approach/logic of this
% section (4.3). GOCI Rrs has been used to show that the GCW region has
% little variability in constituents. But, GOCI Rrs is already contaminated
% by imperfect AC (solar zenith angle issue). I think you need independent
% data source to show that the region has little change in water
% constituents.

% (2) Source of Rrs variability
% In the manuscript, the Rrs variability is sometimes attributed to solar
% zenith angle change (and imperfect AC), and sometimes to variability in
% water constituents. (Section 4.4. the second paragraph). One can be
% analyzed when the other is fixed. If solar zenith angle needs to be
% analyzed, you should either assume that the other factor is constant, or
% remove the effect of the other factors.

% Also, if any prior knowledge about the local variability in the
% bio-geochemical environment is to be used, proper reference needs to be
% added.

% (3) Slot boundary issue
% As I pointed out a few months ago, there is a great radiometric inflation
% in the lower part of each slot (GOCI is composed of 16 (4x4) slots, as you
% know). The inflation has a spatially smooth pattern, and I experienced that
% outlier removal approaches based on spatial statistics (including
% coefficient of variation) cannot screen out the pixels contaminated by the
% inflation. Please verify whether your screening process successfully
% removed those samples. The inflation is sometimes greater than 20% in TOA
% radiance (in 680, 865 nm bands), which is large enough to mess up the AC
% process.

% The cloud edge issue has not been mentioned in the manuscript, but  it is
% highly likely the proposed screening process screen out the cloud edge
% pixels which are not recognized as clouds by the default cloud flag. But, I
% think it is safe to check.

% (4) Vicarious gains and the algorithm coefficients.
% The Navy vicarious gains seems very low in general. I attached my
% validation results that I presented 3 years ago in Ocean Optics. Although
% AC scheme is different, this can give you general ideas of the difference.
% I'm not sure if it is a good idea to show that OBPG GOCI has significant
% underestimation in all bands. To non ocean color people, this may seem as
% inferior performance of GOCI itself (including optics, and data
% processing), not as just biases in VC gains. If possible, application of
% correct VC gains would give much better results in the sense of absolute
% quantification. If relative variation in diurnal cycle is a main focus,
% this issue may be less important.

% Plus, the coefficients for the Chla, POC, aCDOM algorithms are not tuned
% for GOCI. Again, this may provide incorrect estimates in an absolute sense.
% If relative change is to be analyzed, this issue may be less important.

% Best,
% Wonkook Kim
