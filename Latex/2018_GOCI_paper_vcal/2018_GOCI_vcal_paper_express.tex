%%%%%%%%%%%%%%%%%%%%%%%%%%%%%%%%%%%%%%%%%%%%%%%%%%%%%%%
%                File: express_style.tex              %
%             Created: 2 September 2009               %
%              Updated: 29 August 2017                %
%                                                     %
%           LaTeX template file for use with          %
%           OSA's journals Optics Express,            %
%             Biomedical Optics Express,              %
%            and Optical Materials Express            %
%                                                     %
%  send comments to Theresa Miller, tmiller@osa.org   %
%                                                     %
%       (c) 2017 Optical Society of America           %
%%%%%%%%%%%%%%%%%%%%%%%%%%%%%%%%%%%%%%%%%%%%%%%%%%%%%%%

\documentclass[10pt]{article}
%% Specify the Express journal you are submitting to
%\usepackage[OME]{express}
\usepackage[OE]{express}
%\usepackage[BOE]{express}

%%% MY PACKAGES %%%%%%%%%%%%%%%%%%%%%%%%%%%%%%%%%%%%%%%%%%%%%%%
\usepackage{graphicx}
% \usepackage[outdir=./]{epstopdf}
\usepackage{epstopdf}
\epstopdfsetup{update} % only regenerate pdf files when eps file is newer
\usepackage{amsmath,epsfig}

% Select what to do with todonotes: 
% \usepackage[disable]{todonotes} % notes not showed
\usepackage[draft]{todonotes}   % notes showed
% \usepackage[textwidth=2.0cm]{todonotes}
\presetkeys{todonotes}{fancyline, size=\scriptsize}{}
\setlength{\marginparwidth}{3cm}

\usepackage{tikz} % for flow charts
  \usetikzlibrary{shapes,arrows,positioning,shadows,calc}
  % \usetikzlibrary{external}
  % \tikzexternalize[prefix=Figures/]

% \usepackage[nostamp]{draftwatermark}
% \SetWatermarkLightness{0.8}
% \SetWatermarkScale{4}

\usepackage[percent]{overpic}
\usepackage{morefloats} % for the error "Too many unprocessed floats"

\usepackage{multirow}

% \renewcommand*{\bibfont}{\normalsize}

\usepackage{float}
\usepackage{hyperref}
\usepackage{pdflscape}

%%% END MY PACKAGES %%%%%%%%%%%%%%%%%%%%%%%%%%%%%%%%%%%%%%%%%%%

\begin{document}
\title{GOCI's Vicarious Calibration for the SeaDAS/l2gen Package}

\author{Javier Concha\authormark{1,2,*}, Antonio Mannino\authormark{1}, Bryan Franz\authormark{1}, Sean Bailey\authormark{1}, and Wonkook Kim\authormark{3}}

\address{\authormark{1}Ocean Ecology Lab,
NASA Goddard Space Flight Center, Greenbelt, MD, USA\\
\authormark{2}Universities Space Research Association, Columbia, MD, USA\\
\authormark{3}Korea Institute of Ocean Science and Technology, Busan, Republic of Korea}

\email{\authormark{*}javier.concha@nasa.gov} %% email address is required

% \homepage{http:...} %% author's URL, if desired

%%%%%%%%%%%%%%%%%%% abstract and OCIS codes %%%%%%%%%%%%%%%%
%% [use \begin{abstract*}...\end{abstract*} if exempt from copyright]

\begin{abstract}
% 100 words

% Background/motivation/context
Accurate retrieval of satellite-derived biogeochemical products depends upon a correct on-orbit calibration. 
% Aim/objectives(s)/problem statement
Two approaches for vicarious calibration for the Geostationary Ocean Color Imager (GOCI) specific for the SeaDAS/l2gen package is presented.
% Methods
One approach is based on concurrent MODIS-Aqua data over an open ocean region with assumed known aerosol properties.
The second approach uses SeaWiFS climatology over the same region.
% Results
A validation using AERONET-OC data shows an improvement the sensor performance for both approaches when compared with the current vicarious gains.
A good agreement is found when compared to MODIS-Aqua and VIIRS.
% Conclusions
%
\end{abstract}
\todo{update codes following Ahn's}
\ocis{(010.4450) Oceanic optics; (280.4991) Passive remote sensing.} % REPLACE WITH CORRECT OCIS CODES FOR YOUR ARTICLE, MINIMUM OF TWO; Avoid using the OCIS codes for “General” or “General science” whenever possible.
%For a complete list of OCIS codes, visit: https://www.osapublishing.org/oe/submit/ocis/

%%%%%%%%%%%%%%%%%%%%%%% References %%%%%%%%%%%%%%%%%%%%%%%%%
% \begin{thebibliography}{99}
 
\bibliographystyle{osajnl}
\bibliography{/Users/jconchas/Documents/Latex/bib/javier_NASA.bib}

% \end{thebibliography}

% \small
% \tableofcontents
% \normalsize

% % Why --  Purpose
% Present two vicarious calibration schemes for GOCI using SeaDAS/l2gen package as an example to be applied to other sensors.
% % Who -- Audience
% GOCI users
% % What -- Content
% Vicarious Calibration
% Validation
% Comparison

% % When -- Time constraints
% One week
% % Where --  Space constraints
% At Goddard
%%%%%%%%%%%%%%%%%%% SECTION %%%%%%%%%%%%%%%%%%%%%%%%%%%%%%%%
\section{Introduction}
% vcal importance within the big picture
Satellite Ocean Color (OC) sensors enable our ability to studying the processes that governed the Earth's marine biosphere, an important component of Earth's system, by capturing the solar energy interaction with the water properties.
These satellite measurements can be then related to environmental properties within certain degree of accuracy and tolerable uncertainties. 
Even though the calibration efforts before launching and on-board (i.e. lunar and solar calibration) are key for obtaining an estimation of environmental properties, they do not secure an optimal performance because systematic bias caused by the atmospheric correction algorithm, and inherent residual error in instrument calibration are not accounted for. 
In order to obtain a well-calibrated satellite data, an extra on-orbit calibration known as vicarious calibration needs to be applied. 
The vicarious calibration is able to adjust the system performance by updating the pre-launch and on-board instrument calibration \cite{Franz:07}. 
This process ensures the accuracy needed to retrieve the rather small water-leaving signal, which constitutes only $~10\%$ of the total signal reaching the sensor.

% GOCI
Geostationary (GEO) OC sensors have the advantage of capturing short-term (hours) changes in ocean properties due to their temporal resolution. 
The high frequency data from GEO OC helps to study these sub-diurnal processes including primary productivity, tidal variability, formation and dissipation of phytoplankton blooms, to name a few \cite{Ruddick2014}. 
The first, and unfortunately the only to date, geostationary satellite dedicated to Ocean Color, the Geostationary Ocean Color Imager (GOCI) \cite{Ryu2012}, was launched in 2010 by the Republic of Korea. It monitors the Northeast Asian water surrounding the Korean peninsula capturing eight images per day with a spatial resolution of 500 m at nadir and eight spectral bands (6 VIS: 412, 443, 490, 555, 660 and 680 nm; 2 NIR: 745 and 865 nm).

% NASA capavilities
The Ocean Biology Distributed Active Archive Center (OB.DAAC) at the NASA's Goddard Space Flight Center, maintained by the Ocean Biology Processing Group (OBPG), acts a mirror-server for the GOCI data, which is provided by the Korea Ocean Satellite Center. In order to have full advantage of these data, the last version the multisensor level 1 level 2 generator ({\ttfamily l2gen}) version distributed with the sea-viewing wide field-of-view sensor (SeaWiFS \cite{McClain2004}) Data Analysis System (SeaDAS) (\url{https://seadas.gsfc.nasa.gov/}) package has the capability of processing GOCI data. The {\ttfamily l2gen} code also includes tools to perform vicarious calibration (described in more details in the \autoref{sec:appendix_a}). 

% vcal definition
The vicarious calibration here follows the approach described in \cite{Franz:07} and is briefly explained here. The vicarious calibration minimizes the average difference between the expected signal from the water and the retrieved signal by applying multiplicative correction factor or gains to the total top of the atmosphere (TOA) retrieved signal (TOA radiance signal or $L_t$) forcing the retrieved water-leaving signal (water-leaving radiance or $L_w$) to looks like the targeted water-leaving signal (${L_w}^t$). 
These gain, $g_i$, for the observation sample $i$ is calculated as \cite{Franz:07}:
\begin{equation}\label{eq:g_i}
  g_i={L_t}^t/L_t
\end{equation}
where ${L_t}^t$ is the targeted total TOA radiance and the $L_t$ is the total sensor-reaching radiance, defined as
\begin{equation}\label{eq:L_t}
  L_t=[L_r+L_a+t_{d_v}(L_f+L_w)]t_{g_v}t_{g_s}f_p
\end{equation}
where $L_w$ is the water-leaving radiance, which is the quantity of interest. The terms $L_r$, $L_a$, and $L_f$ in Eq. (\ref{eq:L_t}) are the contributions from air molecules, aerosols, and white-caps or sea foam respectively. The term $t_{d_v}$ is the diffuse transmittance along the surface-sensor, and $t_{g_v}$ and $t_{g_s}$  are the transmittance accounting for the losses due to gaseous absorption in the sun-surface and surface-sensor radiant paths, respectively. The term $L_p$ is a polarization correction factor. All terms in Eq. (\ref{eq:g_i}) and Eq. (\ref{eq:L_t}) are spectrally dependent, but this dependency was omitted for brevity\cite{Mobley2016}\todo{check!}. The atmospheric correction scheme \cite{Mobley2016} calculates the unknown terms in Eq. (\ref{eq:L_t}) in order to retrieve $L_w$.

% summary
Although the vicarious calibration is independent of the satellite sensor or the source of the targeted data, it is specific to the atmospheric correction algorithm\cite{Franz:07}. 
Vicarious calibration for GOCI has been performed for different atmospheric correction schemes \cite{Ahn2015,Wang:13}.
The primary objective of this work is to provided a vicarious calibration for GOCI specific to the atmospheric correction scheme used by NASA.
The gain for the short near infrared band ($NIR_S$) is calculated using a aerosol model based on Angstrom coefficient from the Moderate Resolution Imaging Spectroradiometer (MODIS) Aqua (MODIS-Aqua) \cite{Esaias1998}.
Then, two different approaches for deriving the gains in the visible bands are tested: one based on MODIS-A data and one based on climatology data from SeaWiFS. The results are validated with data from the AErosol RObotic NETwork-Ocean Color (AERONET-OC)\cite{Zibordi2009} and compared with heritage OC missions.

%%%%%%%%%%%%%%%%%%% SECTION %%%%%%%%%%%%%%%%%%%%%%%%%%%%%%%%
\section{Approach}
The vicarious calibration adopted here is based on the standard method used for calibrating polar-orbiting OC sensors \cite{Franz:07}.
Briefly, the vicarious calibration is the inverse process of the atmospheric correction, i.e. given a targeted normalized water-leaving radiances ${L_{wn}}^t$ as input, the process output a targeted total TOA radiance ${L_t}^t$. 
Then, the gains are calculated using Eq. (\ref{eq:g_i}), with $L_t$ being the satellite data. 
% The components of the vicarious calibration are the satellite measurement $L_t$ and the atmospheric algorithm to retrieve $L_w$ from this $L_t$.
The goal of the vicarious calibration is to minimize the difference between remotely sensed water-leaving radiance retrievals and the expected water-leaving radiance. 
The expected or targeted values could be based on {\it in situ} data, modeled data, regional climatologies or retrievals from another sensor.
In practice this process is divided in two steps.
First, the $NIR$ bands are calibrated, and then the visible bands are calibrated using these calibrated $NIR$ bands. 

% %-------------------sub-section-----------------------------
% \subsection{Calibration Site}
In our case, both steps utilize satellite data over the same calibration site with assumed known atmospheric conditions.
The GOCI Clear Water Subset (GCWS) (Fig. \ref{fig:GOCI_map}) was used as calibration site because it is an open ocean location that is oligotrophic water, and far from land or active volcanics islands, and therefore, the dominant aerosols result from purely maritime processes \cite{Franz:07}.
The GCWS was chosen a subset of the region used in other studies and a subset of GOCI's slot 13 to avoid slot edges distortion \cite{Kim:2015,Kim:2016} and to avoid the difference in acquisition time among slots.
%-%-%-%-%-%-%-%-%-%-%=FIGURE=%-%-%-%-%-%-%-%-%-%-%-%-%
\todo{change figure without GCW box?}
\begin{figure}[H]
  \centering
  \includegraphics[trim=0 0 0 0,width=8cm]{./Figures/GOCI_MAP.eps}
    %\internallinenumbers
    \caption{Map of the GOCI Clear Water Subset (GCWS) region used as the calibration site indicated as a blue box with limits $26^o 5' 45''N$-$28^o 29' 42''N$ and $137^o 20' 16.8''E$-$142^o 5' 31.2''$. The GOCI coverage is indicated as red box and the slot 13 as black box. The GCWS was chosen over a low productivity water region and as a subset of GOCI's slot 13 to avoid slot edges distortion. The GOCI Clear Water (GCW) region, indicated as a magenta box, used in other studies is shown for reference. AERONET-OC stations shown in red circles. \label{fig:GOCI_map}} 
\end{figure}

%-%-%-%-%-%-%-%-%-%-%=END FIGURE=%-%-%-%-%-%-%-%-%-%-%-%-%
%-------------------sub-section-----------------------------
% \subsection{Satellite Data and Processing}

% All the satellite data from the OB.DAAC used in this study (i.e. MODIS-Aqua, VIIRS and GOCI) include the last reprocessing R2018.0, which incorporates advancements in instrument calibration and updates in the instrument-specific vicarious calibration derived from updated MOBY instrument calibration.

% https://seabass.gsfc.nasa.gov/wiki/validation_description

%-------------------sub-section-----------------------------
\subsection{Calibration of the Near-Infrared Bands}
% vcal_opt:
%         1: inverse (calibration) mode, targeting to nLw=0
The main unknown is the vicarious calibration is the aerosol contribution, and a series of assumptions are taken to determine this contribution.
One assumption is that the instrument calibration of the longest NIR wavelength ($NIR_L$) is perfect, and therefore, the vicarious gain for this band is unity ($g_i(NIR_L)=1$). 
Other assumption is that the water contribution in the NIR regime over a predominant oligotrophic region with assumed maritime aerosols is equal to zero, and therefore, the normalized water-leaving radiances in the NIR is equal to zero (i.e. $L_{wn}(NIR_S)=0$ and $L_{wn}(NIR_L)=0$).
The ${L_a}^t(NIR_L)$ can be determined using these two assumptions and the inverse process applied to the satellite data.
Finally, the aerosol model associated with a known aerosol type in combination with the retrieved ${L_a}^t(NIR_L)$ are used to predict the aerosol contribution in the shorter NIR band (${L_a}^t(NIR_S)$) based on $\epsilon$ ratio of the Gordon and Wang algorithm \cite{Gordon1994}.

% - Angstrom for new GCW region: VIIRS, AQUA and SeaWiFS in new GCW region to calculate angstrom AND for time series later. It also will include new R2018.0 processing. Include nLw to be used as target fo r the VIS gain determination

% - 745 gain: Run GOCI with aerosol model derived from AQUA/SeaWiFS angstrom
In practice, the aerosol model can be determined based on the value of the Angstrom coefficient over the calibration site since each of the 80 aerosol models included in the atmospheric correction algorithm have associated an Angstrom coefficient \cite{Mobley2016,Ahmad2010}.
In our case, the Angstrom coefficient was obtained from MODIS-Aqua data over the GCWS region. 
A similar Angstrom coefficient value was obtained from SeaWiFS data. 
The aerosol model used was the r95f10v01 model with an associated Angstrom coefficient equal to 0.852786, which is the closest value to the Angstrom coefficient equal to 0.9 obtained from MODIS-Aqua.\todo{include Angstrom histogram}

Then, a set of vicarious gains for the GOCI's $NIR_S$ band $g_i(NIR_S)$ are calculated by processing the GOCI images using the l2gen code in calibration mode with only the selected aerosol model as input, and over the calibration site (GCWS region).
An exclusion criteria based on \cite{Bailey2006} were applied to the $g_i$ values, and only values that passed the exclusion criteria were used to calculate the mission mean vicarious gain.
Finally, the mission mean vicarious gain for the $NIR_S$ band $\bar{g}(745)$ (Fig. \ref{fig:Gvcal_745}) was calculated as the temporal average using the mean of the semi-interquartile range (MSIQR) to minimize the effect of spurious outliers, as suggested by \cite{Franz:07}.
The calculated vicarious gain for the GOCI's 745 band is equal to 0.9474.
A more detailed explanation focusing on the tools and exclusion criteria used to calculate the Angstrom Coefficient and the vicarious gain for the 745 nm band of GOCI is described in \autoref{sec:appendix_a}.
Now, with the vicarious gain for the NIR bands, $\bar{g}(865)=1.0000$ and $\bar{g}(745)=0.9474$, the atmospheric correction algorithm can be used to determine the aerosol contribution for the rest of visible wavelengths, consequently, their vicarious gains. 

%-%-%-%-%-%-%-%-%-%-%=FIGURE=%-%-%-%-%-%-%-%-%-%-%-%-%
\begin{figure}[H]
  \centering
  \includegraphics[trim=50 0 0 0,width=11cm]{./Figures/Gvcal_745_745.eps}
    %\internallinenumbers
    \caption{Vicarious gains derived for GOCI band at 745 nm based on MODIS-Aqua data spanning the mission lifetime from May 2010 to March 2017. The individual calibration gains (circles) are distributed around the mission mean gain line, which is constant for all time. The filled circles are the gains that passed the quality screening process, with the grey and black fill used to distinguish the cases that fell outside or within the semi-interquartile range (SIQR), respectively. The J, M, and S labels indicate January, May, and September, respectively.  \label{fig:Gvcal_745}} 
\end{figure}
%-%-%-%-%-%-%-%-%-%-%=END FIGURE=%-%-%-%-%-%-%-%-%-%-%-%-%
%-------------------sub-section-----------------------------
\subsection{Calibration of the Visible Bands}
% - VIS gain: run GOCI with SeaWiFS and AQUA nLw target to determine VIS gain
In this step, the targeted normalized water-leaving radiances ${L_{wn}}^t$ need to be determined first in order to retrieve the ${L_t}^t$ later and be able to calculate the vicarious gain $g_i$ (Eq. (\ref{eq:g_i})).
First, the aerosol properties (${L_a}^t$) are retrieved using the Gordon and Wang algorithm \cite{Gordon1994} with the calibrated NIR bands.
Then, the ${L_{wn}}^t$ and the ${L_a}^t$ along with the atmospheric correction algorithm in inverse mode are used to calculate ${L_t}^t$ \cite{Franz:07}, and a set of $g_i$ values are derived for each $L_t$-${L_{wn}}^t$ pair.
An exclusion criteria similar to \cite{Bailey2006} is applied to each $g_i$ value.
Finally, this set of $g_i$ are aggregated via the MISQR to derive the mission-average vicarious gain $\bar{g}$ for each visible wavelength.
A more detailed description of this process and the exclusion criteria is described in \autoref{sec:appendix_b}.
In this work, two sources of ${L_{wn}}^t$ are explored: retrievals from the MODIS-Aqua sensor and a climatology derived from the sea-viewing wide field-of-view sensor (SeaWiFS) \cite{McClain2004}.

%-%-%-%-%-%-%-%-%-%-%-%-%=TABLE=%-%-%-%-%-%-%-%-%-%-%-%-%-
\begin{landscape}
\begin{table}[htbp!]
%\internallinenumbers
\caption{GOCI $\bar{g}$ and standard deviations (in parentheses) calculated using the ${L_{wn}}^t$ from MODIS-Aqua (MODISA) and SeaWIFS climatology. The vicarious gains derived by Wang et al. (2012) \cite{Wang:13} and Ahn et al. (2015) \cite{Ahn2015} were included for comparison. \label{tab:vcal_gains_comp}}

  \centering
    \includegraphics[width=23cm]{./Figures/vcal_gains_comp.pdf}

\end{table}
\end{landscape}
%-%-%-%-%-%-%-%-%-%-%=END TABLE=%-%-%-%-%-%-%-%-%-%-%-%-%-
%-/-/-/-/-/-/-/-/-/-/sub-sub-section/-/-/-/-/-/-/-/-/-/-/-/
\subsubsection{${L_{wn}}^t$ derived from MODIS-Aqua}
  % Apply MODIS-Aqua to accomplish vicarious calibration of GOCI from 2011 through 2015 (or until you have sufficient data points for stable vicarious gains; Bryan does this sound reasonable?) to avoid the recent couple of years where MODIS-A data is more suspect.  This approach would be more similar to using MOBY as described in Franz et al. 2007.  Here you would apply daily mean MODIS Rrs for the GCW that pass through your exclusion criteria rather than MOBY to accomplish the vicarious calibration of GOCI. You can then use VIIRS and AERONET-OC as independent comparisons (validation) of the GOCI Rrs.  You can still show the GOCI comparisons with MODIS-A and VIIRS, just have to indicate that the comparison with MODIS-A is not independent.  Since you already have MODIS data processed, this will be quicker, but I think it’s worth looking at using SeaWiFS too.  
MODIS-Aqua data were used to generate targeted values (i.e. ${L_{wn}}^t$) for the vicarious calibration of GOCI's visible bands.
The calibration was made with the GOCI image closest in time to the MODIS-Aqua image within a three hours time window.
Also, MODIS-Aqua 547 nm band with band shift to 555 nm was used to calibrate the GOCI 555 nm band.
In practice, the gains were obtained by processing the GOCI-MODISA matchups using the l2gen code in calibration (inverse) mode with these MODIS-Aqua-derived ${L_{wn}}^t$ values and the vicarious gains for the NIR band as input (see \autoref{sec:appendix_b} for more details).
Both the MODIS-Aqua and GOCI data were processed over the calibration site GCWS and they have to pass an exclusion criteria similar \cite{Bailey2006}.
One of the exclusion criteria was that at least a third of the area of the calibration site had to be valid pixels to be included in the gain calculation.
\todo{talk about sample size \cite{Franz:07}}
The $\bar{g}$ and individual $g_i$ are plotted as a function of mission time for GOCI's visible bands in Fig. \ref{fig:Gvcal_MA} using ${L_{wn}}^t$ derived from MODIS-Aqua. 

%-/-/-/-/-/-/-/-/-/-/sub-sub-section/-/-/-/-/-/-/-/-/-/-/-/
\subsubsection{${L_{wn}}^t$ derived from SeaWiFS climatology}
% Generate an annual daily climatology of the GOCI clear water region from SeaWiFS observations from 1998 to 2009 (11 complete years) and apply this to compute GOCI vicarious gains as Jeremy did for OCTS and CZCS.  You would match the GOCI time point closest in time with SeaWiFS (noonish).  You can then use MODIS, VIIRS and AERONET-OC as independent comparisons (validation) of the GOCI Rrs from 2011 to 2017. 
An annual daily climatology of the GCWS region from SeaWiFS observations from 1998 to 2009 (11 complete years) was generated in a similar fashion as described in \cite{Werdell:07} and used a targeted values for the vicarious calibration of GOCI's visible bands.
Briefly, the SeaWiFS-derived data were organized by day of year and then binned biweekly resulting in a total of 26 sequential 14-day collections. 
Then, the mean of the semi-interquartile range for each bin was calculated and smoothed using a three-element central moving-average for each bin.
Finally, this time series was expanded to every day of year by applying a cubic spline interpolation.

The $g_i$ values were obtained by processing the GOCI image acquired at 1:00 PM local time with the SeaWiFS-derived ${L_{wn}}^t$ associated with the day of year of the GOCI image.
Again, both the SeaWiFS and GOCI data had to pass the exclusion criteria to be included in the process including the criteria that at least a third of the area need to be valid pixels.
The $\bar{g}$ and individual $g_i$ are plotted as a function of mission time for GOCI's visible bands in Fig. \ref{fig:Gvcal_SW} using ${L_{wn}}^t$ derived from SeaWiFS climatology. 

The $\bar{g}$ values and the standard deviation of $g_i$ within the MISQR are shown in \autoref{tab:vcal_gains_comp} for both approaches. 
The values between the two different approaches are similar with a difference less than $1.6\%$.
The values are similar but not identical to previous estimations for other atmospheric correction schemes \cite{Wang:13,Ahn2015}.
%-%-%-%-%-%-%-%-%-%-%=FIGURE=%-%-%-%-%-%-%-%-%-%-%-%-%
\begin{figure}[H]
  \centering
  \includegraphics[trim=50 0 0 0,width=13cm]{./Figures/Gvcal_MA.pdf}
    %\internallinenumbers
    \caption{Vicarious gains derived for GOCI bands at 412, 443, 490, 555, 660, and 680 nm based on MODIS-Aqua data spanning the mission lifetime from May 2010 to March 2017. Color code same as Fig. \ref{fig:Gvcal_745}.  \label{fig:Gvcal_MA}} 
\end{figure}
%-%-%-%-%-%-%-%-%-%-%=END FIGURE=%-%-%-%-%-%-%-%-%-%-%-%-%
%-%-%-%-%-%-%-%-%-%-%=FIGURE=%-%-%-%-%-%-%-%-%-%-%-%-%
\begin{figure}[H]
  \centering
  \includegraphics[trim=50 0 0 0,width=13cm]{./Figures/Gvcal_SW.pdf}
    %\internallinenumbers
    \caption{Vicarious gains derived for GOCI bands at 412, 443, 490, 555, 660, and 680 nm based on SeaWiFS data spanning the mission lifetime from May 2010 to March 2017. Color code same as Fig. \ref{fig:Gvcal_745}.  \label{fig:Gvcal_SW}} 
\end{figure}

% 412 0.9885-0.9867=0.0018
% 443 0.9703-0.9671=0.0032
% 490 0.9489-0.9373=0.0116
% 555 0.9173-0.9017=0.0156
% 660 0.9167-0.9131=0.0036
% 680 0.9071-0.9017=0.0054
%-%-%-%-%-%-%-%-%-%-%=END FIGURE=%-%-%-%-%-%-%-%-%-%-%-%-%

% jconchas:~/Documents/Research/GOCI/GOCI_ViCal/test$ cat l2gen_test.param 
% ifile=COMS_GOCI_L1B_GA_20121025011640.he5
% ofile1=G20121025011640.OCCAL_valregion 
% sline=4144 
% eline=5142 
% spixl=3169 
% epixl=5167 
% l2prod=default,ag_412_mlrc,poc,angstrom,aot_nnn,sena,senz,sola,solz,brdf,Lw_nnn,nLw_nnn,vgain_vvv
% gain=[1.0,1.0,1.0,1.0,1.0,1.0,1.0,1.0] 
% vcal_nLw=2.17247,1.84507,1.21365,0.30201,0.01827,0.01969,0.00000,0.00000
% vcal_opt=2



% jconchas:~/Documents/Research/GOCI/GOCI_ViCal/test$ cat val_extract_test.param 
% ifile=G20121025011640.OCCAL_valregion 
% ofile=G20121025011640.OCCAL_valregion.o 
% elat=29.4736 
% slat=24.2842 
% slon=131.9067 
% elon=142.3193
% valid_ranges=vgain_=0.5:1.5
% ignore_flags=LAND CLDICE STRAYLIGHT ATMFAIL


% GOCI slot edge area were removed

%%%%%%%%%%%%%%%%%%% SECTION %%%%%%%%%%%%%%%%%%%%%%%%%%%%%%%%
\section{Verification of the calibration}
The stability of the vicarious calibration gains for the GOCI bands are tested using two approaches. First, {\it in situ} data from the AERONET-OC were used as matchups to validate the vicarious calibration and the atmospheric algorithm scheme. Then, GOCI-derived data are compared to heritage sensors (i.e. MODIS-Aqua and VIIRS).
%-------------------sub-section-----------------------------
\subsection{Validation using AERONET-OC}
% - Validation with AERONET-OC matchus: run GOCI for AERONET-OC matchup
The atmospheric correction and vicarious calibration were validated using matchups from {\it in situ} observations of the AErosol RObotic NETwork-Ocean Color (AERONET-OC) \cite{Zibordi2009}. The quality-assurance (QA) level used was level 2.0, which is the highest quality for the AERONET-OC data. The dataset from two stations included in the GOCI's footprint were used for the analysis: Gageocho (N=20; PIs: Jae-Seol Shim and Joo-Hyung Ryu) and Ieodo (N=25; PIs: Young-Je Park and Hak-Yeol You) (shown in Fig. \ref{fig:GOCI_map} as red circles). The AERONET-OC stations are located in two different kind of waters, which is reflected in the scatter plots in Fig. \ref{fig:GOCI_AERO}, with the Ieodo having greater $R_{rs}$ values than Gaeocho for all bands.
% Antonio: since we focus on diurnal variability, the time window for the AERONET validation should be reduced to +/- 1 hour because it doesn't make sense to use a 6 hour time window if our premise is that coastal waters express diurnal variability.  AERONET-OC sites are located in coastal waters, and we don't have knowledge as to their diurnal variability.   

The selection of matchups followed the satellite validation protocols described in \cite{Bailey2006}. GOCI data acquired within a 30 minutes window of the {\it in situ} sampling were considered as potential validation matchups. A $5\times5$ GOCI pixel array is extracted centered at the {\it in situ} sampling location.
% Antonio: why 7x7?  Why not 5x5?  These sites may have spatial variability.
A filtered mean is calculated from these $5\times5$ arrays using in a similar fashion as \cite{Bailey2006}, but with the using the median to filter the data instead of the mean (i.e. $[Median-1.5*\sigma] <  X_i < [Median+1.5*\sigma]$ with $X_i$ the value of the $i^{th}$ pixel and $\sigma$ the standard deviation). A minimum of at least half the total of pixels in the $5\times5$ pixel array, i.e. $25/2\approx12$ pixels, were required to be valid (unflagged) for inclusion of the matchup in the validation analysis. Additionally, a coefficient of variation (CV, filtered mean divided by the filtered standard deviation) for the visible bands 412 to 555 nm and the aerosol optical thickness (AOT) at 864 nm was calculated for each pixel array that passes the exclusion criteria described above, and then, the median value of these coefficients of variation ($\text{Median}[CV]$) was calculated. Finally, the pixel arrays whose $\text{Median}[CV]>0.15$, as suggested by \cite{Bailey2006}, are excluded from the validation analysis.

A validation analysis was performed by comparing the satellite-derived retrievals of products with the {\it in situ} observations based on different statistical parameters (\autoref{tab:val_stats}). These statistical parameters are: the slope and offset of the fitted Reduced Major Axis (RMA) regression line of the form $y=m*x+b$, and its coefficient of determination $R^2$, the root mean squared error (RMSE), the mean, standard deviation, and median of the absolute percentage difference (APD)(MAPD, $\pm$sd APD, and Median APD, respectively), the percentage bias, median ratio of computed filtered mean satellite value ($R_{rs:ret}$) to {\it in situ} measurement ($R_{rs:in}$), and the semi-interquartile range (SIQR) \cite{Bailey2006}. These parameters are defined as:
% \begin{linenomath*}
\begin{equation}
  \text{APD$_n$(\%)}=\left[\frac{\displaystyle \left|R_{rs:in}^1-R_{rs:ret}^1 \right|}{R_{rs:in}^1},\dots,\frac{\displaystyle \left|R_{rs:in}^n-R_{rs:ret}^n \right|}{R_{rs:in}^n},\dots,\frac{\displaystyle \left|R_{rs:in}^N-R_{rs:ret}^N \right|}{R_{rs:in}^N}\right]*100,\ n=1,\dots,N
\end{equation}
% \end{linenomath*}
\noindent where N is the total number of matchups, and $R_{rs:in}^n$ is the $n^{th}$ {\it in situ} $R_{rs}$ and $R_{rs:ret}^n$ is the $n^{th}$ satellite-derived $R_{rs}$.
% \begin{linenomath*}
\begin{equation}
  \text{MAPD(\%)} = \frac{1}{N} \sum_{n=1}^{N} \text{APD$_n$(\%)}
\end{equation}
% \end{linenomath*}
% \begin{linenomath*}
\begin{equation}
  \text{$\pm$ \text{sd} APD(\%)} =  SD\left[\text{APD$_n$(\%)}\right]
\end{equation}
% \end{linenomath*}
\noindent with $SD$ the standard deviation.
% \begin{linenomath*}
\begin{equation}
  \text{\text{Median} APD(\%)} =  Median\left[\text{APD$_n$(\%)}\right]
\end{equation}
% \end{linenomath*}
% \begin{linenomath*}
\begin{equation}
   \text{RMSE} = \sqrt{\frac{\displaystyle \sum_{n=1}^{N} \left(R_{rs:in}^n-R_{rs:ret}^n\right)^2}{N}}
\end{equation}
% \end{linenomath*} 
% \begin{linenomath*}
\begin{equation}
    \text{\% Bias} = \frac{\displaystyle \frac{1}{N}*\sum_{n=1}^N(R_{rs:ret}^n-R_{rs:in}^n)}{\text{Mean}[R_{rs:in}^n]}*100
\end{equation}
% \end{linenomath*}
% \begin{linenomath*}
\begin{equation}
  \text{Median ratio} =  Median\left[\frac{R_{rs:ret}^1}{R_{rs:in}^1},\dots,\frac{R_{rs:ret}^n}{R_{rs:in}^n},\dots,,\frac{R_{rs:ret}^N}{R_{rs:in}^N}\right],\ n=1,\dots,N
\end{equation}
% \end{linenomath*}
% \begin{linenomath*}
\begin{equation}
    \text{SIQR} = \frac{Q_3-Q_1}{2}
\end{equation}
% \end{linenomath*}
\noindent where $Q_3$ and $Q_1$ are the $75^{th}$ and $25^{th}$ percentiles for the ratios of the satellite-derived values to the {\it in situ} measurements.  
Also, two more performance metrics suggested by \cite{Seegers:18} were included: the mean bias and the mean absolute error (MAE). The mean bias is defined as
\begin{equation}
    \text{Mean bias} = \frac{1}{N}*\sum_{n=1}^N(R_{rs:ret}^n-R_{rs:in}^n)
\end{equation}
and the MAE is defined as
\begin{equation}
    \text{MAE} = \frac{1}{N}*\sum_{n=1}^N\left|R_{rs:ret}^n-R_{rs:in}^n\right|
\end{equation}
These metrics are suggested when the data is non-Gaussian, as in this case, and to minimize the influence of outliers \cite{Seegers:18}. Both the mean bias and the MAE have the same units as the variable under consideration.

% Antonio: results show some near-zero values at 412 and 443nm?  clearly, there is a bias where GOCI Rrs is generally lower than AERONET.
% Fig. \ref{fig:GOCI_AERO} shows scatter plots for AERONET-OC data versus GOCI matchups. 
% The data were separated by stations and color coded by the times of the day in order to evaluate the influence of the solar zenith angle in the validation matchups. {}
% The statistics calculated for each time of the day and for all matchups (highlighted in bold cases) are shown in \autoref{tab:val_stats}.
The two approaches for the source of ${L_{wn}}^t$, i.e. from MODIS-Aqua and Climatology from SeaWiFS, were analyzed using these statistical parameters and compared with results without vicarious cal{}ibration (uncalibrated) and the current vicarious calibration included in SeaDAS/l2gen (\autoref{tab:val_stats}).
The current vicarious calibration included in SeaDAS/l2gen are based on matchups with VIIRS and they were derived by the Naval Research Lab (NRL) based on AERONET-OC {\it in situ} data.

A good agreement was found between the retrieved $R_{rs}$ and {\it in situ} observations, with $R^2$ values varying from 0.83 to 0.98 for the 412 to 660 nm GOCI bands for both approaches. 
When both approaches for targeted values for the vicarious calibration of the visible bands are compared, they perform similar in all bands based on the $Median~APD(\%)$ and $RMSE$ values.
When compared with the uncalibrated data, both approaches improve the results for the 412 nm based on the $Median~APD(\%)$, while having similar values for 443 nm and greater values for the 490, 555 and 660 nm bands.
Overall, the two test approaches perform better than the current calibration gains, which is reflected in a smaller mean bias and MAE.
The validation statistics indicate that our results show worse agreement than Ahn et {\it al.} (2015) \cite{Ahn2015} in all visible bands, except the 660 nm band, which is reflected in a larger MAPD, even though the $R^2$ values are slightly larger for this study. 
Also, GOCI $R_{rs}$ is lower than the AERONET-OC values for the 490, 555, and 660 nm bands.


 
% , exceeding the values previously reported by \cite{Ahn2015}
% From Antonio: it's not worthwhile to compare individual statistics metrics from another paper, but rather a group a statistics; our results show better (or worse) agreement than Ahn et al. 2015 based a combination of metrics (R2, RSME, APD, etc.).
%-%-%-%-%-%-%-%-%-%-%-%-%=FIGURE=%-%-%-%-%-%-%-%-%-%-%-%-%
\begin{figure}[htbp!]
  \centering
    \includegraphics[height=9cm]{./Figures/GOCI_AERO.pdf}

    %\internallinenumbers
    \caption{Scatter plots showing the comparison between the satellite-derived GOCI values and AERONET-OC {\it in situ} observations (Gaeocho: circles; Ieodo: triangles) for the uncalibrated data (blue markers) and for vicarious calibration based on MODIS-Aqua (red markers) and SeaWiFS (black markers) data. The dashed black line is the 1:1 line, and the Reduced Major Axis (RMA) regression line is drawn in red. \label{fig:GOCI_AERO} } 
\end{figure}
%-%-%-%-%-%-%-%-%-%-%=END FIGURE=%-%-%-%-%-%-%-%-%-%-%-%-%
% - - - - - - - - - - - - - - - - - - - - - - - - - - - - - - - -
% \subsubsection{Cruises Matchups?}
% wavelength  APD    APD         RMSE    RMSE       R^2     R^2
%             (ours) (Ahn's)     (ours)  (Ahn's)    (ours)  (Ahn's)
% 412         40.1  > 22.3        0.0021 > 0.0015     0.81  >  0.78
% 443         28.0  > 22.0        0.0017 > 0.0013     0.91  >  0.89
% 490         25.7  > 12.7        0.0029 > 0.0013     0.95  >  0.93
% 555         22.8  > 10.4        0.0029 > 0.0015     0.98  >  0.94
% 660         39.3  > 34.7        0.0007 < 0.0008     0.97  >  0.87
%-%-%-%-%-%-%-%-%-%-%-%-%=TABLE=%-%-%-%-%-%-%-%-%-%-%-%-%-
\begin{landscape}
\begin{table}[htbp!]
%\internallinenumbers
\caption{Statistics of the atmospheric correction validation with and without the vicarious calibration. \label{tab:val_stats} }

  \centering
    \includegraphics[width=22cm]{./Figures/val_stats_R2018vcal_All_Final.pdf}

\end{table}
\end{landscape}
%-%-%-%-%-%-%-%-%-%-%=END TABLE=%-%-%-%-%-%-%-%-%-%-%-%-%-

%-------------------sub-section-----------------------------
\subsection{Sensor Cross-comparison}

% - GOCI time serie: run GOCI with new vcal gains
We computed the time series of monthly means for the Visible Infrared Imaging Radiometer Suite (VIIRS)\cite{Wang2014} on board the Suomi National Polar-orbiting Partnership (Suomi NPP) weather satellite and MODIS-Aqua over the same GCWS region for cross-comparison with the time series from GOCI (Fig. \ref{fig:CrossCompAllRrs}). 
These data were filtered following the same previous exclusion criteria and then averaged by month. 
For GOCI, the mean of the three midday values were used.
All the satellite data from the OB.DAAC used in this study (i.e. MODIS-Aqua, VIIRS and GOCI) include the last reprocessing R2018.0, which incorporates advancements in instrument calibration and updates in the instrument-specific vicarious calibration derived from updated MOBY instrument calibration.

Overall, the GOCI-derived $R_{rs}$ follows a similar trend as MODISA and VIIRS, with some differences that vary seasonally, with the largest discrepancies during winter (Fig. \ref{fig:CrossCompAllRrs}.(a)). 
When the ratios among all the three mission are analyzed (Fig. \ref{fig:CrossCompAllRrs}.(b-d)), a consistency is found in all bands except for the red bands.
For the GOCI/MODISA ratio, the mean value fluctuates around one except for the red bands, which fluctuates around 1.6 suggesting that the GOCI's red band are brighter overall. 
For the GOCI/VIIRS, the ratio fluctuates around one for all bands with the 660 nm band having a larger spread overall. 
For the MODISA/VIIRS, the ratio varies close to one as well, except for the 660 nm band, which fluctuates around 0.6. 
GOCI displays a consistent behavior from year to year and a no evident relative drift was found.
%-%-%-%-%-%-%-%-%-%-%=FIGURE=%-%-%-%-%-%-%-%-%-%-%-%-%
\begin{figure}[H]
  \centering
  \includegraphics[width=14cm]{./Figures/CrossCompAllRrs.pdf}
    %\internallinenumbers
    \caption{Cross-comparison with MODIS and VIIRS for all wavelengths. (a) Rrs, (b) GOCI/MODIS-Aqua ratio, (c) GOCI/VIIRS ratio, and (d) MODIS-Aqua/VIIRS ratio. \label{fig:CrossCompAllRrs} } 
\end{figure}
%-%-%-%-%-%-%-%-%-%-%=END FIGURE=%-%-%-%-%-%-%-%-%-%-%-%-%
The satellite-derived products Chlorophyll-{\it a} \cite{OReilly1998_Chl}, the absorption coefficient for chromophoric dissolved organic matter (CDOM) at 412 nm  ($a_g(412)$) \cite{Mannino2014} and particulate organic carbon (POC) \cite{Stramski2008} (Fig. \ref{fig:GOCI_TimeSeriesComp_par}) follow similar seasonal trends.
A good consistency in range of the retrieved values was found, for the most part, for the three missions and for the three different products, and a good consistency in phasing of seasonal cycles. Note that certain discrepancy occur for MODIS-Aqua and GOCI on 2017. This could be caused by problem in the instrument calibration.\todo{correct?}
%-%-%-%-%-%-%-%-%-%-%=FIGURE=%-%-%-%-%-%-%-%-%-%-%-%-%
\begin{figure}[H]
  \centering
  \includegraphics[width=14cm]{./Figures/GOCI_TimeSeriesComp_par.pdf}
    %\internallinenumbers
    \caption{Time Series comparison for GOCI (blue solid line), MODISA (red solid line) and VIIRS (black solid line) for (a) chlor-{\it a}, (b) $a_g(412)$ and (c) POC. Overall, all the products follow a similar pattern. \label{fig:GOCI_TimeSeriesComp_par}} 
\end{figure}
%-%-%-%-%-%-%-%-%-%-%=END FIGURE=%-%-%-%-%-%-%-%-%-%-%-%-%

Fig. \ref{fig:scatterRrs} shows the scatter plots for the $R_{rs}(\lambda)$ cross-comparison for GOCI, MODISA and VIIRS. 
The selection of the data over the GCWS region followed similar procedures to the ones described by \cite{Bailey2006} and only data that passed the exclusion criteria are used. 
These data are daily values and only values greater than zero are shown. 
There are significant fewer matchups from MODIS than from VIIRS.
The $R^2$ values are high for the 412 and 443 nm bands, and start to decrease for 490 nm and beyond. 
%-%-%-%-%-%-%-%-%-%-%=FIGURE=%-%-%-%-%-%-%-%-%-%-%-%-%
\begin{figure}[H]
  \centering
  \includegraphics[width=15cm]{./Figures/scatterRrs.pdf}
    %\internallinenumbers
    \caption{Scatter plots for the $R_{rs}(\lambda)$ cross-comparison for GOCI, MODIS-Aqua and VIIRS. Linear regression in solid red line. \label{fig:scatterRrs} } 
\end{figure}
%-%-%-%-%-%-%-%-%-%-%=END FIGURE=%-%-%-%-%-%-%-%-%-%-%-%-%

%%%%%%%%%%%%%%%%%%% SECTION %%%%%%%%%%%%%%%%%%%%%%%%%%%%%%%%
\section{Conclusions}
% Practical applications
% Disadvantages and Advantages
% Limitations
% Challenges

% General
% final conclusion

% ----------------------------
% 1 Introduction 2
% GOCI provides capability of studying coastal and ocean processes at an unprecedented temporal scale.
Two approaches for the vicarious calibration for GOCI are evaluated in this study.
This calibration is particular to the atmospheric correction scheme included in the SeaDAS/l2gen package developed by the Ocean Biology Processing Group (OBPG) at the NASA Goddard Space Flight Center.

% 2 Approach 3
% vicarious calibration
% 2.1 CalibrationoftheNear-InfraredBands............................ 4
The vicarious calibration was separated in two stages following the NASA's standard protocols \cite{Franz:07}.
First, the NIR bands were calibrated, and then, the visible bands were calibrated with the NIR calibration fixed.
% 2.2 CalibrationoftheVisibleBands ............................... 5 
% 2.2.1 LwntderivedfromMODIS-Aqua .......................... 7 
% 2.2.2 LwntderivedfromSeaWiFSclimatology...................... 7
The gain for the 745 nm band was derived using a fixed aerosol model chosen based on the Angstrom Coefficient derived from MODIS-Aqua data over the calibration site.
This derived gain is almost identical to the one derived by \cite{Wang:13}.
For the derivation of the vicarious gains for the visible bands, two sources for the targeted water-leaving radiances ${L_{wn}}^t$ were used: matchups from MODIS-Aqua and climatology data from SeaWiFS.
The vicarious calibration gains are similar for both approaches, and they are similar but not identical to previous studies (\autoref{tab:vcal_gains_comp}).
These approaches are options when few or none {\it in situ} are available.
The use of climatology for the calibration of the visible bands yields similar results when MODIS-Aqua matchups are used, and therefore, this proves to be an alternative when there is lack of matchups from well-calibrated sensor data.

% 3 Verification of the calibration 9
% 3.1 ValidationusingAERONET-OC ............................... 10 
The derived vicarious gains for both approaches were validated using {\it in situ} data from the AERONET-OC data over two stations within GOCI's footprint.
The mean bias and the mean absolute error (MAE) are smaller for both approaches compared with the current vicarious gains included in the SeaDAS/l2gen package, which proved to be an improvement in the sensor performance.
% The validation with {\it in situ} data exhibit results comparable to heritage satellite sensors (Fig. \ref{fig:GOCI_AERO}).
A good agreement was found between the GOCI $R_{rs}$ retrievals and {\it in situ} data for both approaches, reflected in $R^2$ values varying from 0.83 to 0.98 for the 412 to 660 nm bands.
When the atmospheric correction scheme and vicarious calibration are compared in a statistical sense with the scheme used by \cite{Ahn2015}, our results agree less with the AERONET-OC dataset. However, a direct comparison is not possible because \cite{Ahn2015} used different kind of data from the AERONET-OC.

% 3.2 SensorCross-comparison................................... 13
Lastly, monthly time series for GOCI-derived products with the vicarious calibration from MODIS-Aqua were cross-compared with the time series for the heritage sensors MODIS-Aqua and VIIRS (Fig. \ref{fig:CrossCompAllRrs} and Fig. \ref{fig:GOCI_TimeSeriesComp_par}).
% from Bryan's ppt
Some seasonally varying differences in $R_{rs}$ are observed, with a larger discrepancy for the red bands. 
The mission present a consistent behavior from year to year without a evident relative drift.
The time series for the mission to mission ratios vary seasonally, fluctuating around a value close to one, with the exception of the red bands.
Based on scatter plots (Fig. \ref{fig:scatterRrs}) for the cross-comparison with the heritage missions, the differences vary spectrally, being smaller in the blue and increasing towards the red. 


% 4 Conclusions 16 Appendix 16
% A Calculation Vicarious Gains for the NIR Bands 16
% B Calculation Vicarious Gains for the Visible Bands 18
%%%%%%%%%%%%%%%%%%% SECTION %%%%%%%%%%%%%%%%%%%%%%%%%%%%%%%%
\section*{Funding}
NASA Project ROSES Earth Science U.S. Participating Investigator (NNH12ZDA001N-ESUSPI)
%%%%%%%%%%%%%%%%%%% SECTION %%%%%%%%%%%%%%%%%%%%%%%%%%%%%%%%
\section*{Acknowledgments}
We want to acknowledge the Korea Ocean Satellite Center for providing the GOCI L1B data to OBPG, and the Ocean Biology Processing Group at the Goddard Space Flight Center, NASA. Also, the principal investigator for the AERONET-OC data: Jae-Seol Shim and Joo-Hyung Ryu (Gageocho station), Young-Je Park and Hak-Yeol You (Ieodo station).


\appendix
\section*{Appendix}
\addcontentsline{toc}{section}{Appendix}
%%%%%%%%%%%%%%%%%%% SECTION %%%%%%%%%%%%%%%%%%%%%%%%%%%%%%%%
\section{Calculation Vicarious Gains for the NIR Bands}\label{sec:appendix_a}
The satellite validation matchup tool {\ttfamily val\_extract} (included in source code for l2gen) was used for the statistic calculations from the L2 files. The {\ttfamily val\_extract} tool can be applied to an specific region, and it can ignored flagged pixels and a valid range can be specified.

For the 745 band calibration, at least a third of the region needed to valid to pass. %(there is not median CV for GOCI data used to calculated 745, i.e. not CV for the 400-555 and AOT 865 since not valid values for this products).

A CV lim 0.25 was used for the 745 nm band.

l2gen

lonlat2pixline

val\_extract
% \addcontentsline{toc}{section}{Appendix A: Calculation Vicarious Gains for the NIR Bands}
%-%-%-%-%-%-%-%-%-%-%= FIGURE =%-%-%-%-%-%-%-%-%-%-%-%-%
\begin{figure}[H]
  \centering
  \includegraphics[trim=50 0 0 0,width=15cm]{./Figures/angstrom_cal.pdf}
    %\internallinenumbers
    \caption{Angstrom coefficient determination.  \label{fig:angstrom_cal}} 
\end{figure}
%-%-%-%-%-%-%-%-%-%-%=END FIGURE=%-%-%-%-%-%-%-%-%-%-%-%-%
%-%-%-%-%-%-%-%-%-%-%=FIGURE=%-%-%-%-%-%-%-%-%-%-%-%-%-%-%
\begin{figure}[H]
  \centering
  \includegraphics[trim=50 0 0 0,width=15cm]{./Figures/chart_vcal_745.pdf}
    %\internallinenumbers
    \caption{Calibration of the 745 nm band.  \label{fig:chart_vcal_745}} 
\end{figure}
%-%-%-%-%-%-%-%-%-%-%=END FIGURE=%-%-%-%-%-%-%-%-%-%-%-%-%
%%%%%%%%%%%%%%%%%%% SECTION %%%%%%%%%%%%%%%%%%%%%%%%%%%%%%
\section{Calculation Vicarious Gains for the Visible Bands}\label{sec:appendix_b}
%-%-%-%-%-%-%-%-%-%-%=FIGURE=%-%-%-%-%-%-%-%-%-%-%-%-%-%-%
\begin{figure}[H]
  \centering
  \includegraphics[trim=50 0 0 0,width=15cm]{./Figures/chart_vcal_vis.pdf}
    %\internallinenumbers
    \caption{Calibration of the visible (VIS) bands.  \label{fig:chart_vcal_vis}} 
\end{figure}
%-%-%-%-%-%-%-%-%-%-%=END FIGURE=%-%-%-%-%-%-%-%-%-%-%-%-%

\end{document}


%------------------------------------------------------------------------
% Sean's comments:

% The mode and vcal_opt options are effectively interchangeable, and the
% definition of them is as l2gen reports for mode:

%    mode (int) (default=0) = processing mode
%         0: forward processing
%         1: inverse (calibration) mode, targeting to nLw=0
%         2: inverse (calibration) mode, given nLw target
%         3: inverse (calibration) mode, given Lw target (internally normalized)

% Javier,

% The 'trick' is to find a reasonably stable body of water that can be assumed to
% have zero water-leaving radiance in the NIR (i.e. clear, deep ocean) and
% preferably one where an assumption of the aerosol type can be made.  It is best
% if the type is primarily maritime, but whatever it is should be non-absorbing
% and within our available model suite.

% If you have a time series of the angstrom exponent for the region you've been
% using for the visible gain calculation (from SeaWiFS or MODIS, NOT GOCI - as
% that is to be considered suspect until verified), you can use that to choose
% the model.  You *could* let the model vary based on a climatology of angstrom,
% but that might complicate the process too much...

% With the know model, you set aermodels=<my favorite model> and aer_opt=1,
% vcal_opt=1 and voila!  vgain_745 can be achieved.

% Sean
%------------------------------------------------------------------------
% aer_opt	Option for aerosol calculation mode.  (Default=-3) 
 
%   1	Multi-scattering with fixed model (Oceanic, 99% humidity)
%   2	Multi-scattering with fixed model (Maritime, 50% humidity)
%   3 	Multi-scattering with fixed model (Maritime, 70% humidity)
%   4	Multi-scattering with fixed model (Maritime, 90% humidity)
%   5	Multi-scattering with fixed model (Maritime, 99% humidity)
%   6	Multi-scattering with fixed model (Coastal, 50% humidity)
%   7	Multi-scattering with fixed model (Coastal, 70% humidity)
%   8 	Multi-scattering with fixed model (Coastal, 90% humidity)
%   9	Multi-scattering with fixed model (Coastal, 99% humidity)
% 10	Multi-scattering with fixed model (Tropospheric, 50% humidity)
% 11	Multi-scattering with fixed model (Tropospheric, 90% humidity)
% 12	Multi-scattering with fixed model (Tropospheric, 99% humidity)
%   0	Single-scattering white aerosols
% -1	Multi-scattering with 2-band model selection
% -3	Multi-scattering with 2-band model selection and NIR correction
% -9	Multi-scattering with 2-band model selection and SWIR correction(Hi-res MODIS only)

%------------------------------------------------------------------------
% Calibration control options:
% vcal_opt	Vicarious calibration option controls whether gain and offset sensor defaults or input parameters are used: 
 
% 0 - sensor defaults 
% 1 - default offset, parameter gain 
% 2 - default gain, parameter offset 
% 3 - parameter gain and offset
% gain	Calibration gain factors to multiply TOA radiance in each band; the default gain values are read from the $SDSDATA/sensor/sensor_table.dat file. 
% offset	Calibration offset adjustment to TOA radiance; the default gain values are read from the $SDSDATA/sensor/sensor_table.dat file. 

% aer_opt (int) (default=99) = aerosol mode option
%       -99: No aerosol subtraction
%       >0: Multi-scattering with fixed model (provide model number, 1-N,
%            relative to aermodels list)
%         0: White aerosol extrapolation.
%        -1: Multi-scattering with 2-band model selection
%        -2: Multi-scattering with 2-band, RH-based model selection and
%            iterative NIR correction

%------------------------------------------------------------------------

% Hi Concha,
 
% I thank you for sharing the poster. Here are my comments on the poster. 
 
% Best regards,
% Wonkook Kim
 
 
% (1) GOCI L1B areas affected by inter-slot discontinuity
 
% As can be observed in your figure too, there is a discontinuity between slots, and Rrs anomalies in the area can be great particularly for Band 1, 2, and 6, and 8.
 
% IF you do not want to include the sensor calibration and the stray light issues in your evaluation, please refer to the following guidelines.
% - To avoid sensor calibration : avoid at least 50 pixels from the slot boundary, in both directions (Band 1, and 2)
% - To avoid stray light anomaly : avoid 400 pixels from the bottom of each slot (Band 5, 6, and 8)
 
% (2) Availability of diurnal signal in the study area in "winter"
% I see the area is quite full of cloud particularly in winter season and it's very difficult to find pixels having a full-diurnal cycle without cloud cover. Providing the statistics of available pixels will give readers insight on which season is mostly used for your statistics.
 
% Plus, a single-band or two-band thresholding may not be sufficient to screen all the cloud edge pixels in that area because of time difference between bands for a pixel. (refer to Wayne Robinson's IJRS paper) So, please make sure whether the all cloud-contaminated pixels are removed from the analysis.
 
% (3) SRF difference between different sensors
% There is SRF difference between sensors, even for the bands with the same center wavelength. Difference between 660 and 665 needs to be resolved in some way, not to give an impression that the inter-satellite difference is from AC and sensor problem.
 
% (4) Conclusion
% In Conclusion, you have this sentence
% "The atmospheric correction starts to fail for solar zenith angles larger than 60 degrees producing invalid values (negative)."
 
% Do all bands have negative values in all pixels? Or you mean the chance of having negative values increases for th>60? The sentence is not clear to me.
%------------------------------------------------------------------------
% Wonkook's comments:
% Hi Concha,

% Here I attached the first review of the manuscript.

% Overall impression of the manuscript to me is that a great amount of work
% has been done, but the way it is presented can be improved. It would be
% greatly helpful to readers, if you can summarize what you're going to do
% afterward, in the beginning of each paragraph. To me, it was difficult to
% follow the details, because I couldn't catch the
% direction/approach/intention of each paragraph at the beginning.

% Many comments have been made in the attached file, but here I present 4
% important issues that I'd like to share with the other co-authors.

% (1) Weak logic in showing the temporal homogeneity of GCW
% I'm not totally convinced by the overall concept/approach/logic of this
% section (4.3). GOCI Rrs has been used to show that the GCWS region has
% little variability in constituents. But, GOCI Rrs is already contaminated
% by imperfect AC (solar zenith angle issue). I think you need independent
% data source to show that the region has little change in water
% constituents.

% (2) Source of Rrs variability
% In the manuscript, the Rrs variability is sometimes attributed to solar
% zenith angle change (and imperfect AC), and sometimes to variability in
% water constituents. (Section 4.4. the second paragraph). One can be
% analyzed when the other is fixed. If solar zenith angle needs to be
% analyzed, you should either assume that the other factor is constant, or
% remove the effect of the other factors.

% Also, if any prior knowledge about the local variability in the
% bio-geochemical environment is to be used, proper reference needs to be
% added.

% (3) Slot boundary issue
% As I pointed out a few months ago, there is a great radiometric inflation
% in the lower part of each slot (GOCI is composed of 16 (4x4) slots, as you
% know). The inflation has a spatially smooth pattern, and I experienced that
% outlier removal approaches based on spatial statistics (including
% coefficient of variation) cannot screen out the pixels contaminated by the
% inflation. Please verify whether your screening process successfully
% removed those samples. The inflation is sometimes greater than 20% in TOA
% radiance (in 680, 865 nm bands), which is large enough to mess up the AC
% process.

% The cloud edge issue has not been mentioned in the manuscript, but  it is
% highly likely the proposed screening process screen out the cloud edge
% pixels which are not recognized as clouds by the default cloud flag. But, I
% think it is safe to check.

% (4) Vicarious gains and the algorithm coefficients.
% The Navy vicarious gains seems very low in general. I attached my
% validation results that I presented 3 years ago in Ocean Optics. Although
% AC scheme is different, this can give you general ideas of the difference.
% I'm not sure if it is a good idea to show that OBPG GOCI has significant
% underestimation in all bands. To non ocean color people, this may seem as
% inferior performance of GOCI itself (including optics, and data
% processing), not as just biases in VC gains. If possible, application of
% correct VC gains would give much better results in the sense of absolute
% quantification. If relative variation in diurnal cycle is a main focus,
% this issue may be less important.

% Plus, the coefficients for the Chla, POC, aCDOM algorithms are not tuned
% for GOCI. Again, this may provide incorrect estimates in an absolute sense.
% If relative change is to be analyzed, this issue may be less important.

% Best,
% Wonkook Kim
