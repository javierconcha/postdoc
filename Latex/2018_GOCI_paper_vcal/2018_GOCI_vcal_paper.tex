% \documentclass[twocolumn,3p]{elsarticle}
\documentclass[onecolumn,3p,letterpaper,11pt]{elsarticle}
\usepackage{setspace} % added by JC 
\doublespacing % added by JC


\usepackage{lineno} % uncomment \linenumbers after \begin{document}
\modulolinenumbers[1]


  
%%% MY PACKAGES %%%%%%%%%%%%%%%%%%%%%%%%%%%%%%%%%%%%%%%%%%%%%%%
\usepackage{graphicx}
% \usepackage[outdir=./]{epstopdf}
\usepackage{epstopdf}
\epstopdfsetup{update} % only regenerate pdf files when eps file is newer
\usepackage{amsmath,epsfig}

% Select what to do with todonotes: 
\usepackage[disable]{todonotes} % notes not showed
% \usepackage[draft]{todonotes}   % notes showed
% \usepackage[textwidth=2.0cm]{todonotes}
\presetkeys{todonotes}{fancyline, size=\scriptsize}{}
\setlength{\marginparwidth}{3cm}

\usepackage{tikz} % for flow charts
  \usetikzlibrary{shapes,arrows,positioning,shadows,calc}
  % \usetikzlibrary{external}
  % \tikzexternalize[prefix=Figures/]

% \usepackage[nostamp]{draftwatermark}
% \SetWatermarkLightness{0.8}
% \SetWatermarkScale{4}

\usepackage[percent]{overpic}
\usepackage{morefloats} % for the error "Too many unprocessed floats"

\usepackage{multirow}

\renewcommand*{\bibfont}{\normalsize}

\usepackage{float}
\usepackage{hyperref}
\usepackage{pdflscape}
%%% END MY PACKAGES %%%%%%%%%%%%%%%%%%%%%%%%%%%%%%%%%%%%%%%%%%%

\journal{Remote Sensing of Environment}

\begin{document}

% \linenumbers

\begin{frontmatter}

\title{Vicarious Calibration of the Geostationary Ocean Color Imager (GOCI)}

% %% Group authors per affiliation:
% \author{Javier A. Concha\fnref{myfootnote}}
% \address{Radarweg 29, Amsterdam}
% \fntext[myfootnote]{Since 1880.}

%% or include affiliations in footnotes:
\author[oeladdress,usraaddress]{Javier Concha\corref{mycorrespondingauthor}}
\cortext[mycorrespondingauthor]{Corresponding author at: Ocean Ecology Lab,
NASA Goddard Space Flight Center,
8800 Greenbelt Rd, Greenbelt, MD 20771, USA. Tel.: +1 585 290 3145.}
\ead{javier.concha@nasa.gov}

\author[oeladdress]{Antonio Mannino}

\author[oeladdress]{Bryan Franz}

% \author[oeladdress]{Amir Ibrahim}

% \author[kiostaddress]{Wonkook Kim}

% \author[usgsaddress]{Michael Ondrusek}

\address[oeladdress]{Ocean Ecology Lab, NASA Goddard Space Flight Center, Greenbelt, MD, USA}
\address[usraaddress]{Universities Space Research Association, Columbia, MD, USA}
% \address[kiostaddress]{Korea Institute of Ocean Science and Technology, 787 Haean-ro, Ansan, Republic of Korea}

% \address[usgsaddress]{NOAA/NESDIS Center for Weather and Climate Prediction, College Park, Maryland, USA}
% ===============================================================
\begin{abstract}

% Background/motivation/context

%
 
% Aim/objectives(s)/problem statement

% 

% Methods

%
 
%

%

%

% Results

%
 
% Conclusions

%


\end{abstract}

\begin{keyword}
Geostationary Ocean Color Imager\sep GOCI\sep Vicarious Calibration
\end{keyword}

\end{frontmatter}
%%%%%%%%%%%%%%%%%%% SECTION %%%%%%%%%%%%%%%%%%%%%%%%%%%%%%%%
\singlespacing
\small
\tableofcontents
\normalsize
\doublespacing
%%%%%%%%%%%%%%%%%%% SECTION %%%%%%%%%%%%%%%%%%%%%%%%%%%%%%%%
\section{Introduction}

%%%%%%%%%%%%%%%%%%% SECTION %%%%%%%%%%%%%%%%%%%%%%%%%%%%%%%%
\section{Methodology}
\subsection{NIR Bands Vicarious Calibration}
The aerosol model used was r95f10v01 ($\alpha=0.852786$ from the 80 standard models) for an angstrom coefficient ($\alpha$) equal to 0.9.

\subsection{VIS Bands Vicarious Calibration}
% jconchas:~/Documents/Research/GOCI/GOCI_ViCal/test$ cat l2gen_test.param 
% ifile=COMS_GOCI_L1B_GA_20121025011640.he5
% ofile1=G20121025011640.OCCAL_valregion 
% sline=4144 
% eline=5142 
% spixl=3169 
% epixl=5167 
% l2prod=default,ag_412_mlrc,poc,angstrom,aot_nnn,sena,senz,sola,solz,brdf,Lw_nnn,nLw_nnn,vgain_vvv
% gain=[1.0,1.0,1.0,1.0,1.0,1.0,1.0,1.0] 
% vcal_nLw=2.17247,1.84507,1.21365,0.30201,0.01827,0.01969,0.00000,0.00000
% vcal_opt=2



% jconchas:~/Documents/Research/GOCI/GOCI_ViCal/test$ cat val_extract_test.param 
% ifile=G20121025011640.OCCAL_valregion 
% ofile=G20121025011640.OCCAL_valregion.o 
% elat=29.4736 
% slat=24.2842 
% slon=131.9067 
% elon=142.3193
% valid_ranges=vgain_=0.5:1.5
% ignore_flags=LAND CLDICE STRAYLIGHT ATMFAIL


% GOCI slot edge area were removed
% • 400 pixels from bottom, 150 pixels from left, right, and top end

\end{document}

% Wonkook's comments:
% Hi Concha,

% Here I attached the first review of the manuscript.

% Overall impression of the manuscript to me is that a great amount of work
% has been done, but the way it is presented can be improved. It would be
% greatly helpful to readers, if you can summarize what you're going to do
% afterward, in the beginning of each paragraph. To me, it was difficult to
% follow the details, because I couldn't catch the
% direction/approach/intention of each paragraph at the beginning.

% Many comments have been made in the attached file, but here I present 4
% important issues that I'd like to share with the other co-authors.

% (1) Weak logic in showing the temporal homogeneity of GCW
% I'm not totally convinced by the overall concept/approach/logic of this
% section (4.3). GOCI Rrs has been used to show that the GCW region has
% little variability in constituents. But, GOCI Rrs is already contaminated
% by imperfect AC (solar zenith angle issue). I think you need independent
% data source to show that the region has little change in water
% constituents.

% (2) Source of Rrs variability
% In the manuscript, the Rrs variability is sometimes attributed to solar
% zenith angle change (and imperfect AC), and sometimes to variability in
% water constituents. (Section 4.4. the second paragraph). One can be
% analyzed when the other is fixed. If solar zenith angle needs to be
% analyzed, you should either assume that the other factor is constant, or
% remove the effect of the other factors.

% Also, if any prior knowledge about the local variability in the
% bio-geochemical environment is to be used, proper reference needs to be
% added.

% (3) Slot boundary issue
% As I pointed out a few months ago, there is a great radiometric inflation
% in the lower part of each slot (GOCI is composed of 16 (4x4) slots, as you
% know). The inflation has a spatially smooth pattern, and I experienced that
% outlier removal approaches based on spatial statistics (including
% coefficient of variation) cannot screen out the pixels contaminated by the
% inflation. Please verify whether your screening process successfully
% removed those samples. The inflation is sometimes greater than 20% in TOA
% radiance (in 680, 865 nm bands), which is large enough to mess up the AC
% process.

% The cloud edge issue has not been mentioned in the manuscript, but  it is
% highly likely the proposed screening process screen out the cloud edge
% pixels which are not recognized as clouds by the default cloud flag. But, I
% think it is safe to check.

% (4) Vicarious gains and the algorithm coefficients.
% The Navy vicarious gains seems very low in general. I attached my
% validation results that I presented 3 years ago in Ocean Optics. Although
% AC scheme is different, this can give you general ideas of the difference.
% I'm not sure if it is a good idea to show that OBPG GOCI has significant
% underestimation in all bands. To non ocean color people, this may seem as
% inferior performance of GOCI itself (including optics, and data
% processing), not as just biases in VC gains. If possible, application of
% correct VC gains would give much better results in the sense of absolute
% quantification. If relative variation in diurnal cycle is a main focus,
% this issue may be less important.

% Plus, the coefficients for the Chla, POC, aCDOM algorithms are not tuned
% for GOCI. Again, this may provide incorrect estimates in an absolute sense.
% If relative change is to be analyzed, this issue may be less important.

% Best,
% Wonkook Kim
