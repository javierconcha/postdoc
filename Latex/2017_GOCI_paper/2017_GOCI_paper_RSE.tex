% \documentclass[twocolumn,3p]{elsarticle}
\documentclass[onecolumn,3p,letterpaper,11pt]{elsarticle}
\usepackage{setspace} % added by JC 
\doublespacing % added by JC


\usepackage{lineno} % uncomment \linenumbers after \begin{document}
\modulolinenumbers[1]

\usepackage{hyperref}

%%% MY PACKAGES %%%%%%%%%%%%%%%%%%%%%%%%%%%%%%%%%%%%%%%%%%%%%%%
\usepackage{graphicx}
% \usepackage[outdir=./]{epstopdf}
\usepackage{epstopdf}
\epstopdfsetup{update} % only regenerate pdf files when eps file is newer
\usepackage{amsmath,epsfig}

% Select what to do with todonotes: 
% \usepackage[disable]{todonotes} % notes not showed
% \usepackage[draft]{todonotes}   % notes showed
\usepackage[textwidth=2.5cm]{todonotes}
\presetkeys{todonotes}{fancyline, size=\small}{}
\setlength{\marginparwidth}{2.5cm}

\usepackage{tikz} % for flow charts
  \usetikzlibrary{shapes,arrows,positioning,shadows,calc}
  \usetikzlibrary{external}
  % \tikzexternalize

% \usepackage[nostamp]{draftwatermark}
% \SetWatermarkLightness{0.8}
% \SetWatermarkScale{4}

\usepackage[percent]{overpic}
\usepackage{morefloats} % for the error "Too many unprocessed floats"

\usepackage{multirow}

\renewcommand*{\bibfont}{\normalsize}
%%% END MY PACKAGES %%%%%%%%%%%%%%%%%%%%%%%%%%%%%%%%%%%%%%%%%%%

\journal{Remote Sensing of Environment}

\begin{document}

% \linenumbers

\begin{frontmatter}

\title{Assessing Diurnal Variability of Biogeochemical Processes using the Geostationary Ocean Color Imager (GOCI)}

% %% Group authors per affiliation:
% \author{Javier A. Concha\fnref{myfootnote}}
% \address{Radarweg 29, Amsterdam}
% \fntext[myfootnote]{Since 1880.}

%% or include affiliations in footnotes:
\author[oeladdress,usraaddress]{Javier Concha\corref{mycorrespondingauthor}}
\cortext[mycorrespondingauthor]{Corresponding author at: Ocean Ecology Lab,
NASA Goddard Space Flight Center,
8800 Greenbelt Rd, Greenbelt, MD 20771, USA. Tel.: +1 585 290 3145.}
\ead{javier.concha@nasa.gov}

\author[oeladdress]{Antonio Mannino}

\author[oeladdress]{Bryan Franz}

\author[kiostaddress]{Wonkook Kim}

\author[usgsaddress]{Michael Ondrusek}

\address[oeladdress]{Ocean Ecology Lab, NASA Goddard Space Flight Center, Greenbelt, MD, USA}
\address[usraaddress]{Universities Space Research Association, Columbia, MD, USA}
\address[kiostaddress]{Korea Institute of Ocean Science and Technology, 787 Haean-ro, Ansan, Republic of Korea}

\address[usgsaddress]{NOAA/NESDIS Center for Weather and Climate Prediction, College Park, Maryland, USA}
% ===============================================================
\begin{abstract}

% Background/motivation/context
The GOCI sensor, the first ever geostationary satellite dedicated for ocean color, presents a unique opportunity to investigate diurnal dynamics by taking advantage of its eight images per day.
%

% Aim/objectives(s)/problem statement
GOCI observations allow us to test NASA’s standard atmospheric correction approach over the same locations on the earth for a range of solar geometries.
%
The SeaDAS/l2gen package (NASA’s ocean color satellite data processing software) has recently added the capability to handle GOCI data and produce ocean color standard products.
% Methods

%

%
Because the errors introduced by the atmospheric correction could contribute to an over or underestimation of the changes in these short-term processes, there is a need to have an estimation of the uncertainties introduced by the atmospheric correction.
%
NASA’s standard atmospheric correction approach included in the SeaDAS/l2gen package is tested.
%
Application of GOCI products to study diurnal and day-to-day biogeochemical stocks and processes.
% Results

%
 
%

% Conclusions

%

\end{abstract}

\begin{keyword}
Geostationary Ocean Color Imager\sep GOCI\sep diurnal dynamics\sep diurnal variability\sep CDOM
\end{keyword}

\end{frontmatter}
%%%%%%%%%%%%%%%%%%% SECTION %%%%%%%%%%%%%%%%%%%%%%%%%%%%%%%%
\section{Introduction}
GOCI as a precursor to GEO-CAPE. 
Name GOCI-II and the ESA geostationary satellite.

Spinning Enhanced Visible and Infrared Imager (SEVIRI) \citet{Neukermans2009,Neukermans2012}. Geostationary review \citet{Ruddick2014}. Synergy \citet{Vanhellemont2014}.

Example of citations \citet{Ryu2011,He2013,Hu2016}

Biological and biogeochemical processes in the coastal zones and oceans can occur in short-term time scales (minutes to hours). Heritage ocean color sensors do not have the temporal resolution needed to capture these diurnal dynamics.
%%%%%%%%%%%%%%%%%%%%% SECTION %%%%%%%%%%%%%%%%%%%%%%%%%%%%%%%%
% \section{Data and Methodology}
\section{Data and Sensor Characteristics}
% ============================================================
\subsection{GOCI Data}
The Geostationary Ocean Color Imager (GOCI), launched in June 2010, is the only geostationary ocean color sensor currently on space \citep{Ryu2012}. GOCI monitors the Northeast Asian waters surrounding the Korean peninsula, generating up to eight images per day (from 00:15 GMT to 07:45 GMT) with a spatial resolution of 500 m at 130$^o$E and 36$^o$N. It covers an area of about 2500 km$\times$2500 km. It hast eight spectral bands (6 VIS: 412, 443, 490, 555, 660 and 680 nm; 2 NIR: 745 and 865 nm). GOCI operates in a 2D staring-frame capture mode in a geostationary orbit on board of the Communication Ocean and Meteorological Satellite (COMS) of South Korea. The acquisition of the observational coverage area of GOCI is done with a step-and-staring method that divides the image in 16 slots acquired in a sequential fashion with a dedicated CMOS detector array ($1400\times1400$ pixels).

The images used in this analysis comprehend from the beginning of the GOCI's mission until September $29^{th}$, 2016\todo{update}, resulting in about 16,000 images. The GOCI Level-1B calibrated top-of-atmosphere (TOA) radiance data were obtained from the Ocean Biology Distributed Active Archive Center (OB.DAAC\todo{correct?}) at the NASA's Goddard Space Flight Center, maintained by the Ocean Biology Processing Group (OBPG) \todo{or OEL?}. The OB.DAAC acts a mirror site for the GOCI data provided by the Korea Ocean Satellite Center. These data are freely available for direct download from the OB.DAAC. Each file is provided as a  Hierarchical Data Format Release 5 (file extension: .he5) format and corresponds to one of the eight daily images per day. Each file contains eight images corresponding to the eight spectral bands.
% ---------------------------------------------------------------
\subsection{Study site}

The area of study is located in low productivity waters (clear waters)\todo{lon and lat for box}(\autoref{fig:AreaOfStudy}). The pixel box covers approximately 2 millions pixels ($\text{Number Total Pixels } (NTP) = (499*2+1)*(999*2+1) =  1,997,001$ pixels).

% ---------------------------------------------------------------
\begin{figure}[ht]
	\centering
	\missingfigure[figwidth=10cm]{Area of study in top of the GOCI footprint}
	\caption{Area of study}
	\label{fig:AreaOfStudy}
\end{figure}
%%%%%%%%%%%%%%%%%%%%% SECTION %%%%%%%%%%%%%%%%%%%%%%%%%%%%%%%%
\section{Processing Approach}
% ---------------------------------------------------------------
\subsection{Atmospheric Correction Algorithm}
GOCI Level 1B data (radiometric calibrations applied, L1B) was processed to Level 2 data (geolocated, geophysical values, L2) using the multisensor level 1 level 2 generator (l2gen) version 8.10.2 distributed with the SeaWiFS Data Analysis System (SeaDAS) (\url{http://seadas.gsfc.nasa.gov/}). The l2gen code reads level 1 observed top-of-atmosphere (TOA) radiances, applies an atmospheric correction scheme, and outputs $R_{rs}(\lambda)$ and various derived geophysical properties. The atmospheric correction scheme is based on \citet{Gordon1994}, with a near infrared (NIR) iterative correction by \citet{Bailey2010} ({\ttfamily aer\_opt=-2} in l2gen). GOCI's two near infrared (NIR) bands were used for the aerosol models selection. The current atmospheric correction algorithm is based on plane-parallel geometry \citep{Gordon1994}. \todo{include BRDF and RH-based model selection plus radiative transfer plane-parallel geometry}
% ---------------------------------------------------------------
\subsection{Data screening}
% Filtering Criteria
The exclusion criteria (filtering) was based on the criteria described by \citet{Bailey2006} for the validation of ocean color satellite data products. This exclusion criteria is presented in the flowchart in \autoref{fig:FilteringCriteria}. Pixels flagged by the atmospheric correction algorithm are excluded. From the rest of the unflagged pixels (i.e. filtered) in the box, a filtered mean and standard deviation were calculated. The filtered mean was defined as:
\begin{equation}\label{eq:filteredmean}
  \text{Filtered Mean} =\frac{\sum_i(Med-1.5*\sigma) <  X_i < (Med+1.5*\sigma)}{NFP}
\end{equation}
where $Med$ is the unfiltered median value, $\sigma$ is standard deviation of the unfiltered data and $NFP$ is the Number Filtered Pixels, i.e. the number of unflagged values within $\pm 1.5*\sigma$. Note the difference with Equation 4 in \citet{Bailey2006}, where the mean of the unfiltered data was used instead of the median. The use of the median value for the calculation of the filtered mean minimize the influence of outliers. This is the current standard approach adopted by the Ocean Ecology Lab at the NASA Goddard Space Flight Center. The statistical values filtered mean and standard deviation, among others, were computed operationally using the val-extract tool included in the SeaDAS/l2gen distribution.

In order to have statistical confidence in the filtered mean value, NFP is required to be at least half the number of total pixels in the box (i.e. $NFP\geq NTP/2 = 998,501$) to be considered into the analysis. Additionally, we excluded data where the solar zenith angle of the center of the pixel box exceed $75^o$ to avoid extreme viewing and solar geometries \citep{Bailey2006} (\autoref{fig:FilteringCriteria}).

% ---------------------------------------------------------------
% %%%%%%%%%%%%%%%%%%%%%%%%%%%%%%%%%%%%%%%%%%%%%%%%%%%%%%%%%%%%%%%%%%%%%%%
\begin{figure}[ht]
\vspace{1cm}
  \centering
  % \small
\resizebox{16cm}{!}{%
% Define block styles

% \tikzstyle{startstop} = [rectangle, rounded corners, minimum width=2.5em, minimum height=2em,text centered, draw=black, fill=white]

\tikzstyle{input} = [trapezium, trapezium left angle=70, trapezium right angle=110, minimum width=2em, minimum height=2em, text width=3cm, text centered, draw=black, fill=white]

\tikzstyle{input} = [trapezium, trapezium left angle=70, trapezium right angle=110, minimum width=2em, minimum height=2em, text width=3cm, text centered, draw=black, fill=white]

\tikzstyle{inputsmall} = [trapezium, trapezium left angle=70, trapezium right angle=110, minimum width=1em, minimum height=1em, text width=1.0cm, text centered, draw=black, fill=white]

\tikzstyle{output} = [rounded rectangle, minimum width=3em, minimum height=3em, text width=4.5cm, text centered, draw=black, fill=white, inner sep=-1pt]

\tikzstyle{process} = [rectangle, minimum width=3em, minimum height=2em, text width=6cm, text centered, draw=black, fill=white]

\tikzstyle{process_small} = [rectangle, minimum width=3em, minimum height=2em, text width=3cm, text centered, draw=black, fill=white]

\tikzstyle{decision} = [diamond, aspect=2, minimum width=2em, minimum height=2em, text width=5cm, text centered, draw=black, fill=white]

\tikzstyle{arrow} = [thick,->,>=triangle 45]
\tikzstyle{arrowdashed} = [thick,dashed,->,>=stealth]

\begin{tikzpicture}[node distance=2.5cm]


% \node (start) [startstop] {Start};

\node (L1B) [input] {Select L1B file to be \\processed to L2};
\node (l2gen) [process, below of=L1B] {Process to L2 using l2gen:\\ - Extract ROI\\- Atmospheric Correction};
\node (flag) [process, below of=l2gen, yshift=0.5cm] {Exclude flagged pixels};
\node (filtered) [process, below of=flag, yshift=-0.5cm] {Filtered Mean $=\frac{\displaystyle \sum_i^{NFP} X_i}{\displaystyle NFP}$\vspace{.3cm}\\$Med- 1.5*\sigma \leq X_i\leq Med+ 1.5*\sigma$\vspace{.3cm}\\Number Filtered Pixels (NFP)};
\node (enough) [decision, right of=L1B, inner sep=1pt, xshift=6cm] {Is $NFP > NTP/2$?\\ Number Total Pixels (NTP)};

\node (fail) [output, right of=enough, xshift=5cm] {Failed filtering criteria};

\node (zenith) [decision, below of=enough, yshift=-2cm] {Solar Zenith $< 75^o$?};

\node (pass) [output, below of=zenith, yshift=-1cm] {Passed filtering criteria};

\node (out) [input, below of=pass, yshift=0.5cm, inner sep=6pt] {Filtered Mean};


\draw [arrow] (L1B) -- (l2gen);
\draw [arrow] (l2gen) -- (flag);
\draw [arrow] (flag) -- (filtered);
\draw [arrow] (filtered.east) -|([xshift=-1cm]enough.west) -- (enough.west);

\draw [arrow] (enough) -- node[anchor=south] {NO} (fail);
\draw [arrow] (zenith) -| node[anchor=south,xshift=-3.0cm] {NO} (fail);
\draw [arrow] (enough) -- node[anchor=east] {YES} (zenith);
\draw [arrow] (zenith) -- node[anchor=east] {YES} (pass);
\draw [arrow] (pass) -- (out);

\end{tikzpicture}
} % resizebox end
	\caption{Applied exclusion criteria}
	\label{fig:FilteringCriteria}
\end{figure}
% ---------------------------------------------------------------
\subsection{Bio-Optical Algorithm}
% ---------------------------------------------------------------
\subsection{Vicarious Calibration?}
Vicarious calibration was applied.
%%%%%%%%%%%%%%%%%%% SECTION %%%%%%%%%%%%%%%%%%%%%%%%%%%%%%%%
\section{Results}
\label{sec:Results}
% ============================================================
% ---------------------------------------------------------------
\subsection{Matchups}
% - - - - - - - - - - - - - - - - - - - - - - - - - - - - - - - -
\subsubsection{AERONET-OC}
Validation of the atmospheric correction is performed by comparing the remote-sensing reflectance (Rrs) to in situ data from the
AErosol RObotic NETwork-Ocean Color (AERONET-OC).

Two stations: Gageocho (N=20; PI: Jae-Seol Shim and Joo-Hyung Ryu) and Ieodo (N=25; PI: Young-Je Park and Hak-Yeol You)

%-%-%-%-%-%-%-%-%-%-%=FIGURE=%-%-%-%-%-%-%-%-%-%-%-%-%
\begin{figure}[htb!]
    \begin{minipage}[c]{0.48\linewidth}
      \centering
      \includegraphics[height=6.5cm]{./Figures/GOCI_AERO_412.eps}
    \centerline{(a)}\medskip
    \end{minipage}  
    \hfill
    \begin{minipage}[c]{0.48\linewidth}
      \centering
      \includegraphics[height=6.5cm]{./Figures/GOCI_AERO_443.eps}
      \centerline{(b)}\medskip
    \end{minipage}  

  \begin{minipage}[c]{0.48\linewidth}
      \centering
      \includegraphics[height=6.5cm]{./Figures/GOCI_AERO_490.eps}
    \centerline{(c)}\medskip
    \end{minipage}  
    \hfill
    \begin{minipage}[c]{0.48\linewidth}
      \centering
      \includegraphics[height=6.5cm]{./Figures/GOCI_AERO_555.eps}
      \centerline{(d)}\medskip
    \end{minipage}  

  \begin{minipage}[c]{1.0\linewidth}
      \centering
      \includegraphics[height=6.5cm]{./Figures/GOCI_AERO_660.eps}
      \centerline{(e)}\medskip
    \end{minipage}   

    \caption{Comparison between GOCI and AERONET-OC data. \label{fig:GOCI_AERO} } 
\end{figure}
%-%-%-%-%-%-%-%-%-%-%=END FIGURE=%-%-%-%-%-%-%-%-%-%-%-%-%
% - - - - - - - - - - - - - - - - - - - - - - - - - - - - - - - -
\subsubsection{Cruises Matchups?}

% ---------------------------------------------------------------
\subsection{Time Series}
Time series for the $R_{rs}(\lambda)$ and Chlorophyll-{\it a} products are shown in \autoref{fig:GOCI_TimeSeries} and \autoref{fig:GOCI_TimeSeries2}. The values displayed correspond to the hourly mean of the area of study. Only values that passed the filtering criteria described previously were used.
%-%-%-%-%-%-%-%-%-%-%=FIGURE=%-%-%-%-%-%-%-%-%-%-%-%-%
\begin{figure}[htb!]
    \begin{minipage}[c]{0.66\linewidth}
      \centering
      \begin{overpic}[trim=70 400 0 30,clip,height=3.6cm]{./Figures/TimeSerie_Rrs412.eps} \put (50,23) {(a)}
      \end{overpic}
    \end{minipage}  
    \hfill
    \begin{minipage}[c]{0.33\linewidth}
      \centering
      \begin{overpic}[trim=0 0 0 0,clip,height=3.2cm]{./Figures/Hist_Rrs412.eps} \put (79,60) {(b)}
      \end{overpic} 
    \end{minipage}  

    \begin{minipage}[c]{0.66\linewidth}
      \centering
      \begin{overpic}[trim=70 400 0 30,clip,height=3.6cm]{./Figures/TimeSerie_Rrs443.eps} \put (50,23) {(c)}
      \end{overpic}
    \end{minipage}  
    \hfill
    \begin{minipage}[c]{0.33\linewidth}
      \centering
      \begin{overpic}[trim=0 0 0 0,clip,height=3.2cm]{./Figures/Hist_Rrs443.eps} \put (79,60) {(d)}
      \end{overpic} 
    \end{minipage}  

    \begin{minipage}[c]{0.66\linewidth}
      \centering
      \begin{overpic}[trim=70 400 0 30,clip,height=3.6cm]{./Figures/TimeSerie_Rrs490.eps} \put (50,23) {(e)}
      \end{overpic}
    \end{minipage}  
    \hfill
    \begin{minipage}[c]{0.33\linewidth}
      \centering
      \begin{overpic}[trim=0 0 0 0,clip,height=3.2cm]{./Figures/Hist_Rrs490.eps} \put (79,60) {(f)}
      \end{overpic} 
    \end{minipage}  

    \begin{minipage}[c]{0.66\linewidth}
      \centering
      \begin{overpic}[trim=70 400 0 30,clip,height=3.6cm]{./Figures/TimeSerie_Rrs555.eps} \put (50,23) {(g)}
      \end{overpic}
    \end{minipage}  
    \hfill
    \begin{minipage}[c]{0.33\linewidth}
      \centering
      \begin{overpic}[trim=0 0 0 0,clip,height=3.2cm]{./Figures/Hist_Rrs555.eps} \put (79,60) {(h)}
      \end{overpic} 
    \end{minipage}  

    \begin{minipage}[c]{0.66\linewidth}
      \centering
      \begin{overpic}[trim=70 400 0 30,clip,height=3.6cm]{./Figures/TimeSerie_Rrs660.eps} \put (50,23) {(i)}
      \end{overpic}
    \end{minipage}  
    \hfill
    \begin{minipage}[c]{0.33\linewidth}
      \centering
      \begin{overpic}[trim=0 0 0 0,clip,height=3.2cm]{./Figures/Hist_Rrs660.eps} \put (79,60) {(j)}
      \end{overpic} 
    \end{minipage}  

    \begin{minipage}[c]{0.66\linewidth}
      \centering
      \begin{overpic}[trim=70 0 0 430,clip,height=3.6cm]{./Figures/TimeSerie_Rrs680.eps} \put (50,23) {(k)}
      \end{overpic}
    \end{minipage}   

    \caption{Time Series. \label{fig:GOCI_TimeSeries} } 
\end{figure}
%-%-%-%-%-%-%-%-%-%-%=END FIGURE=%-%-%-%-%-%-%-%-%-%-%-%-%

%-%-%-%-%-%-%-%-%-%-%=FIGURE=%-%-%-%-%-%-%-%-%-%-%-%-%
\begin{figure}[htb!]
    \begin{minipage}[c]{0.66\linewidth}
      \centering
      \begin{overpic}[trim=70 400 0 30,clip,height=3.6cm]{./Figures/TimeSerie_Rrs680.eps} \put (50,23) {(l)}
      \end{overpic}
    \end{minipage}  
    \hfill
    \begin{minipage}[c]{0.33\linewidth}
      \centering
      \begin{overpic}[trim=0 0 0 0,clip,height=3.2cm]{./Figures/Hist_Rrs680.eps} \put (75,60) {(m)}
      \end{overpic} 
    \end{minipage} 

    \begin{minipage}[c]{0.66\linewidth}
      \centering
      \begin{overpic}[trim=70 400 0 30,clip,height=3.6cm]{./Figures/TimeSerie_Chla.eps} \put (50,23) {(n)}
      \end{overpic}
    \end{minipage}  
    \hfill
    \begin{minipage}[c]{0.33\linewidth}
      \centering
      \begin{overpic}[trim=0 0 0 0,clip,height=3.2cm]{./Figures/Hist_Chla.eps} \put (75,60) {(o)}
      \end{overpic} 
    \end{minipage}        

    \begin{minipage}[c]{0.66\linewidth}
      \centering
      \begin{overpic}[trim=70 0 0 430,clip,height=3.6cm]{./Figures/TimeSerie_Rrs680.eps} \put (50,23) {(p)}
      \end{overpic}
    \end{minipage}   

    \caption{Time Series (con't). \label{fig:GOCI_TimeSeries2} } 
\end{figure}
%-%-%-%-%-%-%-%-%-%-%=END FIGURE=%-%-%-%-%-%-%-%-%-%-%-%-%
% ---------------------------------------------------------------
\subsection{Products versus solar zenith angle}
The $R_{rs}$ and Chlorophyll-{\it a} mean values for the region of interest were compared against the solar zenith angle.
%-%-%-%-%-%-%-%-%-%-%=FIGURE=%-%-%-%-%-%-%-%-%-%-%-%-%
\begin{figure}[htb!]
    \begin{minipage}[c]{1.0\linewidth}
      \centering
      \begin{overpic}[trim=0 0 0 0,clip,height=10cm]{./Figures/Rrs_vs_Zenith.eps}
      \end{overpic}
	\end{minipage}  

    \caption{$R_{rs}$ versus solar zenith angle. \label{fig:Rrs_vs_zenith} } 
\end{figure}
%-%-%-%-%-%-%-%-%-%-%=END FIGURE=%-%-%-%-%-%-%-%-%-%-%-%-%

%-%-%-%-%-%-%-%-%-%-%=FIGURE=%-%-%-%-%-%-%-%-%-%-%-%-%
\begin{figure}[htb!]
    \begin{minipage}[c]{1.0\linewidth}
      \centering
      \begin{overpic}[trim=0 0 0 0,clip,height=5cm]{./Figures/Chl_vs_Zenith.eps}
      \end{overpic}
	\end{minipage}  

    \caption{$R_{rs}$ versus solar zenith angle. \label{fig:Chl_vs_zenith} } 
\end{figure}
%-%-%-%-%-%-%-%-%-%-%=END FIGURE=%-%-%-%-%-%-%-%-%-%-%-%-%

% ---------------------------------------------------------------
\subsection{Diurnal Differences and Uncertainties}
% Daily standard deviation
A standard deviation (SD) of the eight values per day was calculated. This daily standard deviation is an indicator of the temporal stability of the selected homogeneous ocean region. The hope is to have minimal variation during the day over this region. Then, all of the daily standard deviation were averaged to have a mean value per band. These values are compared to the results obtained from the comparison the AERONET-OC dataset (\autoref{tab:stdev_aero}).

\begin{table}[htbp!]
\caption{ \label{tab:stdev_aero} } 
\small
\centering
\begin{tabular}{c|c|c} 
 \bfseries{Product Name} & \bfseries{SD} & \bfseries{RMSE}\\
 & (Time Series)) & (AERONET-OC) \\ 
 & $[1/sr]$ & $[1/sr]$ \\ \hline \hline
$R_{rs}(412)$ & $6.84\times10^4$ & $1.8\times10^3$\\ 
$R_{rs}(443)$ & $4.78\times10^4$ & $1.7\times10^3$\\ 
$R_{rs}(490)$ & $4.00\times10^4$ & $1.8\times10^3$\\ 
$R_{rs}(555)$ & $1.73\times10^4$ & $2.4\times10^3$\\ 
$R_{rs}(660)$ & $2.88\times10^5$ & $7.0\times10^4$\\ 
$R_{rs}(680)$ & $1.94\times10^5$ & N/A \\ 
 \end{tabular}
\end{table}

% Relative difference
There are eight images for each day. For every image, the mean of the entire box was calculated. Then, these mean values were averaged to obtain a daily mean, as follow

\begin{equation}
	\hat{X} = \frac{1}{N} \sum_{t}^N x_t
\end{equation}

The difference con respect to a reference is $\Delta_t=x_t-x_{reference}$. Then, the relative difference is defined as
\begin{equation}
	RD_t = \frac{\Delta_t}{|x_{reference}|} = \frac{x_t-x_{reference}}{|x_{reference}|}
	\times 100[\%]
\end{equation}
where $x_t$ is the satellite data at the local time $t=09h,10h\dots16h$ of the day.



%-%-%-%-%-%-%-%-%-%-%=FIGURE=%-%-%-%-%-%-%-%-%-%-%-%-%
\begin{figure}[htb!]
    \begin{minipage}[c]{0.48\linewidth}
      \centering
      \includegraphics[height=5.5cm]{./Figures/Rel_Diff_Daily_Mean_Rrs412.eps}
    \centerline{(a)}\medskip
    \end{minipage}  
    \hfill
    \begin{minipage}[c]{0.48\linewidth}
      \centering
      \includegraphics[height=5.5cm]{./Figures/Rel_Diff_Daily_Mean_Rrs443.eps}
      \centerline{(b)}\medskip
    \end{minipage}  

  \begin{minipage}[c]{0.48\linewidth}
      \centering
      \includegraphics[height=5.5cm]{./Figures/Rel_Diff_Daily_Mean_Rrs490.eps}
    \centerline{(c)}\medskip
    \end{minipage}  
    \hfill
    \begin{minipage}[c]{0.48\linewidth}
      \centering
      \includegraphics[height=5.5cm]{./Figures/Rel_Diff_Daily_Mean_Rrs555.eps}
      \centerline{(d)}\medskip
    \end{minipage}  

  \begin{minipage}[c]{0.48\linewidth}
      \centering
      \includegraphics[height=5.5cm]{./Figures/Rel_Diff_Daily_Mean_Rrs660.eps}
      \centerline{(e)}\medskip
    \end{minipage}   
        \hfill
    \begin{minipage}[c]{0.48\linewidth}
      \centering
      \includegraphics[height=5.5cm]{./Figures/Rel_Diff_Daily_Mean_Rrs680.eps}
      \centerline{(f)}\medskip
    \end{minipage} 
    \caption{Relative difference with respect to the daily mean value for Rrs. \label{fig:DiffDailyMeanRrs} } 
\end{figure}
%-%-%-%-%-%-%-%-%-%-%=END FIGURE=%-%-%-%-%-%-%-%-%-%-%-%-%

%-%-%-%-%-%-%-%-%-%-%=FIGURE=%-%-%-%-%-%-%-%-%-%-%-%-%
\begin{figure}[htb!]
    \begin{minipage}[c]{0.33\linewidth}
      \centering
      \includegraphics[height=4.5cm]{./Figures/Rel_Diff_Daily_Mean_chlor_a.eps}
    \centerline{(a)}\medskip
    \end{minipage}  
    % \hfill
    \begin{minipage}[c]{0.33\linewidth}
      \centering
      \includegraphics[height=4.5cm]{./Figures/Rel_Diff_Daily_Mean_poc.eps}
      \centerline{(b)}\medskip
    \end{minipage}  
    % \hfill
  	\begin{minipage}[c]{0.33\linewidth}
      \centering
      \includegraphics[height=4.5cm]{./Figures/Rel_Diff_Daily_Mean_ag_412_mlrc.eps}
    \centerline{(c)}\medskip
    \end{minipage}  

    \caption{Relative Difference with respect to the daily mean value for chlor-a, poc and ag(412)mlrc. \label{fig:DiffDailyMeanProd} } 
\end{figure}
%-%-%-%-%-%-%-%-%-%-%=END FIGURE=%-%-%-%-%-%-%-%-%-%-%-%-%

% ---------------------------------------------------------------
\subsection{Sensor Cross-comparison}

%-%-%-%-%-%-%-%-%-%-%=FIGURE=%-%-%-%-%-%-%-%-%-%-%-%-%
\begin{figure}[htb!]
    \begin{minipage}[c]{1.0\linewidth}
      \centering
      \begin{overpic}[trim=0 0 0 0,clip,height=3.5cm]{./Figures/CrossComp_Rrs412.eps} \put (10,28) {(a)}
      \end{overpic}
    \end{minipage}   
    
    \begin{minipage}[c]{1.0\linewidth}
      \centering
      \begin{overpic}[trim=0 0 0 0,clip,height=3.5cm]{./Figures/CrossComp_Rrs443.eps} \put (10,28) {(b)}
      \end{overpic}
    \end{minipage}   

    \begin{minipage}[c]{1.0\linewidth}
      \centering
      \begin{overpic}[trim=0 0 0 0,clip,height=3.5cm]{./Figures/CrossComp_Rrs490.eps} \put (10,28) {(c)}
      \end{overpic}
    \end{minipage}  
    
    \begin{minipage}[c]{1.0\linewidth}
      \centering
      \begin{overpic}[trim=0 0 0 0,clip,height=3.5cm]{./Figures/CrossComp_Rrs555.eps} \put (10,28) {(d)}
      \end{overpic}
    \end{minipage}   

    \begin{minipage}[c]{1.0\linewidth}
      \centering
      \begin{overpic}[trim=0 0 0 0,clip,height=3.5cm]{./Figures/CrossComp_Rrs660.eps} \put (10,28) {(e)}
      \end{overpic}
    \end{minipage}  
    
    \begin{minipage}[c]{1.0\linewidth}
      \centering
      \begin{overpic}[trim=0 0 0 0,clip,height=3.5cm]{./Figures/CrossComp_Rrs680.eps} \put (10,28) {(f)}
      \end{overpic}
    \end{minipage}   

    \caption{Cross-comparison. \label{fig:CrossComp} } 
\end{figure}
%-%-%-%-%-%-%-%-%-%-%=END FIGURE=%-%-%-%-%-%-%-%-%-%-%-%-%

%-%-%-%-%-%-%-%-%-%-%=FIGURE=%-%-%-%-%-%-%-%-%-%-%-%-%
\begin{figure}[htb!]
    \begin{minipage}[c]{1.0\linewidth}
      \centering
      \begin{overpic}[trim=0 0 0 0,clip,height=3.5cm]{./Figures/TimeSerie_chlor_a.eps} \put (9,30) {(a)}
      \end{overpic}
    \end{minipage} 

    \begin{minipage}[c]{1.0\linewidth}
      \centering
      \begin{overpic}[trim=0 0 0 0,clip,height=3.5cm]{./Figures/TimeSerie_ag_412_mlrc.eps} \put (9,30) {(b)}
      \end{overpic}
    \end{minipage} 

    \begin{minipage}[c]{1.0\linewidth}
      \centering
      \begin{overpic}[trim=0 0 0 0,clip,height=3.5cm]{./Figures/TimeSerie_poc.eps} \put (9,30) {(c)}
      \end{overpic}
    \end{minipage} 

    \caption{Time Series for derived geophysical parameters. \label{fig:GOCI_TimeSeries_geophysical_par} } 
\end{figure}
%-%-%-%-%-%-%-%-%-%-%=END FIGURE=%-%-%-%-%-%-%-%-%-%-%-%-%

%-%-%-%-%-%-%-%-%-%-%=FIGURE=%-%-%-%-%-%-%-%-%-%-%-%-%
\begin{figure}[htb!]
    \begin{minipage}[c]{1.0\linewidth}
      \centering
      \begin{overpic}[trim=0 0 0 0,clip,height=3.5cm]{./Figures/TimeSerie_Angstrom.eps} \put (9,30) {(d)}
      \end{overpic}
    \end{minipage}   
    
    \begin{minipage}[c]{1.0\linewidth}
      \centering
      \begin{overpic}[trim=0 0 0 0,clip,height=3.5cm]{./Figures/TimeSerie_AOT_865.eps} \put (9,30) {(e)}
      \end{overpic}
    \end{minipage}       

    \begin{minipage}[c]{1.0\linewidth}
      \centering
      \begin{overpic}[trim=0 0 0 0,clip,height=3.5cm]{./Figures/TimeSerie_brdf.eps} \put (9,30) {(f)}
      \end{overpic}
    \end{minipage} 

    \caption{Time Series for derived geophysical paremeters (Angstrom and AOT(865)) and atmospheric correction intermediate (BRDF). \label{fig:GOCI_TimeSeries_intermed_par} } 
\end{figure}
%-%-%-%-%-%-%-%-%-%-%=END FIGURE=%-%-%-%-%-%-%-%-%-%-%-%-%

% ---------------------------------------------------------------
\subsection{Plane-parallel versus Pseudo-Spherical geometry}

% ---------------------------------------------------------------
\subsection{BRDF correction sensitivity analysis}
%%%%%%%%%%%%%%%%%%% SECTION %%%%%%%%%%%%%%%%%%%%%%%%%%%%%%%%
\section{Discussion}

%%%%%%%%%%%%%%%%%%% SECTION %%%%%%%%%%%%%%%%%%%%%%%%%%%%%%%%
\section{Conclusions}
% Practical applications  
% Disadvantages and Advantages
% Limitations
% Challenges

% from OO poster
The validation with in situ data exhibit results comparable to heritage sensors.

Expected seasonality and trends were observed through the complete mission.

The atmospheric correction starts to fail for solar zenith angles larger than 60 degrees producing invalid values (negative). 

Uncertainties vary spectrally, being larger in the blue and decreasing towards the red.

Uncertainty patterns are similar for the six first images of the day, increasing for the last two images across all wavelengths

% future work
Estimation of changes due to diurnal and day-to-day biogeochemical stocks and processes in coastal oceans.


%%%%%%%%%%%%%%%%%%% SECTION %%%%%%%%%%%%%%%%%%%%%%%%%%%%%%%%
% \vspace{-.4cm}
\section*{Acknowledgments}
\vspace{-.2cm}
We want to acknowledge the NASA Project ROSES Earth Science U.S. Participating Investigator (NNH12ZDA001N-ESUSPI), the AERONET-OC, the Korea Ocean Satellite Center for providing the GOCI L1B data to OBPG, and the Ocean Biology Processing Group at the Goddard Space Flight Center, NASA. 

%%%%%%%%%%%%%%%%%%% SECTION %%%%%%%%%%%%%%%%%%%%%%%%%%%%%%%%
% \vspace{-.4cm}
\section*{References}

%%%%%%%%%%%%%%%%%%%%%%%
%% Elsevier bibliography styles
%%%%%%%%%%%%%%%%%%%%%%%
%% To change the style, put a % in front of the second line of the current style and
%% remove the % from the second line of the style you would like to use.
%%%%%%%%%%%%%%%%%%%%%%%

%% Numbered
%\bibliographystyle{model1-num-names}

%% Numbered without titles
%\bibliographystyle{model1a-num-names}

%% Harvard
% \bibliographystyle{model2-names.bst}\biboptions{authoryear}

%% Vancouver numbered
%\usepackage{numcompress}\bibliographystyle{model3-num-names}

%% Vancouver name/year
%\usepackage{numcompress}\bibliographystyle{model4-names}\biboptions{authoryear}

%% APA style
\bibliographystyle{model5-names}\biboptions{authoryear}

%% AMA style
%\usepackage{numcompress}\bibliographystyle{model6-num-names}

%% `Elsevier LaTeX' style
% \bibliographystyle{elsarticle-num}
% \bibliographystyle{apalike}
\bibliography{/Users/jconchas/Documents/Latex/bib/javier_NASA.bib} 

% \listoffigures

% \listoftables

\end{document}