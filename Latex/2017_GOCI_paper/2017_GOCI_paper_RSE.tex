% \documentclass[twocolumn,3p]{elsarticle}
\documentclass[onecolumn,3p,letterpaper,11pt]{elsarticle}
\usepackage{setspace} % added by JC 
\doublespacing % added by JC


\usepackage{lineno} % uncomment \linenumbers after \begin{document}
\modulolinenumbers[1]

\usepackage{hyperref}

%%% MY PACKAGES %%%%%%%%%%%%%%%%%%%%%%%%%%%%%%%%%%%%%%%%%%%%%%%
\usepackage{graphicx}
% \usepackage[outdir=./]{epstopdf}
\usepackage{epstopdf}
\epstopdfsetup{update} % only regenerate pdf files when eps file is newer
\usepackage{amsmath,epsfig}

% Select what to do with todonotes: 
% \usepackage[disable]{todonotes} % notes not showed
% % \usepackage[draft]{todonotes}   % notes showed
% \setlength{\marginparwidth}{2cm}
% \usepackage[textwidth=3.7cm]{todonotes}

% \usepackage{tikz} % for flow charts
%   \usetikzlibrary{shapes,arrows,positioning,shadows,calc}
%   \usetikzlibrary{external}
%   \tikzexternalize

% \usepackage[nostamp]{draftwatermark}
% \SetWatermarkLightness{0.8}
% \SetWatermarkScale{4}

\usepackage[percent]{overpic}
\usepackage{morefloats} % for the error "Too many unprocessed floats"

\usepackage{multirow}

\renewcommand*{\bibfont}{\normalsize}
%%% END MY PACKAGES %%%%%%%%%%%%%%%%%%%%%%%%%%%%%%%%%%%%%%%%%%%

\journal{Remote Sensing of Environment}

\begin{document}

% \linenumbers

\begin{frontmatter}

\title{Assesing Diurnal Variability of Biogeochical Processes using the Geostationary Ocean Color Imager (GOCI)}

% %% Group authors per affiliation:
% \author{Javier A. Concha\fnref{myfootnote}}
% \address{Radarweg 29, Amsterdam}
% \fntext[myfootnote]{Since 1880.}

%% or include affiliations in footnotes:
\author[oeladdress,usraaddress]{Javier Concha\corref{mycorrespondingauthor}}
\cortext[mycorrespondingauthor]{Corresponding author at: Ocean Ecology Lab,
NASA Goddard Space Flight Center,
8800 Greenbelt Rd, Greenbelt, MD 20771, USA. Tel.: +1 585 290 3145.}
\ead{javier.concha@nasa.gov}

\author[oeladdress]{Antonio Mannino}

\author[oeladdress]{Bryan Franz}

\author[kiostaddress]{Wonkook Kim}

\author[usgsaddress]{Michael Ondrusek}

\address[oeladdress]{Ocean Ecology lab, NASA Goddard Space Flight Center, Greenbelt, MD, USA}
\address[usraaddress]{Universities Space Research Association, Columbia, MD, USA}
\address[kiostaddress]{Korea Institute of Ocean Science and Technology, 787 Haean-ro, Ansan, Republic of Korea}

\address[usgsaddress]{NOAA/NESDIS Center for Weather and Climate Prediction, College Park, Maryland, USA}
% ===============================================================
\begin{abstract}

% Background/motivation/context

%

% Aim/objectives(s)/problem statement

%
 
% Methods

%

%

%

% Results

%
 
%

% Conclusions

%

\end{abstract}

\begin{keyword}
Geostationary Ocean Color Imager\sep GOCI\sep diurnal dynamics\sep diurnal variability\sep CDOM
\end{keyword}

\end{frontmatter}
%%%%%%%%%%%%%%%%%%% SECTION %%%%%%%%%%%%%%%%%%%%%%%%%%%%%%%%
\section{Introduction}
GOCI as a precursor to GEO-CAPE. 
Name GOCI-II and the ESA geostationary satellite.

Spinning Enhanced Visible and Infrared Imager (SEVIRI) \cite{Neukermans2009,Neukermans2012}. Geostationary review \cite{Ruddick2014}. Synergy \cite{Vanhellemont2014}.

Example of citations \cite{Ryu2011,He2013,Hu2016}


%%%%%%%%%%%%%%%%%%%%% SECTION %%%%%%%%%%%%%%%%%%%%%%%%%%%%%%%%
% \section{Data and Methodology}
\section{Data and Sensor Characteristics}
% ============================================================
\subsection{GOCI Data}
%%%%%%%%%%%%%%%%%%%%% SECTION %%%%%%%%%%%%%%%%%%%%%%%%%%%%%%%%
\section{Processing Approach}
% ---------------------------------------------------------------
\subsection{Atmospheric Correction Algorithm}

% ---------------------------------------------------------------
\subsection{Bio-Optical Algorithm}
% ---------------------------------------------------------------
\subsection{Vicarious Calibration?}

%%%%%%%%%%%%%%%%%%% SECTION %%%%%%%%%%%%%%%%%%%%%%%%%%%%%%%%
\section{Results}
\label{sec:Results}
% ============================================================
% ---------------------------------------------------------------
\subsection{Matchups}
% - - - - - - - - - - - - - - - - - - - - - - - - - - - - - - - -
\subsubsection{AERONET-OC}

%-%-%-%-%-%-%-%-%-%-%=FIGURE=%-%-%-%-%-%-%-%-%-%-%-%-%
\begin{figure}[htb!]
    \begin{minipage}[c]{0.48\linewidth}
      \centering
      \includegraphics[height=6.5cm]{./Figures/GOCI_AERO_412.eps}
    \centerline{(a)}\medskip
    \end{minipage}  
    \hfill
    \begin{minipage}[c]{0.48\linewidth}
      \centering
      \includegraphics[height=6.5cm]{./Figures/GOCI_AERO_443.eps}
      \centerline{(b)}\medskip
    \end{minipage}  

  \begin{minipage}[c]{0.48\linewidth}
      \centering
      \includegraphics[height=6.5cm]{./Figures/GOCI_AERO_490.eps}
    \centerline{(c)}\medskip
    \end{minipage}  
    \hfill
    \begin{minipage}[c]{0.48\linewidth}
      \centering
      \includegraphics[height=6.5cm]{./Figures/GOCI_AERO_555.eps}
      \centerline{(d)}\medskip
    \end{minipage}  

  \begin{minipage}[c]{1.0\linewidth}
      \centering
      \includegraphics[height=6.5cm]{./Figures/GOCI_AERO_660.eps}
      \centerline{(e)}\medskip
    \end{minipage}   

    \caption{Comparison between GOCI and AERONET-OC data. \label{fig:GOCI_AERO} } 
\end{figure}
%-%-%-%-%-%-%-%-%-%-%=END FIGURE=%-%-%-%-%-%-%-%-%-%-%-%-%
% - - - - - - - - - - - - - - - - - - - - - - - - - - - - - - - -
\subsubsection{Cruises Matchups}

% ---------------------------------------------------------------
\subsection{Time Series}

%-%-%-%-%-%-%-%-%-%-%=FIGURE=%-%-%-%-%-%-%-%-%-%-%-%-%
\begin{figure}[htb!]
    \begin{minipage}[c]{0.66\linewidth}
      \centering
      \begin{overpic}[trim=70 400 0 30,clip,height=3.6cm]{./Figures/TimeSerie_Rrs412.eps} \put (50,23) {(a)}
      \end{overpic}
    \end{minipage}  
    \hfill
    \begin{minipage}[c]{0.33\linewidth}
      \centering
      \begin{overpic}[trim=0 0 0 0,clip,height=3.2cm]{./Figures/Hist_Rrs412.eps} \put (79,60) {(b)}
      \end{overpic} 
    \end{minipage}  

    \begin{minipage}[c]{0.66\linewidth}
      \centering
      \begin{overpic}[trim=70 400 0 30,clip,height=3.6cm]{./Figures/TimeSerie_Rrs443.eps} \put (50,23) {(c)}
      \end{overpic}
    \end{minipage}  
    \hfill
    \begin{minipage}[c]{0.33\linewidth}
      \centering
      \begin{overpic}[trim=0 0 0 0,clip,height=3.2cm]{./Figures/Hist_Rrs443.eps} \put (79,60) {(d)}
      \end{overpic} 
    \end{minipage}  

    \begin{minipage}[c]{0.66\linewidth}
      \centering
      \begin{overpic}[trim=70 400 0 30,clip,height=3.6cm]{./Figures/TimeSerie_Rrs490.eps} \put (50,23) {(e)}
      \end{overpic}
    \end{minipage}  
    \hfill
    \begin{minipage}[c]{0.33\linewidth}
      \centering
      \begin{overpic}[trim=0 0 0 0,clip,height=3.2cm]{./Figures/Hist_Rrs490.eps} \put (79,60) {(f)}
      \end{overpic} 
    \end{minipage}  

    \begin{minipage}[c]{0.66\linewidth}
      \centering
      \begin{overpic}[trim=70 400 0 30,clip,height=3.6cm]{./Figures/TimeSerie_Rrs555.eps} \put (50,23) {(g)}
      \end{overpic}
    \end{minipage}  
    \hfill
    \begin{minipage}[c]{0.33\linewidth}
      \centering
      \begin{overpic}[trim=0 0 0 0,clip,height=3.2cm]{./Figures/Hist_Rrs555.eps} \put (79,60) {(h)}
      \end{overpic} 
    \end{minipage}  

    \begin{minipage}[c]{0.66\linewidth}
      \centering
      \begin{overpic}[trim=70 400 0 30,clip,height=3.6cm]{./Figures/TimeSerie_Rrs660.eps} \put (50,23) {(i)}
      \end{overpic}
    \end{minipage}  
    \hfill
    \begin{minipage}[c]{0.33\linewidth}
      \centering
      \begin{overpic}[trim=0 0 0 0,clip,height=3.2cm]{./Figures/Hist_Rrs660.eps} \put (79,60) {(j)}
      \end{overpic} 
    \end{minipage}  

    \begin{minipage}[c]{0.66\linewidth}
      \centering
      \begin{overpic}[trim=70 0 0 430,clip,height=3.6cm]{./Figures/TimeSerie_Rrs680.eps} \put (50,23) {(k)}
      \end{overpic}
    \end{minipage}   

    \caption{Time Series. \label{fig:GOCI_TimeSeries} } 
\end{figure}
%-%-%-%-%-%-%-%-%-%-%=END FIGURE=%-%-%-%-%-%-%-%-%-%-%-%-%

%-%-%-%-%-%-%-%-%-%-%=FIGURE=%-%-%-%-%-%-%-%-%-%-%-%-%
\begin{figure}[htb!]
    \begin{minipage}[c]{0.66\linewidth}
      \centering
      \begin{overpic}[trim=70 400 0 30,clip,height=3.6cm]{./Figures/TimeSerie_Rrs680.eps} \put (50,23) {(l)}
      \end{overpic}
    \end{minipage}  
    \hfill
    \begin{minipage}[c]{0.33\linewidth}
      \centering
      \begin{overpic}[trim=0 0 0 0,clip,height=3.2cm]{./Figures/Hist_Rrs680.eps} \put (75,60) {(m)}
      \end{overpic} 
    \end{minipage} 

    \begin{minipage}[c]{0.66\linewidth}
      \centering
      \begin{overpic}[trim=70 400 0 30,clip,height=3.6cm]{./Figures/TimeSerie_Chla.eps} \put (50,23) {(n)}
      \end{overpic}
    \end{minipage}  
    \hfill
    \begin{minipage}[c]{0.33\linewidth}
      \centering
      \begin{overpic}[trim=0 0 0 0,clip,height=3.2cm]{./Figures/Hist_Chla.eps} \put (75,60) {(o)}
      \end{overpic} 
    \end{minipage}        

    \begin{minipage}[c]{0.66\linewidth}
      \centering
      \begin{overpic}[trim=70 0 0 430,clip,height=3.6cm]{./Figures/TimeSerie_Rrs680.eps} \put (50,23) {(p)}
      \end{overpic}
    \end{minipage}   

    \caption{Time Series (con't). \label{fig:GOCI_TimeSeries2} } 
\end{figure}
%-%-%-%-%-%-%-%-%-%-%=END FIGURE=%-%-%-%-%-%-%-%-%-%-%-%-%
% ---------------------------------------------------------------
\subsection{Products versus solar zenith angle}

%-%-%-%-%-%-%-%-%-%-%=FIGURE=%-%-%-%-%-%-%-%-%-%-%-%-%
\begin{figure}[htb!]
    \begin{minipage}[c]{1.0\linewidth}
      \centering
      \begin{overpic}[trim=0 0 0 0,clip,height=10cm]{./Figures/Rrs_vs_Zenith.eps}
      \end{overpic}
	\end{minipage}  

    \caption{Rrs versus solar zenith angle. \label{fig:Rrs_vs_zenith} } 
\end{figure}
%-%-%-%-%-%-%-%-%-%-%=END FIGURE=%-%-%-%-%-%-%-%-%-%-%-%-%

%-%-%-%-%-%-%-%-%-%-%=FIGURE=%-%-%-%-%-%-%-%-%-%-%-%-%
\begin{figure}[htb!]
    \begin{minipage}[c]{1.0\linewidth}
      \centering
      \begin{overpic}[trim=0 0 0 0,clip,height=5cm]{./Figures/Chl_vs_Zenith.eps}
      \end{overpic}
	\end{minipage}  

    \caption{Rrs versus solar zenith angle. \label{fig:Chl_vs_zenith} } 
\end{figure}
%-%-%-%-%-%-%-%-%-%-%=END FIGURE=%-%-%-%-%-%-%-%-%-%-%-%-%

% ---------------------------------------------------------------
\subsection{Diurnal Differences and Uncertainties}
There are eight images for each day. For every image, the mean of the entire box was calculated. Then, these mean values were averaged to obtain a daily mean, as follow

\begin{equation}
	\hat{X} = \frac{1}{N} \sum_{t}^N x_t
\end{equation}

The difference con respect to a reference is $\Delta_t=x_t-x_{reference}$. Then, the relative difference is defined as
\begin{equation}
	RD_t = \frac{\Delta_t}{|x_{reference}|} = \frac{x_t-x_{reference}}{|x_{reference}|}
	\times 100[\%]
\end{equation}
where $x_t$ is the satellite data at the local time $t=09h,10h\dots16h$ of the day.



%-%-%-%-%-%-%-%-%-%-%=FIGURE=%-%-%-%-%-%-%-%-%-%-%-%-%
\begin{figure}[htb!]
    \begin{minipage}[c]{0.48\linewidth}
      \centering
      \includegraphics[height=5.5cm]{./Figures/Rel_Diff_Daily_Mean_Rrs412.eps}
    \centerline{(a)}\medskip
    \end{minipage}  
    \hfill
    \begin{minipage}[c]{0.48\linewidth}
      \centering
      \includegraphics[height=5.5cm]{./Figures/Rel_Diff_Daily_Mean_Rrs443.eps}
      \centerline{(b)}\medskip
    \end{minipage}  

  \begin{minipage}[c]{0.48\linewidth}
      \centering
      \includegraphics[height=5.5cm]{./Figures/Rel_Diff_Daily_Mean_Rrs490.eps}
    \centerline{(c)}\medskip
    \end{minipage}  
    \hfill
    \begin{minipage}[c]{0.48\linewidth}
      \centering
      \includegraphics[height=5.5cm]{./Figures/Rel_Diff_Daily_Mean_Rrs555.eps}
      \centerline{(d)}\medskip
    \end{minipage}  

  \begin{minipage}[c]{0.48\linewidth}
      \centering
      \includegraphics[height=5.5cm]{./Figures/Rel_Diff_Daily_Mean_Rrs660.eps}
      \centerline{(e)}\medskip
    \end{minipage}   
        \hfill
    \begin{minipage}[c]{0.48\linewidth}
      \centering
      \includegraphics[height=5.5cm]{./Figures/Rel_Diff_Daily_Mean_Rrs680.eps}
      \centerline{(f)}\medskip
    \end{minipage} 
    \caption{Relative difference with respect to the daily mean value for Rrs. \label{fig:DiffDailyMeanRrs} } 
\end{figure}
%-%-%-%-%-%-%-%-%-%-%=END FIGURE=%-%-%-%-%-%-%-%-%-%-%-%-%

%-%-%-%-%-%-%-%-%-%-%=FIGURE=%-%-%-%-%-%-%-%-%-%-%-%-%
\begin{figure}[htb!]
    \begin{minipage}[c]{0.33\linewidth}
      \centering
      \includegraphics[height=4.5cm]{./Figures/Diff_Mean_chlor_a.eps}
    \centerline{(a)}\medskip
    \end{minipage}  
    % \hfill
    \begin{minipage}[c]{0.33\linewidth}
      \centering
      \includegraphics[height=4.5cm]{./Figures/Diff_Mean_poc.eps}
      \centerline{(b)}\medskip
    \end{minipage}  
    % \hfill
  	\begin{minipage}[c]{0.33\linewidth}
      \centering
      \includegraphics[height=4.5cm]{./Figures/Diff_Mean_ag_412_mlrc.eps}
    \centerline{(c)}\medskip
    \end{minipage}  

    \caption{Difference with respect to the daily mean value for chlor-a, poc and ag(412)mlrc. \label{fig:DiffDailyMeanProd} } 
\end{figure}
%-%-%-%-%-%-%-%-%-%-%=END FIGURE=%-%-%-%-%-%-%-%-%-%-%-%-%
% ---------------------------------------------------------------
\subsection{Sensor Cross-comparison}

%-%-%-%-%-%-%-%-%-%-%=FIGURE=%-%-%-%-%-%-%-%-%-%-%-%-%
\begin{figure}[htb!]
    \begin{minipage}[c]{1.0\linewidth}
      \centering
      \begin{overpic}[trim=0 0 0 0,clip,height=3.5cm]{./Figures/CrossComp_Rrs412.eps} \put (10,28) {(a)}
      \end{overpic}
    \end{minipage}   
    
    \begin{minipage}[c]{1.0\linewidth}
      \centering
      \begin{overpic}[trim=0 0 0 0,clip,height=3.5cm]{./Figures/CrossComp_Rrs443.eps} \put (10,28) {(b)}
      \end{overpic}
    \end{minipage}   

    \begin{minipage}[c]{1.0\linewidth}
      \centering
      \begin{overpic}[trim=0 0 0 0,clip,height=3.5cm]{./Figures/CrossComp_Rrs490.eps} \put (10,28) {(c)}
      \end{overpic}
    \end{minipage}  
    
    \begin{minipage}[c]{1.0\linewidth}
      \centering
      \begin{overpic}[trim=0 0 0 0,clip,height=3.5cm]{./Figures/CrossComp_Rrs555.eps} \put (10,28) {(d)}
      \end{overpic}
    \end{minipage}   

    \begin{minipage}[c]{1.0\linewidth}
      \centering
      \begin{overpic}[trim=0 0 0 0,clip,height=3.5cm]{./Figures/CrossComp_Rrs660.eps} \put (10,28) {(e)}
      \end{overpic}
    \end{minipage}  
    
    \begin{minipage}[c]{1.0\linewidth}
      \centering
      \begin{overpic}[trim=0 0 0 0,clip,height=3.5cm]{./Figures/CrossComp_Rrs680.eps} \put (10,28) {(f)}
      \end{overpic}
    \end{minipage}   

    \caption{Cross-comparison. \label{fig:CrossComp} } 
\end{figure}
%-%-%-%-%-%-%-%-%-%-%=END FIGURE=%-%-%-%-%-%-%-%-%-%-%-%-%

%-%-%-%-%-%-%-%-%-%-%=FIGURE=%-%-%-%-%-%-%-%-%-%-%-%-%
\begin{figure}[htb!]
    \begin{minipage}[c]{1.0\linewidth}
      \centering
      \begin{overpic}[trim=0 0 0 0,clip,height=3.5cm]{./Figures/TimeSerie_chlor_a.eps} \put (9,30) {(a)}
      \end{overpic}
    \end{minipage} 

    \begin{minipage}[c]{1.0\linewidth}
      \centering
      \begin{overpic}[trim=0 0 0 0,clip,height=3.5cm]{./Figures/TimeSerie_ag_412_mlrc.eps} \put (9,30) {(b)}
      \end{overpic}
    \end{minipage} 

    \begin{minipage}[c]{1.0\linewidth}
      \centering
      \begin{overpic}[trim=0 0 0 0,clip,height=3.5cm]{./Figures/TimeSerie_poc.eps} \put (9,30) {(c)}
      \end{overpic}
    \end{minipage} 

    \caption{Time Series for derived geophysical parameters. \label{fig:GOCI_TimeSeries_geophysical_par} } 
\end{figure}
%-%-%-%-%-%-%-%-%-%-%=END FIGURE=%-%-%-%-%-%-%-%-%-%-%-%-%

%-%-%-%-%-%-%-%-%-%-%=FIGURE=%-%-%-%-%-%-%-%-%-%-%-%-%
\begin{figure}[htb!]
    \begin{minipage}[c]{1.0\linewidth}
      \centering
      \begin{overpic}[trim=0 0 0 0,clip,height=3.5cm]{./Figures/TimeSerie_Angstrom.eps} \put (9,30) {(d)}
      \end{overpic}
    \end{minipage}   
    
    \begin{minipage}[c]{1.0\linewidth}
      \centering
      \begin{overpic}[trim=0 0 0 0,clip,height=3.5cm]{./Figures/TimeSerie_AOT_865.eps} \put (9,30) {(e)}
      \end{overpic}
    \end{minipage}       

    \begin{minipage}[c]{1.0\linewidth}
      \centering
      \begin{overpic}[trim=0 0 0 0,clip,height=3.5cm]{./Figures/TimeSerie_brdf.eps} \put (9,30) {(f)}
      \end{overpic}
    \end{minipage} 

    \caption{Time Series for derived geophysical paremeters (Angstrom and AOT(865)) and atmospheric correction intermediate (BRDF). \label{fig:GOCI_TimeSeries_intermed_par} } 
\end{figure}
%-%-%-%-%-%-%-%-%-%-%=END FIGURE=%-%-%-%-%-%-%-%-%-%-%-%-%

%%%%%%%%%%%%%%%%%%% SECTION %%%%%%%%%%%%%%%%%%%%%%%%%%%%%%%%
\section{Discussion}

%%%%%%%%%%%%%%%%%%% SECTION %%%%%%%%%%%%%%%%%%%%%%%%%%%%%%%%
\section{Conclusions}
% Practical applications  
% Disadvantages and Advantages
% Limitations
% Challenges

%%%%%%%%%%%%%%%%%%% SECTION %%%%%%%%%%%%%%%%%%%%%%%%%%%%%%%%
% \vspace{-.4cm}
\section*{Acknowledgments}
\vspace{-.2cm}
We want to acknowledge the NASA Project ROSES Earth Science U.S. Participating Investigator (NNH12ZDA001N-ESUSPI), the AERONET-OC, the Korea Ocean Satellite Center for providing the GOCI L1B data to OBPG, and the Ocean Biology Processing Group at the Goddard Space Flight Center, NASA. 

%%%%%%%%%%%%%%%%%%% SECTION %%%%%%%%%%%%%%%%%%%%%%%%%%%%%%%%
% \vspace{-.4cm}
\section*{References}

%%%%%%%%%%%%%%%%%%%%%%%
%% Elsevier bibliography styles
%%%%%%%%%%%%%%%%%%%%%%%
%% To change the style, put a % in front of the second line of the current style and
%% remove the % from the second line of the style you would like to use.
%%%%%%%%%%%%%%%%%%%%%%%

%% Numbered
%\bibliographystyle{model1-num-names}

%% Numbered without titles
%\bibliographystyle{model1a-num-names}

%% Harvard
% \bibliographystyle{model2-names.bst}\biboptions{authoryear}

%% Vancouver numbered
%\usepackage{numcompress}\bibliographystyle{model3-num-names}

%% Vancouver name/year
%\usepackage{numcompress}\bibliographystyle{model4-names}\biboptions{authoryear}

%% APA style
\bibliographystyle{model5-names}\biboptions{authoryear}

%% AMA style
%\usepackage{numcompress}\bibliographystyle{model6-num-names}

%% `Elsevier LaTeX' style
% \bibliographystyle{elsarticle-num}
% \bibliographystyle{apalike}
\bibliography{/Users/jconchas/Documents/Latex/bib/javier_NASA.bib} 

% \listoffigures

% \listoftables

\end{document}