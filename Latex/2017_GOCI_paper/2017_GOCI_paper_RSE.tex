% \documentclass[twocolumn,3p]{elsarticle}
\documentclass[onecolumn,3p,letterpaper,11pt]{elsarticle}
\usepackage{setspace} % added by JC 
\doublespacing % added by JC


\usepackage{lineno} % uncomment \linenumbers after \begin{document}
\modulolinenumbers[1]


  
%%% MY PACKAGES %%%%%%%%%%%%%%%%%%%%%%%%%%%%%%%%%%%%%%%%%%%%%%%
\usepackage{graphicx}
% \usepackage[outdir=./]{epstopdf}
\usepackage{epstopdf}
\epstopdfsetup{update} % only regenerate pdf files when eps file is newer
\usepackage{amsmath,epsfig}

% Select what to do with todonotes: 
% \usepackage[disable]{todonotes} % notes not showed
% \usepackage[draft]{todonotes}   % notes showed
\usepackage[textwidth=2.5cm]{todonotes}
\presetkeys{todonotes}{fancyline, size=\small}{}
\setlength{\marginparwidth}{2.5cm}

\usepackage{tikz} % for flow charts
  \usetikzlibrary{shapes,arrows,positioning,shadows,calc}
  \usetikzlibrary{external}
  % \tikzexternalize

% \usepackage[nostamp]{draftwatermark}
% \SetWatermarkLightness{0.8}
% \SetWatermarkScale{4}

\usepackage[percent]{overpic}
\usepackage{morefloats} % for the error "Too many unprocessed floats"

\usepackage{multirow}

\renewcommand*{\bibfont}{\normalsize}

\usepackage{float}
\usepackage{hyperref}
%%% END MY PACKAGES %%%%%%%%%%%%%%%%%%%%%%%%%%%%%%%%%%%%%%%%%%%

\journal{Remote Sensing of Environment}

\begin{document}

% \linenumbers

\begin{frontmatter}

\title{Assessing Diurnal Variability of Biogeochemical Processes using the Geostationary Ocean Color Imager (GOCI)}

% %% Group authors per affiliation:
% \author{Javier A. Concha\fnref{myfootnote}}
% \address{Radarweg 29, Amsterdam}
% \fntext[myfootnote]{Since 1880.}

%% or include affiliations in footnotes:
\author[oeladdress,usraaddress]{Javier Concha\corref{mycorrespondingauthor}}
\cortext[mycorrespondingauthor]{Corresponding author at: Ocean Ecology Lab,
NASA Goddard Space Flight Center,
8800 Greenbelt Rd, Greenbelt, MD 20771, USA. Tel.: +1 585 290 3145.}
\ead{javier.concha@nasa.gov}

\author[oeladdress]{Antonio Mannino}

\author[oeladdress]{Bryan Franz}

% \author[oeladdress]{Amir Ibrahim}

\author[kiostaddress]{Wonkook Kim}

% \author[usgsaddress]{Michael Ondrusek}

\address[oeladdress]{Ocean Ecology Lab, NASA Goddard Space Flight Center, Greenbelt, MD, USA}
\address[usraaddress]{Universities Space Research Association, Columbia, MD, USA}
\address[kiostaddress]{Korea Institute of Ocean Science and Technology, 787 Haean-ro, Ansan, Republic of Korea}

% \address[usgsaddress]{NOAA/NESDIS Center for Weather and Climate Prediction, College Park, Maryland, USA}
% ===============================================================
\begin{abstract}

% Background/motivation/context
Short-term (hours) biological and biogeochemical processes cannot be captured by heritage ocean color satellites because their temporal resolution is limited to potentially one clear image per day. 
%
Geostationary satellites, such as the Geostationary Ocean Color Imager (GOCI), allow the study of these short-term processes because their orbits permits the collection of multiple images throughout each day. 
% Aim/objectives(s)/problem statement
In order to be able to detect the changes in the water properties caused by these processes, the levels of uncertainties introduced by the instrument and/or algorithms need to be assessed first.
% 
This work presents a study of the variability during the day over a water region of low-productivity with the assumption that only small changes in the water properties occur during the day over the area of study. 
% Methods
The complete GOCI mission data were processed to level 2 using the SeaDAS/l2gen package.
%
Filtering criteria were applied to assure the quality of the data. 
%
Relative differences with respect to the daily mean were calculated for each time of the day. 
%
Also, the relationship between the solar zenith angle and remote sensing reflectances was analyzed.
%
The GOCI time series was compared to the MODIS/Aqua and Suomi-NPP VIIRS missions.
% Results
Preliminary results suggest that the last two images of the day deviate significantly from the prior six hourly images, presenting errors on the order of $30\%$ or higher in the blue and green bands, and higher than $50\%$ in the red bands. 
%
Additionally, the atmospheric correction begins to fail for solar zenith angles greater than 60 degrees. 
% Conclusions

%


\end{abstract}

\begin{keyword}
Geostationary Ocean Color Imager\sep GOCI\sep diurnal dynamics\sep diurnal variability\sep CDOM
\end{keyword}

\end{frontmatter}
%%%%%%%%%%%%%%%%%%% SECTION %%%%%%%%%%%%%%%%%%%%%%%%%%%%%%%%
\section{Introduction}
GOCI as a precursor to GEO-CAPE. 
Name GOCI-II and the ESA geostationary satellite.

Spinning Enhanced Visible and Infrared Imager (SEVIRI) \citet{Neukermans2009,Neukermans2012}. Geostationary review \citet{Ruddick2014}. Synergy \citet{Vanhellemont2014}.

Example of citations \citet{Ryu2011,He2013,Hu2016}

Biological and biogeochemical processes in the coastal zones and oceans can occur in short-term time scales (minutes to hours). Heritage ocean color sensors do not have the temporal resolution needed to capture these diurnal dynamics.

In this paper, a uncertainties analysis was performed ...

The SeaDAS/l2gen code was modified to support NASA standard atmospheric correction for GOCI.
%%%%%%%%%%%%%%%%%%%%% SECTION %%%%%%%%%%%%%%%%%%%%%%%%%%%%%%%%
% \section{Data and Methodology}
\section{Data and Sensor Characteristics}
% ============================================================
\subsection{GOCI Data}
The Geostationary Ocean Color Imager (GOCI), launched in June 2010, is the only geostationary ocean color sensor currently on space \citep{Ryu2012}. GOCI monitors the Northeast Asian waters surrounding the Korean peninsula, generating up to eight images per day (from 00:15 Greenwich Mean Time (GMT) to 07:45 GMT at one hour rate) with a spatial resolution of 500 m at 130$^o$E and 36$^o$N. It covers an area of about 2500 km$\times$2500 km. It hast eight spectral bands (6 VIS: 412, 443, 490, 555, 660 and 680 nm; 2 NIR: 745 and 865 nm). GOCI operates in a 2D staring-frame capture mode in a geostationary orbit on board of the Communication Ocean and Meteorological Satellite (COMS) of South Korea. The acquisition of the observational coverage area of GOCI is done with a step-and-staring method that divides the image in 16 slots acquired in a sequential fashion with a dedicated CMOS detector array ($1400\times1400$ pixels).

The images used in this analysis span from the beginning of the GOCI's mission (June 2010) until December $31^{st}$, 2016\todo{update}, resulting in a total of about 18,000 images. The GOCI Level-1B calibrated top-of-atmosphere (TOA) radiance data were obtained from the Ocean Biology Distributed Active Archive Center (OB.DAAC\todo{correct?}) at the NASA's Goddard Space Flight Center, maintained by the Ocean Biology Processing Group (OBPG) \todo{or OEL?}. The OB.DAAC acts as a mirror site for the GOCI data provided by the Korea Ocean Satellite Center. These data are freely available for direct download from the OB.DAAC. Each file is provided as a  Hierarchical Data Format Release 5 format (file extension: .he5) and corresponds to one of the eight daily images per day. Each file contains eight images corresponding to the eight spectral bands.
% ---------------------------------------------------------------
\subsection{Area of Study}

The area of study (\autoref{fig:AreaOfStudy}) is a boxed area that covers an open-ocean region of oligotrophic waters located to the south of Japan, in the North Pacific Ocean, with the boundaries north$=29.4736^o$, south$=24.2842^o$, west$=131.9067^o$, and east$=142.3193^o$, centered at $27^oN$ and $137^oE$. \autoref{fig:AreaOfStudy} shows the area of study (white box) and the GOCI coverage area (red box) for reference. The area of study is approximately $2000\times 1000$ GOCI pixels, and therefore, covering approximately two million GOCI pixels, equivalent to $500,000\ km^2$. The reasoning behind the selection of this area of study is the assumption that most of the variability in this region during the day will be caused by physical (e.g. wind, waves) changes and not by biogeochemical processes (e.g. $CO_2$ fixation). In this manner the variability due to water composition will be minimized, and therefore, the variability introduced by the sensor, viewing geometry and algorithm can be analyzed. For this region, the range of solar zenith angle during the acquisition time  varies between approximately $0^o$ to $90^o$ through the year, and between approximately $29^o$ to $37^o$ for the sensor (viewing) zenith angle.

% ($\text{Number Total Pixels } (NTP) = (499*2+1)*(999*2+1) =  1,997,001$ pixels)
% ---------------------------------------------------------------
\begin{figure}[ht]
	\centering
    \includegraphics[height=5.5cm]{./Figures/GOCI_ROI_footprint.png}
	\caption{Area of study. GOCI foot print (red box) and the area of study (white box). This area was selected because the assumption that the most of the daily variability is caused by physical factors.}
	\label{fig:AreaOfStudy}
\end{figure}
%%%%%%%%%%%%%%%%%%%%% SECTION %%%%%%%%%%%%%%%%%%%%%%%%%%%%%%%%
\section{Processing Approach}
\label{sec:processing}
% ---------------------------------------------------------------
\subsection{Conversion to Level 2}
GOCI Level 1B data (radiometric calibrations applied, L1B) was processed to Level 2 data (geolocated, geophysical values, L2) using the multisensor level 1 level 2 generator (l2gen) version 8.10.2 distributed with the SeaWiFS Data Analysis System (SeaDAS) (\url{http://seadas.gsfc.nasa.gov/}). The l2gen code reads level 1 observed top-of-atmosphere (TOA) radiances, applies one of the atmospheric correction scheme available, and output various products such as radiances or reflectances (e.g.  spectral remote-sensing reflectance, $R_{rs}(\lambda)$) and derived geophysical parameter (e.g. chlorophyll-{\it a} concentration). As part of the l2gen processing, each pixel is masked with different flags that reflect warnings or errors generated during the processing. Examples of these flags are atmospheric correction failure, land, cloud or ice, straight light, sun glint, high top-of-atmosphere radiance, among others \citep{Bailey2006}. 
% ATMFAIL
% LAND
% PRODWARN
% HIGLINT
% HILT
% HISATZEN
% COASTZ
% STRAYLIGHT
% CLDICE
% COCCOLITH
% TURBIDW
% HISOLZEN
% LOWLW
% CHLFAIL
% NAVWARN
% ABSAER
% MAXAERITER
% MODGLINT
% CHLWARN
% ATMWARN
% SEAICE
% NAVFAIL
% FILTER
% BOWTIEDEL
% HIPOL
% PRODFAIL

The atmospheric correction scheme applied for this study was the default algorithm ({\ttfamily aer\_opt=-2}) that uses an estimation of the aerosol contribution described by \citet{Gordon1994}, including a near infrared (NIR) iterative correction by \citet{Bailey2010} and a selection of the aerosol model dependent of the relative humidity by \citet{Ahmad2010}. GOCI's two near infrared (NIR) bands were used for the aerosol models selection. This approach assumes a plane-parallel geometry, ignoring earth curvature, for the vector radiative transfer simulations used for the computation of the {\todo{taken from Franz 2015} look-up tables of Rayleigh reflectance}. \todo{include BRDF correction} 

A vicarious calibration specific for GOCI was applied (included in SeaDAS). The calibration coefficients are based on match-up with VIIRS and they were derived by the National Research Lab based on AERONET-OC {\it in situ} data.
% ---------------------------------------------------------------
\subsection{Data screening}
% Filtering Criteria
The mean and standard deviation for the whole area of study are calculated for each level 2 product. In order to assure the quality of the data, an exclusion criteria (filtering) was applied. This criteria was based on the criteria described by \citet{Bailey2006} for the validation of ocean color satellite data products. This exclusion criteria is presented in the flowchart in \autoref{fig:FilteringCriteria}. Pixels flagged by the atmospheric correction algorithm are excluded. In order to avoid the effect of outliers in the calculations, the following screening criteria was applied:
\begin{equation}\label{eq:filtered_value}
  (Med-1.5*\sigma) <  X_i < (Med+1.5*\sigma)
\end{equation}
where $X_i$ is the $i^{th}$ filtered pixel within the box, $Med$ is the median value of the unflagged pixels, and $\sigma$ is standard deviation of the unflagged pixels. Then, the filtered mean was calculated:
\begin{equation}\label{eq:filtered_mean}
  \text{Filtered Mean} =\frac{\displaystyle \sum_i^{NFP} X_i}{NFP}
\end{equation}
where $NFP$ is the Number Filtered Pixels, i.e. the number of unflagged values within $\pm 1.5*\sigma$. Note the difference with Equation 4 in \citet{Bailey2006}, where the mean of the unfiltered data was used instead of the median. The use of the median value for the calculation of the filtered mean minimize the influence of outliers. This is the current standard approach adopted by the Ocean Ecology Lab at the NASA Goddard Space Flight Center. The statistical values filtered mean and standard deviation, among others, were computed operationally using the val-extract tool included in the SeaDAS/l2gen distribution (located on {\ttfamily \$OCSSWROOT/run/bin/macosx\_intel/val\_extract}  for the OSX installation) specifying the latitude and longitude limits as input parameters (slon, elon, slat and elat).

In order to have statistical confidence in the filtered mean value, NFP is required to be at least half the number of total pixels in the box (i.e. $NFP\geq NTP/2 = 998,501$) to be considered into the analysis. This is equivalent to stating that at least half of the area of study has valid pixel values associated with it. Additionally, we excluded data where the solar and viewing zenith angle of the center of the pixel box exceeded $75^o$ and $60^o$ to avoid extreme solar and  viewing geometries \citep{Bailey2006} (\autoref{fig:FilteringCriteria}). 

A coefficient of variation (CV), which is defined as the filtered standard deviation divided by the filtered mean, is calculated for all the visible bands and for the aerosol optical thickness at 865 nm, and the median CV is recorded. Then, the mean and standard deviation of the median CV are calculated for the whole mission ($mean_{Median(CV)}=0.23$ and $\sigma_{Median(CV)}=0.26$, respectively). Finally, level 2 products with median CV greater than $mean_{Median(CV)}+\sigma_{Median(CV)}=0.49$ are excluded.
% ---------------------------------------------------------------
% %%%%%%%%%%%%%%%%%%%%%%%%%%%%%%%%%%%%%%%%%%%%%%%%%%%%%%%%%%%%%%%%%%%%%%%
\begin{figure}[ht]
\vspace{1cm}
  \centering
  % \small
\resizebox{16cm}{!}{%
% Define block styles

% \tikzstyle{startstop} = [rectangle, rounded corners, minimum width=2.5em, minimum height=2em,text centered, draw=black, fill=white]

\tikzstyle{input} = [trapezium, trapezium left angle=70, trapezium right angle=110, minimum width=2em, minimum height=2em, text width=3cm, text centered, draw=black, fill=white]

\tikzstyle{input} = [trapezium, trapezium left angle=70, trapezium right angle=110, minimum width=2em, minimum height=2em, text width=3cm, text centered, draw=black, fill=white]

\tikzstyle{inputsmall} = [trapezium, trapezium left angle=70, trapezium right angle=110, minimum width=1em, minimum height=1em, text width=1.0cm, text centered, draw=black, fill=white]

\tikzstyle{output} = [rounded rectangle, minimum width=3em, minimum height=3em, text width=4.5cm, text centered, draw=black, fill=white, inner sep=-1pt]

\tikzstyle{process} = [rectangle, minimum width=3em, minimum height=2em, text width=6cm, text centered, draw=black, fill=white]

\tikzstyle{process_small} = [rectangle, minimum width=3em, minimum height=2em, text width=4cm, text centered, draw=black, fill=white]

\tikzstyle{decision} = [diamond, aspect=2, minimum width=2em, minimum height=2em, text width=5cm, text centered, draw=black, fill=white]

\tikzstyle{arrow} = [thick,->,>=triangle 45]
\tikzstyle{arrowdashed} = [thick,dashed,->,>=stealth]

\begin{tikzpicture}[node distance=2.5cm]


% \node (start) [startstop] {Start};

\node (L1B) [input] {Select L1B file to be \\processed to L2};
\node (l2gen) [process, below of=L1B] {Process to L2 using l2gen:\\ - Extract ROI\\- Atmospheric Correction};
\node (flag) [process, below of=l2gen, yshift=0.5cm] {Exclude flagged pixels};
\node (filtered) [process, below of=flag, yshift=-0.5cm] {Filtered Mean $=\frac{\displaystyle \sum_i^{NFP} X_i}{\displaystyle NFP}$\vspace{.3cm}\\$Med- 1.5*\sigma \leq X_i\leq Med+ 1.5*\sigma$\vspace{.3cm}\\Number Filtered Pixels (NFP)};
\node (enough) [decision, right of=L1B, inner sep=1pt, xshift=6cm] {Is $NFP > NTP/2$?\\ Number Total Pixels (NTP)};

\node (fail) [output, right of=enough, xshift=4.5cm] {Failed filtering criteria};

\node (zenith) [decision, below of=enough, yshift=-2cm] {Solar Zenith $< 75^o$ and \\ Sensor Zenith $< 60^o$?};

\node (CV) [decision, below of=zenith, yshift=-2cm] {Median[CV]$<0.49$?};

\node (bestgeom) [process_small, right of=CV, xshift=7.5cm] {If multiple L2 files for \\the same time, choose the best geometry};

\node (pass) [output, above of=bestgeom, yshift=0cm] {Passes filtering criteria};

\node (out) [input, above of=pass, yshift=0cm, inner sep=6pt] {Filtered Mean and Filtered Standard Deviation};


\draw [arrow] (L1B) -- (l2gen);
\draw [arrow] (l2gen) -- (flag);
\draw [arrow] (flag) -- (filtered);
\draw [arrow] (filtered.east) -|([xshift=-1cm]enough.west) -- (enough.west);

\draw [arrow] (enough) -- node[anchor=south] {NO} (fail);
\draw [arrow] (zenith) -| node[anchor=south,xshift=-3.0cm] {NO} (fail);
\draw [arrow] (CV) -| node[anchor=south,xshift=-3.0cm] {NO} (fail);
\draw [arrow] (enough) -- node[anchor=east] {YES} (zenith);
\draw [arrow] (zenith) -- node[anchor=east] {YES} (CV);
% \draw [arrow] (CV.south) -| node[anchor=east] {YES} (bestgeom.south);
\draw [arrow] (CV.south) -| node[anchor=east,yshift=-0.5cm] {YES} ([yshift=-1cm]CV.south) -| (bestgeom.south);
\draw [arrow] (bestgeom) -- (pass);
\draw [arrow] (pass) -- (out);

\end{tikzpicture}
} % resizebox end
	\caption{Applied exclusion criteria}
	\label{fig:FilteringCriteria}
\end{figure}
% ---------------------------------------------------------------
\subsection{Bio-Optical Algorithms} \label{subsec:bioopalg}
% ------------------------
\subsubsection{Chlro-a}
OCx and \citet{Hu2012}.
% ------------------------
\subsubsection{POC}
\citet{Stramski2008}.
% ------------------------
\subsubsection{ag(412)mlrc}
\citet{Mannino2014}.

% % ---------------------------------------------------------------
% \subsection{Vicarious Calibration?}
% Vicarious calibration was applied.
%%%%%%%%%%%%%%%%%%% SECTION %%%%%%%%%%%%%%%%%%%%%%%%%%%%%%%%
\section{Results}
\label{sec:Results}
% ============================================================
% ---------------------------------------------------------------
\subsection{Validation of the Atmospheric Correction}
% - - - - - - - - - - - - - - - - - - - - - - - - - - - - - - - -
\subsubsection{AERONET-OC}
The atmospheric correction was validated using {\it in situ} observations from the AErosol RObotic NETwork-Ocean Color (AERONET-OC) as ground truth \citep{Zibordi2009}. The quality-assurance (QA) level used was level 2.0, which is the highest quality for the AERONET-OC data. The dataset from two stations were used for the analysis: Gageocho (N=20; PI: Jae-Seol Shim and Joo-Hyung Ryu) and Ieodo (N=25; PI: Young-Je Park and Hak-Yeol You). \autoref{fig:GOCI_AERO} shows scatter plots for satellite derived $R_{rs}(\lambda)$ versus {\it in situ} measurements. 

The selection of matchups followed the satellite validation protocols described in \citet{Bailey2006}. GOCI data acquired within a three hours windows of the {\it in situ} sampling were considered as potential validation matchups. A $7\times7$ GOCI pixel array is extracted centered in the {\it in situ} sampling location. A filtered mean is calculated from these $7\times7$ arrays using \autoref{eq:filtered_value} and \autoref{eq:filtered_mean} \citep{Bailey2006}. A minimum of at least half the total of pixels in the $7\times7$\todo{From Antonio: Why 7x7 and not 5x5?}~pixel array, i.e. $49/2\approx25$ pixels, were required to be valid (unflagged) for inclusion of the matchup in the validation analysis. Additionally, a coefficient of variation (CV, filtered mean divided by the filtered standard deviation) for the visible bands 412 to 555 nm and the aerosol optical thickness (AOT) at 864 nm was calculated for each pixel array that passes the exclusion criteria described above, and then, the median value of these coefficients of variation ($\text{Median}[CV]$) was calculated. Finally, the pixel arrays whose $\text{Median}[CV]>0.15$, as suggested by \citep{Bailey2006}, are excluded from the validation analysis.

A validation analysis was performed by comparing the satellite-derived retrievals of products with the {\it in situ} observations based on different statistical parameters (\autoref{tab:val_stats}). These statistical parameters are: the slope and offset of the fitted Reduced Major Axis (RMA) regression line of the form $y=m*x+b$, and its coefficient of determination $R^2$, the root mean squared error (RMSE), the mean, standard deviation, and median of the absolute percentage difference (APD)(MAPD, $\pm$sd APD, and Median APD, respectively), the percentage bias, median ratio of computed filtered mean satellite value ($R_{rs:ret}$) to {\it in situ} measurement ($R_{rs:in}$), and the semi-interquartile range (SIQR) \citep{Bailey2006}. These parameters are defined as:
\begin{equation}
  \text{APD$_n$(\%)}=\left[\frac{\displaystyle \left|R_{rs:in}^1-R_{rs:ret}^1 \right|}{R_{rs:in}^1},\dots,\frac{\displaystyle \left|R_{rs:in}^n-R_{rs:ret}^n \right|}{R_{rs:in}^n},\dots,\frac{\displaystyle \left|R_{rs:in}^N-R_{rs:ret}^N \right|}{R_{rs:in}^N}\right]*100,\ n=1,\dots,N
\end{equation}
\noindent where N is the total number of matchups, and $R_{rs:in}^n$ is the $n^{th}$ {\it in situ} $R_{rs}$ and $R_{rs:ret}^n$ is the $n^{th}$ satellite-derived $R_{rs}$.
\begin{equation}
  \text{MAPD(\%)} = \frac{1}{N} \sum_{n=1}^{N} \text{APD$_n$(\%)}
\end{equation}
\begin{equation}
  \text{$\pm$ \text{sd} APD(\%)} =  SD\left[\text{APD$_n$(\%)}\right]
\end{equation}
\noindent with $SD$ the standard deviation.
\begin{equation}
  \text{\text{Median} APD(\%)} =  Median\left[\text{APD$_n$(\%)}\right]
\end{equation}
\begin{equation}
   \text{RMSE} = \sqrt{\frac{\displaystyle \sum_{n=1}^{N} \left(R_{rs:in}^n-R_{rs:ret}^n\right)^2}{N}}
\end{equation} 
\begin{equation}
    \text{\% Bias} = \frac{\displaystyle \frac{1}{N}*\sum_{n=1}^N(R_{rs:ret}^n-R_{rs:in}^n)}{\text{Mean}[R_{rs:in}^n]}*100
\end{equation}
\begin{equation}
  \text{Median ratio} =  Median\left[\frac{R_{rs:ret}^1}{R_{rs:in}^1},\dots,\frac{R_{rs:ret}^n}{R_{rs:in}^n},\dots,,\frac{R_{rs:ret}^N}{R_{rs:in}^N}\right],\ n=1,\dots,N
\end{equation}
\begin{equation}
    \text{SIQR} = \frac{Q_3-Q_1}{2}
\end{equation}
\noindent where $Q_3$ and $Q_1$ are the $75^{th}$ and $25^{th}$ percentiles for the ratios of the satellite-derived values to the {\it in situ} measurements.  

\autoref{fig:GOCI_AERO} shows the color coded AERONET-OC data divided by group representing the times of the day (early morning: 0h and 1h (blue circles); early midday: 2h and 3h (red circles); late midday: 4h and 5h (green circles); and late afternoon: 6h and 7h (cyan circles)) in order to evaluate the influence of the solar zenith angle in the validation matchups. Also, the data were separated by stations (Gaeocho: circles and Ieodo: triangles). The statistics calculated for each group and for all matchups (highlighted in bold cases) are shown in (\autoref{tab:val_stats}). The largest MAPD value correspond to the 0h\&1h hour for all bands and the smallest MAPD value correspond to the 4h\&5h suggesting that is an influence of the solar zenith angle. However, the number of matchups decreases throughout the day, having only 2 matchups at the end of the day, and therefore, there are not enough data to be conclusive.
% \begin{table}[htbp!]
% \caption{Satellite validation statistics of the atmospheric correction algorithm for GOCI. Satellite-derived values were compared with {\it in situ} observations from the AERONET-OC. Regression line of the form $y=m*x+b$ using the Reduced Major Axis (RMA). \label{tab:val_stats} }
% \scriptsize
% \centering
% \begin{tabular}{cccccccccccccc} 
%  \hline 
% $\lambda_{sat} (\lambda_{\text in situ})$ & $R^2$ & \multicolumn{2}{c}{RMA Regression} & RMSE & N & MAPD & $\pm$sd & Median  & Bias & Median & SIQR \\ \cline{3-4} 
% (nm) & &  $m$  & $b$ & ($sr^{-1}$) & & ($\%$) & APD ($\%$) & APD ($\%$) & ($\%$) & ratio &  & \\ \hline 
% 412 (412) & 0.81 & 0.83 & -0.0002 & 0.0021 & 45 & 40.1 & 38.7 & 27.5 & -20.4 & 0.83 & 0.14 \\ 
% 443 (443) & 0.91 & 0.91 & -0.0003 & 0.0017 & 45 & 28.0 & 25.1 & 20.1 & -13.2 & 0.86 & 0.14 \\ 
% 490 (490) & 0.95 & 0.82 & -0.0004 & 0.0029 & 45 & 25.7 & 16.3 & 21.9 & -21.0 & 0.79 & 0.09 \\ 
% 555 (555) & 0.98 & 0.80 & -0.0003 & 0.0029 & 45 & 22.8 & 10.1 & 21.4 & -22.4 & 0.78 & 0.07 \\ 
% 660 (665) & 0.97 & 0.90 & -0.0003 & 0.0007 & 45 & 39.3 & 28.2 & 27.1 & -22.2 & 0.72 & 0.25 \\ 
% \hline 
% \end{tabular}
% \end{table}

\begin{table}[htbp!]
\caption{Satellite validation statistics of the atmospheric correction algorithm for GOCI. Satellite-derived values were compared with {\it in situ} observations from the AERONET-OC. Regression line of the form $y=m*x+b$ using the Reduced Major Axis (RMA). The statistics for all the matchups are highlighted in bold cases. \label{tab:val_stats} }
\tiny
\centering
\begin{tabular}{ccccccccccccc} 
 \hline 
$\lambda_{sat} (\lambda_{\text in situ})$ & Time of & $R^2$ & \multicolumn{2}{c}{RMA Regression} & RMSE & N & MAPD & $\pm$sd & Median & Bias & Median & SIQR \\ \cline{4-5}
[nm]([nm])                  &  the day            &         & $m$     & $b$     &             &     & ($\%$)  & APD ($\%$)  & APD ($\%$)  & ($\%$)   & ratio   &         \\ \hline 
\multirow{5}{*}{412 (412)}  &  0h $\&$ 1h  & 0.85    & 0.86    & -0.0009 & 0.0018      & 15  & 57.3    & 39.5        & 54.6        & -35.5    & 0.67    & 0.384   \\ 
                            &  2h $\&$ 3h  & 0.84    & 0.75    & +0.0012 & 0.0015      & 15  & 38.4    & 47.2        & 16.0        & - 3.1    & 0.92    & 0.346   \\ 
                            &  4h $\&$ 5h  & 0.99    & 0.79    & +0.0008 & 0.0012      & 6   & 10.5    &  5.6        & 12.0        & -12.0    & 0.87    & 0.044   \\ 
                            &  6h $\&$ 7h  & 1       & 3.12    & -0.0282 & 0.0036      & 2   & 30.9    &  6.7        & 30.9        & -30.8    & 0.69    & 0.047   \\ 
                              &  \textbf{All}         & \textbf{0.81}    & \textbf{0.83}    & \textbf{-0.0002} & \textbf{0.0021}      & \textbf{45}  & \textbf{40.1}    & \textbf{38.7}        & \textbf{27.5}        & \textbf{-20.4}    & \textbf{0.83}    & \textbf{0.146}   \\ \hline 
\multirow{5}{*}{443 (443)}  &  0h $\&$ 1h  & 0.96    & 0.90    & -0.0007 & 0.0014      & 16  & 34.6    & 25.4        & 37.6        & -23.4    & 0.62    & 0.224   \\ 
                            &  2h $\&$ 3h  & 0.86    & 0.85    & +0.0009 & 0.0017      & 14  & 33.2    & 30.0        & 22.1        & - 2.7    & 0.99    & 0.433   \\ 
                            &  4h $\&$ 5h  & 0.99    & 0.88    & +0.0006 & 0.0008      & 6   &  4.9    &  3.3        &  6.0        & - 5.7    & 0.93    & 0.033   \\ 
                            &  6h $\&$ 7h  & 1       & 0.62    & +0.0025 & 0.0028      & 2   & 19.8    &  1.2        & 19.8        & -19.9    & 0.80    & 0.008   \\ 
                              &  \textbf{All}         & \textbf{0.91}    & \textbf{0.91}    & \textbf{-0.0003} & \textbf{0.0017}      & \textbf{45}  & \textbf{28.0}    & \textbf{25.1}        & \textbf{20.1}        & \textbf{-13.2}    & \textbf{0.86}    & \textbf{0.145}   \\ \hline 
\multirow{5}{*}{490 (490)}  &  0h $\&$ 1h  & 0.98    & 0.79    & -0.0004 & 0.0024      & 16  & 31.3    & 21.4        & 29.1        & -26.8    & 0.70    & 0.094   \\ 
                            &  2h $\&$ 3h  & 0.91    & 0.82    & +0.0001 & 0.0027      & 14  & 22.9    & 14.5        & 22.7        & -15.9    & 0.89    & 0.179   \\ 
                            &  4h $\&$ 5h  & 0.99    & 0.84    & -0.0001 & 0.0026      & 6   & 16.1    &  0.9        & 15.9        & -16.0    & 0.84    & 0.007   \\ 
                            &  6h $\&$ 7h  & 1       & 2.04    & -0.0263 & 0.0055      & 2   & 27.9    &  3.1        & 27.9        & -27.8    & 0.72    & 0.022   \\ 
                              &  \textbf{All}         & \textbf{0.95}    & \textbf{0.82}    & \textbf{-0.0004} & \textbf{0.0029}      & \textbf{45}  & \textbf{25.7}    & \textbf{16.3}        & \textbf{21.9}        & \textbf{-21.0}    & \textbf{0.79}    & \textbf{0.095}   \\ \hline 
\multirow{5}{*}{555 (555)}  &  0h $\&$ 1h  & 0.97    & 0.80    & -0.0004 & 0.0024      & 16  & 26.4    &  9.7        & 28.0        & -24.6    & 0.71    & 0.066   \\ 
                            &  2h $\&$ 3h  & 0.98    & 0.80    & -0.0002 & 0.0027      & 14  & 19.6    & 11.3        & 18.1        & -21.7    & 0.81    & 0.092   \\ 
                            &  4h $\&$ 5h  & 0.99    & 0.75    & +0.0009 & 0.0034      & 6   & 17.4    &  4.2        & 19.6        & -18.7    & 0.80    & 0.026   \\ 
                            &  6h $\&$ 7h  & 1       & 0.61    & +0.0027 & 0.0049      & 2   & 24.7    &  1.0        & 24.7        & -24.7    & 0.75    & 0.007   \\ 
                              &  \textbf{All}         & \textbf{0.98}    & \textbf{0.80}    & \textbf{-0.0003} & \textbf{0.0029}      & \textbf{45}  & \textbf{22.8}    & \textbf{10.1}        & \textbf{21.4}        & \textbf{-22.4}    & \textbf{0.78}    & \textbf{0.072}   \\ \hline 
\multirow{5}{*}{660 (665)}  &  0h $\&$ 1h  & 0.98    & 0.96    & -0.0004 & 0.0005      & 16  & 55.1    & 28.8        & 65.5        & -24.0    & 0.34    & 0.264   \\ 
                            &  2h $\&$ 3h  & 0.98    & 0.81    & -0.0002 & 0.0008      & 14  & 40.0    & 22.2        & 31.7        & -26.8    & 0.68    & 0.187   \\ 
                            &  4h $\&$ 5h  & 0.99    & 0.84    & +0.0002 & 0.0005      & 6   &  9.2    &  5.8        & 11.1        & -11.6    & 0.88    & 0.053   \\ 
                            &  6h $\&$ 7h  & 1       & 1.56    & -0.0042 & 0.0009      & 2   & 17.5    &  2.4        & 17.5        & -17.4    & 0.82    & 0.017   \\ 
                              &  \textbf{All}         & \textbf{0.97}    & \textbf{0.90}    & \textbf{-0.0003} & \textbf{0.0007}      & \textbf{45}  & \textbf{39.3}    & \textbf{28.2}        & \textbf{27.1}        & \textbf{-22.2}    & \textbf{0.72}    & \textbf{0.253}   \\ 
\hline 
\end{tabular}
\end{table}
% ++++++++++++++++++++++++++++++++++++++++++
% Sat (nm) InSitu (nm) R^2     m      b      RMSE       N   Mean APD (%) St.Dev. APD (%) Median APD (%) Bias (%)  Median ratio SIQR      
% 412      412         0.8677  0.9725 0.0013 0.0017938  45  53.0129      82.8063         20.4735        17.6568   1.1561       0.20052   
% 443      443         0.9396  1.0090 0.0011 0.0017186  45  38.7793      47.7934         17.5797        15.2017   1.158        0.18159   
% 490      490         0.96532 0.8851 0.0002 0.0018144  45  17.6378      15.7819         13.2356        -9.4884   0.91378      0.087913  
% 555      555         0.98747 0.8456 0.0003 0.0024224  45  19.0986      8.7084          18.1675        -18.0058  0.81832      0.054675  
% 665      660         0.95841 1.0226 0.0004 0.00066126 45  59.005       77.528          22.5573        16.3546   1.2202       0.32806    

%-%-%-%-%-%-%-%-%-%-%=FIGURE=%-%-%-%-%-%-%-%-%-%-%-%-%
\begin{figure}[H]
    \begin{minipage}[c]{0.32\linewidth}
      \centering
       \begin{overpic}[trim=0 0 0 0,clip,height=4.0cm]{./Figures/GOCI_AERO_412.eps} \put (85,20) {\colorbox{white}{(a)}}
       \end{overpic}
    \end{minipage}  
    \hspace{-1.0cm}
    \begin{minipage}[c]{0.32\linewidth}
      \centering
       \begin{overpic}[trim=0 0 0 0,clip,height=4.0cm]{./Figures/GOCI_AERO_443.eps} \put (85,20) {\colorbox{white}{(b)}}
       \end{overpic}
    \end{minipage}  
    \hspace{-1.0cm}
    \begin{minipage}[c]{0.32\linewidth}
      \centering
      \hspace{1cm}
       \begin{overpic}[trim=0 0 0 0,clip,height=4.0cm]{./Figures/GOCI_AERO_490.eps} \put (85,20) {\colorbox{white}{(c)}}
       \end{overpic}
    \end{minipage}  

    \vspace{0.5cm}

    \begin{minipage}[c]{0.32\linewidth}
      \centering
       \begin{overpic}[trim=0 0 0 0,clip,height=4.0cm]{./Figures/GOCI_AERO_555.eps} \put (85,20) {\colorbox{white}{(d)}}
       \end{overpic}
    \end{minipage}  
    \hspace{-1.0cm}
    \begin{minipage}[c]{0.32\linewidth}
      \centering
       \begin{overpic}[trim=0 0 0 0,clip,height=4.0cm]{./Figures/GOCI_AERO_660.eps} \put (85,20) {\colorbox{white}{(e)}}
       \end{overpic}
    \end{minipage}   
    \hspace{-1.0cm}
    \begin{minipage}[c]{0.32\linewidth}
      \centering
       \begin{overpic}[trim=0 0 0 0,clip,height=3.0cm]{./Figures/LegendScatterAERO_2.pdf}
       \end{overpic}
    \end{minipage} 

    \caption{Scatter plots showing the comparison between the satellite-derived GOCI values and AERONET-OC {\it in situ} observations (Gaeocho: circles; Ieodo: triangles). The dashed black line is the 1:1 line, and the Reduced Major Axis (RMA) regression line is drawn in red. Color code: early morning: 0h and 1h (blue circles); early midday: 2h and 3h (red circles); late midday: 4h and 5h (green circles); and late afternoon: 6h and 7h (cyan circles). \label{fig:GOCI_AERO} } 
\end{figure}

When all the data are used for the analyses, a good agreement was found between the retrieved $R_{rs}$ and {\it in situ} observations, with $R^2$ values varying from 0.81 to 0.98 for the 412 to 660 nm GOCI bands, exceeding the values previously reported by \citet{Ahn2015}\todo{From Antonio: it's not worthwhile to compare individual statistics metrics from another paper, but rather a group a statistics; our results show better (or worse) agreement than Ahn et al. 2015 based a combination of metrics (R2, RSME, APD, etc.).}. The validation statistics indicate that the atmospheric correction did not perform as well for satellite retrieval of $R_{rs}$ when compared to the atmospheric correction described by \citet{Ahn2015} that includes a vicarious calibration using {\it in situ} data, which is reflected in a consistently greater mean absolute percent difference (MAPD) and RMSE for all bands. This suggests that the vicarious calibration used can be improved. 
%-%-%-%-%-%-%-%-%-%-%=END FIGURE=%-%-%-%-%-%-%-%-%-%-%-%-%
% - - - - - - - - - - - - - - - - - - - - - - - - - - - - - - - -
\subsubsection{Cruises Matchups?}
% ---------------------------------------------------------------
\subsection{Time Series}\label{subsec:timeseries}
As mentioned previously (\S\ref{sec:processing}.1-2), the complete GOCI mission was processed to Level 2 and then a filtered mean for the area of study was calculated. Only values that passed the filtering criteria described previously were used \citep{Bailey2006}. The filtered mean of the boxed area for each time of the day are shown in \autoref{fig:GOCI_TimeSeries}.(a-l), where results are plotted as a function of time and histograms for the $R_{rs}(\lambda)$ for all GOCI's bands. The data are color coded by time of the day in GMT (red: 0h, green: 1h, blue: 2h, black: 3h, cyan: 4h, magenta: 5h, orange: 6h, and purple: 7h). The time series exhibit an expected seasonal variability, which is correlated\todo{Check!}~to the changes in solar zenith angles throughout the year (\autoref{fig:GOCI_TimeSeries2}.(g)). Note that for each day there are potentially eight values displayed, which explains the daily spread of the data in \autoref{fig:GOCI_TimeSeries} and \autoref{fig:GOCI_TimeSeries2}. 

There some data that passed the exclusion criteria and have negative values for all bands, reflecting a failure on the atmospheric correction. However, these are few for the 412, 443, 490 and 550 nm spectral bands, and most of them were captured at the time of the day equal to 6h and during winter time, when the solar zenith are the largest. As for the 660 and 680 nm spectral bands, there are more negative values, with the 680 nm band having the largest amount. However, from the histogram for these bands, it can be seen that the mean values are close to zero.

\autoref{fig:GOCI_TimeSeries2}.(a-f) shows time series for the Chlorophyll-{\it a}, $a_{g:mlrc}(412)$ and POC, which were obtained from the filtered mean of $R_{rs}(\lambda)$ using the empirical algorithm described in \S\ref{subsec:bioopalg}.  The data are color coded by time of the day in GMT (red: 0h, green: 1h, blue: 2h, black: 3h, cyan: 4h, magenta: 5h, orange: 6h, and purple: 7h). \autoref{fig:GOCI_TimeSeries2}.(g) shows the time series for the solar zenith angle for reference and color coded by time of the day. As expected, there are more data for spring-summer than fall-winter, when the solar zenith angle is larger and the sky conditions are not optimal (e.g. presence of clouds).
%-%-%-%-%-%-%-%-%-%-%=FIGURE=%-%-%-%-%-%-%-%-%-%-%-%-%
\begin{figure}[H]
    \begin{minipage}[c]{0.66\linewidth}
      \centering
      \begin{overpic}[trim=0 352 0 0,clip,height=3.4cm]{./Figures/TimeSerie_Rrs412.eps} \put (90,28) {\colorbox{white}{(a)}}
      \end{overpic}
    \end{minipage}  
    \hfill
    \begin{minipage}[c]{0.33\linewidth}
      \centering
      \begin{overpic}[trim=0 0 0 0,clip,height=3.2cm]{./Figures/Hist_Rrs412.eps} \put (82,67) {\colorbox{white}{(b)}}
      \end{overpic} 
    \end{minipage}  

    \begin{minipage}[c]{0.66\linewidth}
      \centering
      \begin{overpic}[trim=0 352 0 0,clip,height=3.4cm]{./Figures/TimeSerie_Rrs443.eps} \put (90,28) {\colorbox{white}{(c)}}
      \end{overpic}
    \end{minipage}  
    \hfill
    \begin{minipage}[c]{0.33\linewidth}
      \centering
      \begin{overpic}[trim=0 0 0 0,clip,height=3.2cm]{./Figures/Hist_Rrs443.eps} \put (82,67) {\colorbox{white}{(d)}}
      \end{overpic} 
    \end{minipage}  

    \begin{minipage}[c]{0.66\linewidth}
      \centering
      \begin{overpic}[trim=0 352 0 0,clip,height=3.4cm]{./Figures/TimeSerie_Rrs490.eps} \put (90,28) {\colorbox{white}{(e)}}
      \end{overpic}
    \end{minipage}  
    \hfill
    \begin{minipage}[c]{0.33\linewidth}
      \centering
      \begin{overpic}[trim=0 0 0 0,clip,height=3.2cm]{./Figures/Hist_Rrs490.eps} \put (82,67) {\colorbox{white}{(f)}}
      \end{overpic} 
    \end{minipage}  

    \begin{minipage}[c]{0.66\linewidth}
      \centering
      \begin{overpic}[trim=0 352 0 0,clip,height=3.4cm]{./Figures/TimeSerie_Rrs555.eps} \put (90,28) {\colorbox{white}{(g)}}
      \end{overpic}
    \end{minipage}  
    \hfill
    \begin{minipage}[c]{0.33\linewidth}
      \centering
      \begin{overpic}[trim=0 0 0 0,clip,height=3.2cm]{./Figures/Hist_Rrs555.eps} \put (82,67) {\colorbox{white}{(h)}}
      \end{overpic} 
    \end{minipage}  

    \begin{minipage}[c]{0.66\linewidth}
      \centering
      \begin{overpic}[trim=0 352 0 0,clip,height=3.4cm]{./Figures/TimeSerie_Rrs660.eps} \put (90,28) {\colorbox{white}{(i)}}
      \end{overpic}
    \end{minipage}  
    \hfill
    \begin{minipage}[c]{0.33\linewidth}
      \centering
      \begin{overpic}[trim=0 0 0 0,clip,height=3.2cm]{./Figures/Hist_Rrs660.eps} \put (82,67) {\colorbox{white}{(j)}}
      \end{overpic} 
    \end{minipage}  

    \begin{minipage}[c]{0.66\linewidth}
      \centering
      \begin{overpic}[trim=0 352 0 0,clip,height=3.4cm]{./Figures/TimeSerie_Rrs680.eps} \put (90,28) {\colorbox{white}{(k)}}
      \end{overpic}
    \end{minipage}  
    \hfill
    \begin{minipage}[c]{0.33\linewidth}
      \centering
      \begin{overpic}[trim=0 0 0 0,clip,height=3.2cm]{./Figures/Hist_Rrs680.eps} \put (78,68) {\colorbox{white}{(l)}}
      \end{overpic} 
    \end{minipage} 

    \caption{Time Series and histograms for the area of study. The complete GOCI mission was processed to Level 2 and a filtered mean was calculated for each image over the area of study. The data are color coded by time of the day (red: 0h, green: 1h, blue: 2h, black: 3h, cyan: 4h, magenta: 5h, orange: 6h, and purple: 7h). \label{fig:GOCI_TimeSeries} } 
\end{figure}
%-%-%-%-%-%-%-%-%-%-%=END FIGURE=%-%-%-%-%-%-%-%-%-%-%-%-%
%-%-%-%-%-%-%-%-%-%-%=FIGURE=%-%-%-%-%-%-%-%-%-%-%-%-%
\begin{figure}[H]

    \begin{minipage}[c]{0.66\linewidth}
      \centering
      \begin{overpic}[trim=0 352 0 0,clip,height=3.6cm]{./Figures/TimeSerie_chlor_a.eps} \put (90,28) {\colorbox{white}{(a)}}
      \end{overpic}
    \end{minipage}  
    \hfill
    \begin{minipage}[c]{0.33\linewidth}
      \centering
      \begin{overpic}[trim=0 0 0 0,clip,height=3.2cm]{./Figures/Hist_chlor_a.eps} \put (78,25) {\colorbox{white}{(b)}}
      \end{overpic} 
    \end{minipage}        

    \begin{minipage}[c]{0.66\linewidth}
      \centering
      \begin{overpic}[trim=12 352 0 0,clip,height=3.6cm]{./Figures/TimeSerie_ag_412_mlrc.eps} \put (90,28) {\colorbox{white}{(c)}}
      \end{overpic}
    \end{minipage}  
    \hfill
    \begin{minipage}[c]{0.33\linewidth}
      \centering
      \begin{overpic}[trim=0 0 0 0,clip,height=3.2cm]{./Figures/Hist_ag_412_mlrc.eps} \put (78,25) {\colorbox{white}{(d)}}
      \end{overpic} 
    \end{minipage}    

        \begin{minipage}[c]{0.66\linewidth}
      \centering
      \begin{overpic}[trim=-15 352 0 0,clip,height=3.6cm]{./Figures/TimeSerie_poc.eps} \put (90,28) {\colorbox{white}{(e)}}
      \end{overpic}
    \end{minipage}  
    \hfill
    \begin{minipage}[c]{0.33\linewidth}
      \centering
      \begin{overpic}[trim=0 0 0 0,clip,height=3.2cm]{./Figures/Hist_poc.eps} \put (78,25) {\colorbox{white}{(f)}}
      \end{overpic} 
    \end{minipage}    

    \begin{minipage}[c]{0.66\linewidth}
      \centering
      \begin{overpic}[trim=10 0 0 390,clip,height=3.5cm]{./Figures/TimeSerie_Rrs680.eps} \put (90,25) {\colorbox{white}{(g)}}
      \end{overpic}
    \end{minipage}   

    \caption{Time Series and histograms for the (a,b) Chlorophyll-{\it a}, (c,d) $a_{g:mlrc}(412)$ , (e,f) POC products, and (g) solar zenith angle for the area of study. The data are color coded by time of the day (red: 0h, green: 1h, blue: 2h, black: 3h, cyan: 4h, magenta: 5h, orange: 6h, and purple: 7h). \label{fig:GOCI_TimeSeries2} } 
\end{figure}
% ---------------------------------------------------------------
\subsection{Temporal Homogeneity}
One assumption for this work is that the water over the area of study stays homogeneous over short periods of time, i.e. the water does not change considerably diurnally (hour-to-hour) nor from day-to-day. In order to test this assumption, different studies were performed and described below.

\autoref{fig:3dayseq} shows two sets of 3-day sequences (July 28$^{th}$-30$^{th}$, 2012 and September 9$^{th}$-11$^{th}$, 2015), both used as examples to have an estimation of the diurnal and day-to-day variability for the $R_{rs}(\lambda)$ and the Chlorophyll-{\it a}, $a_{g:mlrc}(412)$, and POC products. The data were color coded by time of the day. The diurnal variability for $R_{rs}(\lambda)$ is larger for the September 9$^{th}$-11$^{th}$, 2015 sequence. Overall, the diurnal variability is larger than the day-to-day variability, when comparing the same time of the day across all days. This fact suggests that the diurnal variability is influenced by uncertainties in the instrument, geometry or algorithm and not by the biogeochemical processes occurring on the water. 
%-%-%-%-%-%-%-%-%-%-%=FIGURE=%-%-%-%-%-%-%-%-%-%-%-%-%
\begin{figure}[H]
% \centering
    \begin{minipage}[c]{0.24\linewidth}
      \centering
      \begin{overpic}[trim=0 0 0 0,clip,height=5cm]{./Figures/3days_20120728.eps}
        \put (20,90) {\colorbox{white}{(a)}}   
      \end{overpic}
    \end{minipage} 
    \hfill
    \begin{minipage}[c]{0.24\linewidth}
      \centering
      \begin{overpic}[trim=0 0 0 0,clip,height=4.5cm]{./Figures/3day_chlor_a_18.eps}
        \put (15,15) {\colorbox{white}{(b)}}   
      \end{overpic}
    \end{minipage} 
    \hfill
    \begin{minipage}[c]{0.24\linewidth}
      \centering
      \begin{overpic}[trim=0 0 0 0,clip,height=4.5cm]{./Figures/3day_ag_412_mlrc_18.eps}
        \put (20,15) {\colorbox{white}{(c)}}   
      \end{overpic}
    \end{minipage}
    \hfill
    \begin{minipage}[c]{0.24\linewidth}
      \centering
      \begin{overpic}[trim=0 0 0 0,clip,height=4.5cm]{./Figures/3day_poc_18.eps}
        \put (15,15) {\colorbox{white}{(d)}}   
      \end{overpic}
    \end{minipage}

    \vspace{0.5cm}

    \begin{minipage}[c]{0.24\linewidth}
      \centering
      \begin{overpic}[trim=0 0 0 0,clip,height=5cm]{./Figures/3days_20150909.eps}
        \put (20,90) {\colorbox{white}{(e)}}   
      \end{overpic}
    \end{minipage}
    \begin{minipage}[c]{0.24\linewidth}
      \centering
      \begin{overpic}[trim=0 0 0 0,clip,height=4.5cm]{./Figures/3day_chlor_a_57.eps}
        \put (15,15) {\colorbox{white}{(f)}}   
      \end{overpic}
    \end{minipage} 
    \hfill
    \begin{minipage}[c]{0.24\linewidth}
      \centering
      \begin{overpic}[trim=0 0 0 0,clip,height=4.5cm]{./Figures/3day_ag_412_mlrc_57.eps}
        \put (20,15) {\colorbox{white}{(g)}}   
      \end{overpic}
    \end{minipage}
     \hfill
    \begin{minipage}[c]{0.24\linewidth}
      \centering
      \begin{overpic}[trim=0 0 0 0,clip,height=4.5cm]{./Figures/3day_poc_57.eps}
        \put (15,15) {\colorbox{white}{(h)}}   
      \end{overpic}
    \end{minipage}

\caption{ Statistics for two cases. GOCI $R_{rs}(\lambda)$ for 3-day sequences for (a-d) July 28-30, 2012 and (e-h) September 9-11, 2015. (a,e) $R_{rs}(\lambda)$, (b,f) Chlorophyll-{\it a}, (c,g) $a_{g:mlrc}(412)$, and (d,h) POC products. Data are color coded by time of the day (red: 0h, green: 1h, blue: 2h, black: 3h, cyan: 4h, magenta: 5h, orange: 6h, and purple: 7h) and by spectral bands (circle: 412 nm, triangle: 443 nm, asterisk: 490 nm and cross: 555 nm).\label{fig:3dayseq} }     

% \caption{ GOCI Chlorophyll-{\it a}, $a_{g:mlrc}(412)$, POC products for 3-day sequences for (a-c) July 28-30, 2012 and (d-f) September 9-11, 2015. Data are color coded by time of the day (red: 0h, green: 1h, blue: 2h, black: 3h, cyan: 4h, magenta: 5h, orange: 6h).\label{fig:3dayseq_par} } 

\end{figure}
%-%-%-%-%-%-%-%-%-%-%=END FIGURE=%-%-%-%-%-%-%-%-%-%-%-%-%
In order to have a quantitative estimation of the homogeneity, statistics were calculated for all 3-day sequences. Out of the 6435 days for the whole GOCI mission, there are only 228 3-day sequences with valid values for the 412, 443, 490 and 555 nm bands and for all times of the days. Note that there is not a single 3-day sequence with valid values for all bands and all times of the days. This shows the difficulty of finding clear sky data in the area of study. 

The mean, standard deviation and coefficient of variation ($CV[\%]=100\times SD/mean$) were calculated for all 3-day sequences . \autoref{fig:3dayseq_stats}.(a) shows the mean of the mean, and the mean of the standard deviation while \autoref{fig:3dayseq_stats}.(b) shows the mean of the coefficient of variation for the 228 3-day sequences for $R_{rs}(\lambda)$. \autoref{fig:3dayseq_stats_par} shows the same parameters for the Chlorophyll-{\it a}, $a_{g:mlrc}(412)$, and POC products. The data were color coded by time of the day. It can be seen that the mean values remain similar for the first six hours for all bands. Also, the day-to-day variability is small, which is reflected in the small standard deviation and a mean of CV less than $7\%$  for all cases but the last two images per day.

From this previous analysis, we can conclude that the water over the area of study is homogeneous in short periods of time. However, it also exhibits seasonality, as expected, which is reflected in the time series of \S\ref{subsec:timeseries}.
%-%-%-%-%-%-%-%-%-%-%=FIGURE=%-%-%-%-%-%-%-%-%-%-%-%-%
\begin{figure}[H]
    \begin{minipage}[c]{0.49\linewidth}
      \centering
      \begin{overpic}[trim=0 0 250 0,clip,height=5cm]{./Figures/3dayseq_mean_SD.eps}
        \put (88,60) {\colorbox{white}{(a)}}   
      \end{overpic}
    \end{minipage} 
    \hfill
    \begin{minipage}[c]{0.49\linewidth}
      \centering
      \begin{overpic}[trim=0 0 0 0,clip,height=5cm]{./Figures/3dayseq_CV.eps}
        \put (15,49) {\colorbox{white}{(b)}}   
      \end{overpic}
    \end{minipage}

\caption{Global statistics for the 3-day sequences for the GOCI $R_{rs}(\lambda)$. (a) mean and standard deviation (SD) and (b) coefficient of variation (CV). Data are color coded by time of the day (red: 0h, green: 1h, blue: 2h, black: 3h, cyan: 4h, magenta: 5h, orange: 6h, and purple: 7h) and by spectral bands (circle: 412 nm, triangle: 443 nm, asterisk: 490 nm and cross: 555 nm).\label{fig:3dayseq_stats} } 
\end{figure}
%-%-%-%-%-%-%-%-%-%-%=END FIGURE=%-%-%-%-%-%-%-%-%-%-%-%-%
%-%-%-%-%-%-%-%-%-%-%=FIGURE=%-%-%-%-%-%-%-%-%-%-%-%-%
\begin{figure}[H]
    \begin{minipage}[c]{0.49\linewidth}
      \centering
      \begin{overpic}[trim=0 0 0 0,clip,height=5cm]{./Figures/3dayseq_mean_SD_chlor_a.eps}
        \put (15,10) {\colorbox{white}{(a)}}   
      \end{overpic}
    \end{minipage} 
    \hfill
    \begin{minipage}[c]{0.49\linewidth}
      \centering
      \begin{overpic}[trim=0 0 0 0,clip,height=5cm]{./Figures/3dayseq_mean_SD_ag_412_mlrc.eps}
        \put (15,10) {\colorbox{white}{(b)}}   
      \end{overpic}
    \end{minipage} 

    \vspace{0.5cm}

    \begin{minipage}[c]{0.49\linewidth}
      \centering
      \begin{overpic}[trim=0 0 0 0,clip,height=5cm]{./Figures/3dayseq_mean_SD_poc.eps}
        \put (15,10) {\colorbox{white}{(c)}}   
      \end{overpic}
    \end{minipage} 
    \hfill
    \begin{minipage}[c]{0.49\linewidth}
      \centering
      \begin{overpic}[trim=0 0 0 0,clip,height=5cm]{./Figures/3dayseq_CV_par.eps}
        \put (15,10) {\colorbox{white}{(d)}}   
      \end{overpic}
    \end{minipage}

\caption{Global statistics for the 3-day sequences. (a-c) mean and standard deviation (SD) and (b) coefficient of variation (CV) for the (a) Chlorophyll-{\it a}, (b) $a_{g:mlrc}(412)$ and (c) POC products for GOCI. Data are color coded by time of the day (red: 0h, green: 1h, blue: 2h, black: 3h, cyan: 4h, magenta: 5h, orange: 6h, and purple: 7h) and by products (circle: Chlorophyll-{\it a}, triangle: $a_{g:mlrc}(412)$, and asterisk: POC).\label{fig:3dayseq_stats_par} } 
\end{figure}
%-%-%-%-%-%-%-%-%-%-%=END FIGURE=%-%-%-%-%-%-%-%-%-%-%-%-%
% ---------------------------------------------------------------
\subsection{Products versus solar zenith angle}
\autoref{fig:Rrs_vs_zenith} shows the $R_{rs}$ filtered mean values for each time of the day for the area of study versus the solar zenith angle (SZA). Only data that passed the exclusion criteria described previously are plotted. The data are color coded by time of the day (red: 0h, green: 1h, blue: 2h, black: 3h, cyan: 4h, magenta: 5h, orange: 6h, and purple: 7h).
%-%-%-%-%-%-%-%-%-%-%=FIGURE=%-%-%-%-%-%-%-%-%-%-%-%-%
\begin{figure}[H]
    \begin{minipage}[c]{0.49\linewidth}
      \centering
      \begin{overpic}[trim=0 0 0 0,clip,height=5cm]{./Figures/Rrs_412_vs_Zenith.eps}
        \put (9,55) {\colorbox{white}{(a)}}   
      \end{overpic}
    \end{minipage}  
    \hfill
    \begin{minipage}[c]{0.49\linewidth}
      \centering
      \begin{overpic}[trim=0 0 0 0,clip,height=5cm]{./Figures/Rrs_443_vs_Zenith.eps}
        \put (16,22) {\colorbox{white}{(b)}}   
      \end{overpic}
    \end{minipage} 

    \vspace{0.5cm}

    \begin{minipage}[c]{0.49\linewidth}
      \centering
      \begin{overpic}[trim=0 0 0 0,clip,height=5cm]{./Figures/Rrs_490_vs_Zenith.eps}
        \put (16,22) {\colorbox{white}{(c)}}   
      \end{overpic} 
    \end{minipage}  
    \hfill
    \begin{minipage}[c]{0.49\linewidth}
      \centering
      \begin{overpic}[trim=0 0 0 0,clip,height=5cm]{./Figures/Rrs_555_vs_Zenith.eps}
        \put (16,22) {\colorbox{white}{(d)}}   
      \end{overpic}
    \end{minipage} 

    \vspace{0.5cm}

    \begin{minipage}[c]{0.49\linewidth}
      \centering
      \begin{overpic}[trim=0 0 0 0,clip,height=5cm]{./Figures/Rrs_660_vs_Zenith.eps}
        \put (16,22) {\colorbox{white}{(e)}}   
      \end{overpic}
    \end{minipage}  
    \hfill
    \begin{minipage}[c]{0.49\linewidth}
      \centering
      \begin{overpic}[trim=0 0 0 0,clip,height=5cm]{./Figures/Rrs_680_vs_Zenith.eps}
        \put (16,22) {\colorbox{white}{(f)}}   
      \end{overpic} 
    \end{minipage}  

    \caption{Filtered mean $R_{rs}(\lambda)$ versus solar zenith angle, color coded by time of the day (red: 0h, green: 1h, blue: 2h, black: 3h, cyan: 4h, magenta: 5h, orange: 6h, and purple: 7h). \label{fig:Rrs_vs_zenith} } 
\end{figure}
%-%-%-%-%-%-%-%-%-%-%=END FIGURE=%-%-%-%-%-%-%-%-%-%-%-%-%

%-%-%-%-%-%-%-%-%-%-%=END FIGURE=%-%-%-%-%-%-%-%-%-%-%-%-%
\autoref{fig:par_vs_zenith} shows the the Chlorophyll-{\it a}, $a_{g:mlrc}(412)$ and POC products versus the solar zenith angle for the area of study. The data are color coded by time of the day. 
%-%-%-%-%-%-%-%-%-%-%=FIGURE=%-%-%-%-%-%-%-%-%-%-%-%-%
\begin{figure}[H]
 \begin{minipage}[c]{0.49\linewidth}
      \centering
      \begin{overpic}[trim=0 0 0 0,clip,height=5.0cm]{./Figures/Par_vs_Zenith_chlor_a.eps}
        \put (9,55) {\colorbox{white}{(a)}}   
      \end{overpic}
    \end{minipage}  
    \hfill
    \begin{minipage}[c]{0.49\linewidth}
      \centering
      \begin{overpic}[trim=0 0 0 0,clip,height=5.0cm]{./Figures/Par_vs_Zenith_ag_412_mlrc.eps}
        \put (16,22) {\colorbox{white}{(b)}}   
      \end{overpic}
    \end{minipage} 

    \vspace{0.3cm}

    \begin{minipage}[c]{1.0\linewidth}
      \centering
      \begin{overpic}[trim=0 0 0 0,clip,height=5.0cm]{./Figures/Par_vs_Zenith_poc.eps}
        \put (16,22) {\colorbox{white}{(c)}}   
      \end{overpic} 
    \end{minipage}  

    \caption{(a) Chlor-{\it a}, (b) $a_{g:mlrc}(412)$ and (c) POC versus solar zenith angle, color coded by time of the day (red: 0h, green: 1h, blue: 2h, black: 3h, cyan: 4h, magenta: 5h, orange: 6h, and purple: 7h). \label{fig:par_vs_zenith} } 
\end{figure}
%-%-%-%-%-%-%-%-%-%-%=END FIGURE=%-%-%-%-%-%-%-%-%-%-%-%-%
Then, the anomalies were calculated in order to remove seasonality. Monthly-climatology by time of day removed seasonality by subtracting the mission monthly hourly mean. This is, a mean value was calculated for each time of day (8 times) for every month (12 months). For example, all the 0h time of day for all the Januaries were averaged. Then, this time of the day monthly average was subtracted from every 0h acquired in all Januaries. These anomalies represent de-seasoned data, i.e. independent of the seasonality. The anomalies of $R_{rs}(\lambda)$ versus the solar zenith angle are shown in \autoref{fig:Rrs_vs_zenith_detrend} color coded by time of the day.

%-%-%-%-%-%-%-%-%-%-%=FIGURE=%-%-%-%-%-%-%-%-%-%-%-%-%
\begin{figure}[H]
    \begin{minipage}[c]{0.49\linewidth}
      \centering
      \begin{overpic}[trim=0 0 0 00,clip,height=5.0cm]{./Figures/Rrs_vs_Zenith_detrend_412_2.eps}
        \put (16,22) {\colorbox{white}{(a)}}   
      \end{overpic}
    \end{minipage}  
    \hfill
    \begin{minipage}[c]{0.49\linewidth}
      \centering
      \begin{overpic}[trim=0 0 0 00,clip,height=5.0cm]{./Figures/Rrs_vs_Zenith_detrend_443_2.eps}
        \put (16,22) {\colorbox{white}{(b)}}   
      \end{overpic}
    \end{minipage}  
      
    \vspace{0.5cm}

    \begin{minipage}[c]{0.49\linewidth}
      \centering
      \begin{overpic}[trim=0 0 0 00,clip,height=5.0cm]{./Figures/Rrs_vs_Zenith_detrend_490_2.eps}
        \put (16,22) {\colorbox{white}{(c)}}   
      \end{overpic} 
    \end{minipage}  
    \hfill
    \begin{minipage}[c]{0.49\linewidth}
      \centering
      \begin{overpic}[trim=0 0 0 00,clip,height=5.0cm]{./Figures/Rrs_vs_Zenith_detrend_555_2.eps}
        \put (16,22) {\colorbox{white}{(d)}}   
      \end{overpic}
    \end{minipage}  

    \vspace{0.5cm}

    \begin{minipage}[c]{0.49\linewidth}
      \centering
      \begin{overpic}[trim=0 0 0 00,clip,height=5.0cm]{./Figures/Rrs_vs_Zenith_detrend_660_2.eps}
        \put (16,22) {\colorbox{white}{(e)}}   
      \end{overpic}
    \end{minipage}  
    \hfill
    \begin{minipage}[c]{0.49\linewidth}
      \centering
      \begin{overpic}[trim=0 0 0 00,clip,height=5.0cm]{./Figures/Rrs_vs_Zenith_detrend_680_2.eps}
        \put (16,22) {\colorbox{white}{(f)}}   
      \end{overpic} 
    \end{minipage}  

    \caption{Anomalies of $R_{rs}(\lambda)$ versus solar zenith angle, color coded by time of the day (red: 0h, green: 1h, blue: 2h, black: 3h, cyan: 4h, magenta: 5h, orange: 6h, and purple: 7h). \label{fig:Rrs_vs_zenith_detrend} } 
\end{figure}
%-%-%-%-%-%-%-%-%-%-%=END FIGURE=%-%-%-%-%-%-%-%-%-%-%-%-%
\autoref{fig:par_vs_zenith_detrend} shows the anomalies of Chlor-{\it a}, $a_{g:mlrc}(412)$ and POC, color coded by time of day.
%-%-%-%-%-%-%-%-%-%-%=FIGURE=%-%-%-%-%-%-%-%-%-%-%-%-%
\begin{figure}[H]
    \begin{minipage}[c]{0.49\linewidth}
      \centering
      \begin{overpic}[trim=0 0 0 00,clip,height=5.0cm]{./Figures/par_vs_Zenith_detrend_chlor_a_2.eps}
        \put (16,22) {\colorbox{white}{(a)}}   
      \end{overpic}
    \end{minipage}  
    \hfill
    \begin{minipage}[c]{0.49\linewidth}
      \centering
      \begin{overpic}[trim=0 0 0 00,clip,height=5.0cm]{./Figures/par_vs_Zenith_detrend_ag_412_mlrc_2.eps}
        \put (16,22) {\colorbox{white}{(b)}}   
      \end{overpic}
    \end{minipage}  

    \vspace{0.5cm}
 
    \begin{minipage}[c]{1.0\linewidth}
      \centering
      \begin{overpic}[trim=0 0 0 00,clip,height=5.0cm]{./Figures/par_vs_Zenith_detrend_poc_2.eps}
        \put (16,22) {\colorbox{white}{(c)}}   
      \end{overpic} 
    \end{minipage}  

    \caption{Anomalies of (a) Chlor-{\it a}, (b) $a_{g:mlrc}(412)$ and (c) POC versus solar zenith angle, color coded by time of the day (red: 0h, green: 1h, blue: 2h, black: 3h, cyan: 4h, magenta: 5h, orange: 6h, and purple: 7h). \label{fig:par_vs_zenith_detrend} } 
\end{figure}
%-%-%-%-%-%-%-%-%-%-%=END FIGURE=%-%-%-%-%-%-%-%-%-%-%-%-%

% ---------------------------------------------------------------
\subsection{Diurnal Differences and Uncertainties}
% Daily standard deviation
A standard deviation (SD) of the eight values per day was calculated. This daily standard deviation is an indicator of the temporal stability of the selected homogeneous ocean region. The hope is to have minimal variation during the day over this region. Then, all of the daily standard deviation were averaged to have a mean value per band. These values are compared to the results obtained from the comparison the AERONET-OC dataset (\autoref{tab:stdev_aero}).

\begin{table}[htbp!]
\caption{Standard deviation (SD) of the time series compared with the root mean squared error (RMSE). \label{tab:stdev_aero} } 
\small
\centering
\begin{tabular}{ccc} \hline

 \bfseries{Product Name} & \bfseries{SD} & \bfseries{RMSE}\\
 & (Time Series)) & (AERONET-OC) \\ 
 & $[1/sr]$ & $[1/sr]$ \\ \hline \hline
$R_{rs}(412)$ & $6.84\times10^4$ & $1.8\times10^3$\\ 
$R_{rs}(443)$ & $4.78\times10^4$ & $1.7\times10^3$\\ 
$R_{rs}(490)$ & $4.00\times10^4$ & $1.8\times10^3$\\ 
$R_{rs}(555)$ & $1.73\times10^4$ & $2.4\times10^3$\\ 
$R_{rs}(660)$ & $2.88\times10^5$ & $7.0\times10^4$\\ 
$R_{rs}(680)$ & $1.94\times10^5$ & N/A \\ \hline
 \end{tabular}
\end{table}



% Relative difference
For each day, the mean of the three midday images (02, 03 and 04h) was calculated as

\begin{equation}
	\hat{X}_{02,03,04h} = \frac{1}{N} \sum_{t}^N x_t,~~~t=02h,03h~\text{and}~04h
\end{equation}

The difference con respect to a reference is $\Delta_t=x_t-x_{reference}$. Then, the relative difference is defined as
\begin{equation}
	RD_t = \frac{\Delta_t}{|x_{reference}|} = \frac{x_t-x_{reference}}{|x_{reference}|}
	\times 100[\%]
\end{equation}
where $x_t$ is the satellite data at the local time $t=09h,10h\dots16h$ of the day and for this case $x_{reference}=\hat{X}_{02,03,04h}$. In order to exclude outliers, only relative differences that were within the mean plus 3 times the standard deviation were included in the analysis.

\autoref{fig:DiffMidThreeMean} shows the mean and the standard deviation of the relative difference with respect to the mean of the three midday images (02, 03 and 04h) for the whole mission for (a-f) $R_{rs}(\lambda)$ and (g) Chlorophyll-{\it a}, (h) POC and (i) $a_{g:mlrc}(412)$ for each time of the day. \autoref{fig:DiffMidThreeMean_detrend} shows the same parameters but for the anomalies. 

\todo{update table!}
\begin{table}[htbp!]
\caption{Relative difference with respect to the daily mean.\label{tab:rel_diff} } 
\scriptsize
\centering
\begin{tabular}{cccccccccc} \hline
      &   \multicolumn{8}{c}{Local Time}   \\ \cline{2-9}
   &    9h    & 10h  &  11h  &  12h  &  13h  &  14h  &  15h   &  16h   \\ 
$\lambda$     &    Mean(SD)    & Mean(SD)  &  Mean(SD)  &  Mean(SD)  &  Mean(SD)  &  Mean(SD)  &  Mean(SD)   &  Mean(SD)   \\ 
(nm)    &    [\%]([\%])    & [\%]([\%])  &  [\%]([\%])  &  [\%]([\%])  &  [\%]([\%])  &  [\%]([\%])  &  [\%]([\%])   &  [\%]([\%])   \\ \hline \hline
412 &  -1.1 (4.7)  & 2.9 (3.8)  &  4.4 (5.2)  &  4.9 (5.9)  &  3.1 (4.2) & -1.1 (7.8) & -10.3 (12.7) & -13.0 (9.5) \\
443 &  -0.9 (4.2) & 2.9 (3.5) &  3.8 (4.3) &  4.1 (5.0) &  2.6 (3.5) & -0.6 (6.9) &  -9.5 (11.5) & -11.8 (8.7)  \\
490 &   0.6 (4.5) & 4.6 (4.2) &  5.0 (4.4) &  4.9 (5.0) &  2.6 (3.8) & -1.2 (6.0) & -12.5 (14.3) & -17.8 (11.3) \\
555 &  -1.5 (7.9) & 7.5 (6.8) &  9.1 (7.4) &  8.8 (8.3) &  4.4 (6.3) & -2.4 (10.1) & -20.7 (18.0) & -27.4 (17.7) \\
660 & -11.2 (24.2) & 14.5 (27.7) &  15.4 (25.7) &  16.4 (31.3) &   0.6 (25.6) & -10.2 (28.6) &  -46.4 (28.9) &  -62.5 (26.8) \\
680 &  29.7 (47.6) & 28.7 (47.8) & -13.3 (38.1) & -26.9 (37.1) & -24.8 (33.9) & -13.1 (42.0) &  -19.8 (47.8) & -6.4 ( 67.5) \\ \hline
 \end{tabular}
\end{table}
%-%-%-%-%-%-%-%-%-%-%=FIGURE=%-%-%-%-%-%-%-%-%-%-%-%-%
\begin{figure}[H]
    \begin{minipage}[c]{0.32\linewidth}
      \centering
      \hspace{1.5cm}
      \begin{overpic}[trim=0 0 0 0,clip,height=4.0cm]{./Figures/Rel_Diff_mid_three_Rrs412.eps}
        \put (25,15) {\colorbox{white}{(a)}}
      \end{overpic}
    \end{minipage}  
    \hfill
    \begin{minipage}[c]{0.32\linewidth}
      \centering
      \begin{overpic}[trim=0 0 0 0,clip,height=4.0cm]{./Figures/Rel_Diff_mid_three_Rrs443.eps}
        \put (25,15) {\colorbox{white}{(b)}}
      \end{overpic}
    \end{minipage}  
    \hfill
    \begin{minipage}[c]{0.32\linewidth}
      \centering
      \hspace{1.5cm}
      \begin{overpic}[trim=0 0 0 0,clip,height=4.0cm]{./Figures/Rel_Diff_mid_three_Rrs490.eps}
        \put (25,15) {\colorbox{white}{(c)}}
      \end{overpic}
    \end{minipage}  
    
    \vspace{0.5cm}

    \begin{minipage}[c]{0.32\linewidth}
      \centering
      \begin{overpic}[trim=0 0 0 0,clip,height=4.0cm]{./Figures/Rel_Diff_mid_three_Rrs555.eps}
        \put (25,15) {\colorbox{white}{(d)}}
      \end{overpic}
    \end{minipage}  
    \hfill
    \begin{minipage}[c]{0.32\linewidth}
      \centering
      \hspace{1.5cm}
      \begin{overpic}[trim=0 0 0 0,clip,height=4.0cm]{./Figures/Rel_Diff_mid_three_Rrs660.eps}
        \put (28,15) {\colorbox{white}{(e)}}
      \end{overpic}
    \end{minipage}   
    \hfill
    \begin{minipage}[c]{0.32\linewidth}
      \centering
      \begin{overpic}[trim=0 0 0 0,clip,height=4.0cm]{./Figures/Rel_Diff_mid_three_Rrs680.eps}
        \put (28,15) {\colorbox{white}{(f)}}
      \end{overpic}
    \end{minipage} 

    \vspace{0.5cm}

    \begin{minipage}[c]{0.32\linewidth}
      \centering
      \begin{overpic}[trim=0 0 0 0,clip,height=4.0cm]{./Figures/Rel_Diff_mid_three_chlor_a.eps}
        \put (28,15) {\colorbox{white}{(g)}}
      \end{overpic}
    \end{minipage}  
    \hfill
    \begin{minipage}[c]{0.32\linewidth}
      \centering
      \begin{overpic}[trim=0 0 0 0,clip,height=4.0cm]{./Figures/Rel_Diff_mid_three_poc.eps}
        \put (28,15) {\colorbox{white}{(h)}}
      \end{overpic}
    \end{minipage}  
    \hfill
  	\begin{minipage}[c]{0.32\linewidth}
      \centering
      \begin{overpic}[trim=0 0 0 0,clip,height=4.0cm]{./Figures/Rel_Diff_mid_three_ag_412_mlrc.eps}
        \put (30,15) {\colorbox{white}{(i)}}
      \end{overpic}
    \end{minipage}  

    \caption{Mean of the relative difference with respect to the mean of the three midday images for (a-f) $R_{rs}(\lambda)$ and (g) Chlorophyll-{\it a}, (h) POC and (i) $a_{g:mlrc}(412)$. Error bars represent the standard deviation. \label{fig:DiffMidThreeMean} } 
\end{figure}
%-%-%-%-%-%-%-%-%-%-%=END FIGURE=%-%-%-%-%-%-%-%-%-%-%-%-%
%-%-%-%-%-%-%-%-%-%-%=FIGURE=%-%-%-%-%-%-%-%-%-%-%-%-%
\begin{figure}[H]
    \begin{minipage}[c]{0.32\linewidth}
      \centering
      \hspace{1.5cm}
      \begin{overpic}[trim=0 0 0 0,clip,height=4.0cm]{./Figures/Rel_Diff_mid_three_Rrs_412_detrend.eps}
        \put (25,15) {\colorbox{white}{(a)}}
      \end{overpic}
    \end{minipage}  
    \hfill
    \begin{minipage}[c]{0.32\linewidth}
      \centering
      \begin{overpic}[trim=0 0 0 0,clip,height=4.0cm]{./Figures/Rel_Diff_mid_three_Rrs_443_detrend.eps}
        \put (25,15) {\colorbox{white}{(b)}}
      \end{overpic}
    \end{minipage}  
    \hfill
    \begin{minipage}[c]{0.32\linewidth}
      \centering
      \hspace{1.5cm}
      \begin{overpic}[trim=0 0 0 0,clip,height=4.0cm]{./Figures/Rel_Diff_mid_three_Rrs_490_detrend.eps}
        \put (25,15) {\colorbox{white}{(c)}}
      \end{overpic}
    \end{minipage}  
    
    \vspace{0.5cm}

    \begin{minipage}[c]{0.32\linewidth}
      \centering
      \begin{overpic}[trim=0 0 0 0,clip,height=4.0cm]{./Figures/Rel_Diff_mid_three_Rrs_555_detrend.eps}
        \put (25,15) {\colorbox{white}{(d)}}
      \end{overpic}
    \end{minipage}  
    \hfill
    \begin{minipage}[c]{0.32\linewidth}
      \centering
      \hspace{1.5cm}
      \begin{overpic}[trim=0 0 0 0,clip,height=4.0cm]{./Figures/Rel_Diff_mid_three_Rrs_660_detrend.eps}
        \put (28,15) {\colorbox{white}{(e)}}
      \end{overpic}
    \end{minipage}   
    \hfill
    \begin{minipage}[c]{0.32\linewidth}
      \centering
      \begin{overpic}[trim=0 0 0 0,clip,height=4.0cm]{./Figures/Rel_Diff_mid_three_Rrs_680_detrend.eps}
        \put (28,15) {\colorbox{white}{(f)}}
      \end{overpic}
    \end{minipage} 

    \vspace{0.5cm}

    \begin{minipage}[c]{0.32\linewidth}
      \centering
      \begin{overpic}[trim=0 0 0 0,clip,height=4.0cm]{./Figures/Rel_Diff_mid_three_chlor_a_detrend.eps}
        \put (28,15) {\colorbox{white}{(g)}}
      \end{overpic}
    \end{minipage}  
    \hfill
    \begin{minipage}[c]{0.32\linewidth}
      \centering
      \begin{overpic}[trim=0 0 0 0,clip,height=4.0cm]{./Figures/Rel_Diff_mid_three_poc_detrend.eps}
        \put (28,15) {\colorbox{white}{(h)}}
      \end{overpic}
    \end{minipage}  
    \hfill
    \begin{minipage}[c]{0.32\linewidth}
      \centering
      \begin{overpic}[trim=0 0 0 0,clip,height=4.0cm]{./Figures/Rel_Diff_mid_three_ag_412_mlrc_detrend.eps}
        \put (30,15) {\colorbox{white}{(i)}}
      \end{overpic}
    \end{minipage}  

    \caption{Mean of the relative difference with respect to the mean of the three midday images for (a-f) $R_{rs}(\lambda)$ and (g) Chlorophyll-{\it a}, (h) POC and (i) $a_{g:mlrc}(412)$ for the anomalies. Error bars represent the standard deviation. \label{fig:DiffMidThreeMean_detrend} } 
\end{figure}
%-%-%-%-%-%-%-%-%-%-%=END FIGURE=%-%-%-%-%-%-%-%-%-%-%-%-%
% ---------------------------------------------------------------
\subsection{Sensor Cross-comparison}

% %-%-%-%-%-%-%-%-%-%-%=FIGURE=%-%-%-%-%-%-%-%-%-%-%-%-%
% \begin{figure}[H]
%     \begin{minipage}[c]{1.0\linewidth}
%       \centering
%       \begin{overpic}[trim=0 0 0 0,clip,height=3.2cm]{./Figures/CrossComp_Rrs412.eps} \put (10,28) {\colorbox{white}{(a)}}
%       \end{overpic}
%     \end{minipage}   
    
%     \begin{minipage}[c]{1.0\linewidth}
%       \centering
%       \begin{overpic}[trim=0 0 0 0,clip,height=3.4cm]{./Figures/CrossComp_Rrs443.eps} \put (10,28) {\colorbox{white}{(b)}}
%       \end{overpic}
%     \end{minipage}   

%     \begin{minipage}[c]{1.0\linewidth}
%       \centering
%       \begin{overpic}[trim=0 0 0 0,clip,height=3.4cm]{./Figures/CrossComp_Rrs490.eps} \put (10,28) {\colorbox{white}{(c)}}
%       \end{overpic}
%     \end{minipage}  
    
%     \begin{minipage}[c]{1.0\linewidth}
%       \centering
%       \begin{overpic}[trim=0 0 0 0,clip,height=3.4cm]{./Figures/CrossComp_Rrs555.eps} \put (10,28) {\colorbox{white}{(d)}}
%       \end{overpic}
%     \end{minipage}   

%     \begin{minipage}[c]{1.0\linewidth}
%       \centering
%       \begin{overpic}[trim=0 0 0 0,clip,height=3.4cm]{./Figures/CrossComp_Rrs660.eps} \put (10,28) {\colorbox{white}{(e)}}
%       \end{overpic}
%     \end{minipage}  
    
%     \begin{minipage}[c]{1.0\linewidth}
%       \centering
%       \begin{overpic}[trim=0 0 0 0,clip,height=3.4cm]{./Figures/CrossComp_Rrs680.eps} \put (10,28) {\colorbox{white}{(f)}}
%       \end{overpic}
%     \end{minipage}   

%     \caption{Cross-comparison. \label{fig:CrossComp} } 
% \end{figure}

%-%-%-%-%-%-%-%-%-%-%=FIGURE=%-%-%-%-%-%-%-%-%-%-%-%-%
\begin{figure}[H]
    \begin{minipage}[c]{0.49\linewidth}
      \centering
      \begin{overpic}[trim=0 0 0 0,clip,height=5.5cm]{./Figures/CrossComp_All_Rrs.eps} \put (13,48.5) {\colorbox{white}{(a)}}
      \end{overpic}
    \end{minipage}   
    \hfill 
    \begin{minipage}[c]{0.49\linewidth}
      \centering
      \begin{overpic}[trim=0 0 0 0,clip,height=4.5cm]{./Figures/Ratio_GOCI_MODISA.eps} \put (15,69) {\colorbox{white}{(b)}}
      \end{overpic}
    \end{minipage}  

    \vspace{.3cm}

    \begin{minipage}[c]{0.49\linewidth}
      \centering
      \begin{overpic}[trim=0 0 0 0,clip,height=4.5cm]{./Figures/Ratio_GOCI_VIIRS.eps} \put (15,69) {\colorbox{white}{(c)}}
      \end{overpic}
    \end{minipage} 
    \hfill      
    \begin{minipage}[c]{0.49\linewidth}
      \centering
      \begin{overpic}[trim=0 0 0 0,clip,height=4.5cm]{./Figures/Ratio_MODISA_VIIRS.eps} \put (15,69) {\colorbox{white}{(d)}}
      \end{overpic}
    \end{minipage} 

    \caption{Cross-comparison with MODIS and VIIRS for all wavelengths. (a) Rrs, (b) GOCI/MODIS Aqua ratio, (c) GOCI/VIIRS ratio, and (d) MODIS Aqua/VIIRS ratio. \label{fig:CrossCompAllRrs} } 
\end{figure}
%-%-%-%-%-%-%-%-%-%-%=END FIGURE=%-%-%-%-%-%-%-%-%-%-%-%-%

%-%-%-%-%-%-%-%-%-%-%=FIGURE=%-%-%-%-%-%-%-%-%-%-%-%-%
\begin{figure}[H]
    \begin{minipage}[c]{1.0\linewidth}
      \centering
      \begin{overpic}[trim=0 0 0 0,clip,height=3.5cm]{./Figures/TimeSerieComp_chlor_a.eps} \put (9,28) {\colorbox{white}{(a)}}
      \end{overpic}
    \end{minipage}   

    \vspace{0.3cm}
    
    \begin{minipage}[c]{1.0\linewidth}
      \centering
      \hspace{-0.3cm}
      \begin{overpic}[trim=0 0 0 0,clip,height=3.65cm]{./Figures/TimeSerieComp_ag_412_mlrc.eps} \put (11,28) {\colorbox{white}{(b)}}
      \end{overpic}
    \end{minipage}

    \vspace{0.3cm}       

    \begin{minipage}[c]{1.0\linewidth}
      \centering
      \hspace{0.2cm}
      \begin{overpic}[trim=0 0 0 0,clip,height=3.65cm]{./Figures/TimeSerieComp_poc.eps} \put (7,30) {\colorbox{white}{(c)}}
      \end{overpic}
    \end{minipage} 

    \caption{Time Series comparison with MODISA and VIIRS for chlor-{\it a}, $a_{g:mlrc}(412)$ and POC. \label{fig:GOCI_TimeSeriesComp_par} } 
\end{figure}
%-%-%-%-%-%-%-%-%-%-%=END FIGURE=%-%-%-%-%-%-%-%-%-%-%-%-%

% %-%-%-%-%-%-%-%-%-%-%=FIGURE=%-%-%-%-%-%-%-%-%-%-%-%-%
% \begin{figure}[H]
%     \begin{minipage}[c]{1.0\linewidth}
%       \centering
%       \begin{overpic}[trim=0 0 0 0,clip,height=3.5cm]{./Figures/TimeSerie_Angstrom.eps} \put (9,30) {\colorbox{white}{(d)}}
%       \end{overpic}
%     \end{minipage}   
    
%     \begin{minipage}[c]{1.0\linewidth}
%       \centering
%       \begin{overpic}[trim=0 0 0 0,clip,height=3.5cm]{./Figures/TimeSerie_AOT_865.eps} \put (9,30) {\colorbox{white}{(e)}}
%       \end{overpic}
%     \end{minipage}       

%     \begin{minipage}[c]{1.0\linewidth}
%       \centering
%       \begin{overpic}[trim=0 0 0 0,clip,height=3.5cm]{./Figures/TimeSerie_brdf.eps} \put (9,30) {\colorbox{white}{(f)}}
%       \end{overpic}
%     \end{minipage} 

%     \caption{Time Series for derived geophysical paremeters (Angstrom and AOT(865)) and atmospheric correction intermediate (BRDF). \label{fig:GOCI_TimeSeries_intermed_par} } 
% \end{figure}
% %-%-%-%-%-%-%-%-%-%-%=END FIGURE=%-%-%-%-%-%-%-%-%-%-%-%-%

%-%-%-%-%-%-%-%-%-%-%=FIGURE=%-%-%-%-%-%-%-%-%-%-%-%-%
\begin{figure}[H]
    \begin{minipage}[c]{0.33\linewidth}
      \centering
      \begin{overpic}[trim=0 0 0 0,clip,height=3.5cm]{./Figures/Scatter_GOCI_AQUA_412.eps} \put (77,30) {\colorbox{white}{(a)}}
      \end{overpic}
    \end{minipage}   
    \begin{minipage}[c]{0.33\linewidth}
      \centering
      \begin{overpic}[trim=0 0 0 0,clip,height=3.5cm]{./Figures/Scatter_GOCI_VIIRS_412.eps} \put (77,30) {\colorbox{white}{(b)}}
      \end{overpic}
    \end{minipage}       
    \begin{minipage}[c]{0.33\linewidth}
      \centering
      \begin{overpic}[trim=0 0 0 0,clip,height=3.5cm]{./Figures/Scatter_VIIRS_AQUA_412.eps} \put (77,30) {\colorbox{white}{(c)}}
      \end{overpic}
    \end{minipage} 

    \begin{minipage}[c]{0.33\linewidth}
      \centering
      \begin{overpic}[trim=0 0 0 0,clip,height=3.5cm]{./Figures/Scatter_GOCI_AQUA_443.eps} \put (77,30) {\colorbox{white}{(d)}}
      \end{overpic}
    \end{minipage}   
    \begin{minipage}[c]{0.33\linewidth}
      \centering
      \begin{overpic}[trim=0 0 0 0,clip,height=3.5cm]{./Figures/Scatter_GOCI_VIIRS_443.eps} \put (77,30) {\colorbox{white}{(e)}}
      \end{overpic}
    \end{minipage}       
    \begin{minipage}[c]{0.33\linewidth}
      \centering
      \begin{overpic}[trim=0 0 0 0,clip,height=3.5cm]{./Figures/Scatter_VIIRS_AQUA_443.eps} \put (77,30) {\colorbox{white}{(f)}}
      \end{overpic}
    \end{minipage} 

    \begin{minipage}[c]{0.33\linewidth}
      \centering
      \begin{overpic}[trim=0 0 0 0,clip,height=3.5cm]{./Figures/Scatter_GOCI_AQUA_490.eps} \put (77,30) {\colorbox{white}{(g)}}
      \end{overpic}
    \end{minipage}   
    \begin{minipage}[c]{0.33\linewidth}
      \centering
      \hspace{.4cm}
      \begin{overpic}[trim=0 0 0 0,clip,height=3.5cm]{./Figures/Scatter_GOCI_VIIRS_490.eps} \put (65,35) {\colorbox{white}{(h)}}
      \end{overpic}
    \end{minipage}       
    \begin{minipage}[c]{0.33\linewidth}
      \centering
      \begin{overpic}[trim=0 0 0 0,clip,height=3.5cm]{./Figures/Scatter_VIIRS_AQUA_490.eps} \put (77,30) {\colorbox{white}{(i)}}
      \end{overpic}
    \end{minipage} 

    \begin{minipage}[c]{0.33\linewidth}
      \centering
      \hspace{.4cm}
      \begin{overpic}[trim=0 0 0 0,clip,height=3.5cm]{./Figures/Scatter_GOCI_AQUA_555.eps} \put (23,17) {\colorbox{white}{(j)}}
      \end{overpic}
    \end{minipage}   
    \begin{minipage}[c]{0.33\linewidth}
      \centering
      \hspace{.4cm}
      \begin{overpic}[trim=0 0 0 0,clip,height=3.5cm]{./Figures/Scatter_GOCI_VIIRS_555.eps} \put (65,30) {\colorbox{white}{(k)}}
      \end{overpic}
    \end{minipage}       
    \begin{minipage}[c]{0.33\linewidth}
      \centering
      \begin{overpic}[trim=0 0 0 0,clip,height=3.5cm]{./Figures/Scatter_VIIRS_AQUA_555.eps} \put (77,30) {\colorbox{white}{(l)}}
      \end{overpic}
    \end{minipage} 

    \begin{minipage}[c]{0.33\linewidth}
      \centering
      \hspace{.4cm}
      \begin{overpic}[trim=0 0 0 0,clip,height=3.5cm]{./Figures/Scatter_GOCI_AQUA_660.eps} \put (64,57) {\colorbox{white}{(m)}}
      \end{overpic}
    \end{minipage}   
    \begin{minipage}[c]{0.33\linewidth}
      \centering
      \hspace{.4cm}
      \begin{overpic}[trim=0 0 0 0,clip,height=3.5cm]{./Figures/Scatter_GOCI_VIIRS_660.eps} \put (65,32) {\colorbox{white}{(n)}}
      \end{overpic}
    \end{minipage}       
    \begin{minipage}[c]{0.33\linewidth}
      \centering
      \begin{overpic}[trim=0 0 0 0,clip,height=3.5cm]{./Figures/Scatter_VIIRS_AQUA_660.eps} \put (77,30) {\colorbox{white}{(o)}}
      \end{overpic}
    \end{minipage} 

    \begin{minipage}[c]{0.33\linewidth}
      \centering
      \hspace{.4cm}
      \begin{overpic}[trim=0 0 0 0,clip,height=3.5cm]{./Figures/Scatter_GOCI_AQUA_680.eps} \put (65,32) {\colorbox{white}{(p)}}
      \end{overpic}
    \end{minipage}   
    \begin{minipage}[c]{0.33\linewidth}
      \centering
      \hspace{.4cm}
      \begin{overpic}[trim=0 0 0 0,clip,height=3.5cm]{./Figures/Scatter_GOCI_VIIRS_680.eps} \put (65,32) {\colorbox{white}{(q)}}
      \end{overpic}
    \end{minipage}       
    \begin{minipage}[c]{0.33\linewidth}
      \centering
      \begin{overpic}[trim=0 0 0 0,clip,height=3.5cm]{./Figures/Scatter_VIIRS_AQUA_680.eps} \put (77,30) {\colorbox{white}{(r)}}
      \end{overpic}
    \end{minipage} 

    \caption{Scatter plots for the $R_{rs}(\lambda)$ cross-comparison for GOCI, MODIS Aqua and VIIRS. \label{fig:scatterRrs} } 
\end{figure}
%-%-%-%-%-%-%-%-%-%-%=END FIGURE=%-%-%-%-%-%-%-%-%-%-%-%-%
% ---------------------------------------------------------------
\subsection{Plane-parallel versus Pseudo-Spherical geometry}

% ---------------------------------------------------------------
\subsection{BRDF correction sensitivity analysis}
%%%%%%%%%%%%%%%%%%% SECTION %%%%%%%%%%%%%%%%%%%%%%%%%%%%%%%%
\section{Discussion} 
The boxed area is covered by two slots, visually noticeable from the image. This can introduced uncertainties in the process.

%%%%%%%%%%%%%%%%%%% SECTION %%%%%%%%%%%%%%%%%%%%%%%%%%%%%%%%
\section{Conclusions}
% Practical applications  
% Disadvantages and Advantages
% Limitations
% Challenges

% from OO poster
The validation with {\it in situ} data exhibit results comparable to heritage sensors.

Expected seasonality and trends were observed through the complete mission.

The atmospheric correction starts to fail for solar zenith angles larger than 60 degrees producing invalid values (negative). 

Uncertainties vary spectrally, being larger in the blue and decreasing towards the red.

Uncertainty patterns are similar for the six first images of the day, increasing for the last two images across all wavelengths

% future work
Estimation of changes due to diurnal and day-to-day biogeochemical stocks and processes in coastal oceans.


%%%%%%%%%%%%%%%%%%% SECTION %%%%%%%%%%%%%%%%%%%%%%%%%%%%%%%%
% \vspace{-.4cm}
\section*{Acknowledgments}
\vspace{-.2cm}
We want to acknowledge the NASA Project ROSES Earth Science U.S. Participating Investigator (NNH12ZDA001N-ESUSPI), the Korea Ocean Satellite Center for providing the GOCI L1B data to OBPG, and the Ocean Biology Processing Group at the Goddard Space Flight Center, NASA. Also, the principal investigator for the AERONET-OC data: Jae-Seol Shim and Joo-Hyung Ryu (Gageocho station), Young-Je Park and Hak-Yeol You (Ieodo station).

%%%%%%%%%%%%%%%%%%% SECTION %%%%%%%%%%%%%%%%%%%%%%%%%%%%%%%%
% \vspace{-.4cm}
\section*{Appendix}\label{sec:appendix}
\autoref{fig:Rrs_vs_zenith_season} shows the same data but color coded by seasons (red: spring, green: summer, fall: blue, and winter: black).
%-%-%-%-%-%-%-%-%-%-%=FIGURE=%-%-%-%-%-%-%-%-%-%-%-%-%
\begin{figure}[H]
    \begin{minipage}[c]{0.49\linewidth}
      \centering
      \begin{overpic}[trim=0 0 0 0,clip,height=5cm]{./Figures/Rrs_412_vs_Zenith_season.eps}
        \put (9,55) {\colorbox{white}{(a)}}   
      \end{overpic}
    \end{minipage}  
    \hfill
    \begin{minipage}[c]{0.49\linewidth}
      \centering
      \begin{overpic}[trim=0 0 0 0,clip,height=5cm]{./Figures/Rrs_443_vs_Zenith_season.eps}
        \put (16,22) {\colorbox{white}{(b)}}   
      \end{overpic}
    \end{minipage} 

    \vspace{0.5cm}

    \begin{minipage}[c]{0.49\linewidth}
      \centering
      \begin{overpic}[trim=0 0 0 0,clip,height=5cm]{./Figures/Rrs_490_vs_Zenith_season.eps}
        \put (16,22) {\colorbox{white}{(c)}}   
      \end{overpic} 
    \end{minipage}  
    \hfill
    \begin{minipage}[c]{0.49\linewidth}
      \centering
      \begin{overpic}[trim=0 0 0 0,clip,height=5cm]{./Figures/Rrs_555_vs_Zenith_season.eps}
        \put (16,22) {\colorbox{white}{(d)}}   
      \end{overpic}
    \end{minipage} 

    \vspace{0.5cm}

    \begin{minipage}[c]{0.49\linewidth}
      \centering
      \begin{overpic}[trim=0 0 0 0,clip,height=5cm]{./Figures/Rrs_660_vs_Zenith_season.eps}
        \put (16,22) {\colorbox{white}{(e)}}   
      \end{overpic}
    \end{minipage}  
    \hfill
    \begin{minipage}[c]{0.49\linewidth}
      \centering
      \begin{overpic}[trim=0 0 0 0,clip,height=5cm]{./Figures/Rrs_680_vs_Zenith_season.eps}
        \put (16,22) {\colorbox{white}{(f)}}   
      \end{overpic} 
    \end{minipage} 

    \caption{Filtered mean $R_{rs}(\lambda)$ versus solar zenith angle, color coded by seasons (red: spring, green: summer, fall: blue, and winter: black). \label{fig:Rrs_vs_zenith_season} } 
\end{figure}

\autoref{fig:par_vs_zenith_season} show the same data but color coded by seasons.
%-%-%-%-%-%-%-%-%-%-%=FIGURE=%-%-%-%-%-%-%-%-%-%-%-%-%
\begin{figure}[H]
 \begin{minipage}[c]{0.49\linewidth}
      \centering
      \begin{overpic}[trim=0 0 0 0,clip,height=5.0cm]{./Figures/Par_vs_Zenith_chlor_a_season.eps}
        \put (9,55) {\colorbox{white}{(a)}}   
      \end{overpic}
    \end{minipage}  
    \hfill
    \begin{minipage}[c]{0.49\linewidth}
      \centering
      \begin{overpic}[trim=0 0 0 0,clip,height=5.0cm]{./Figures/Par_vs_Zenith_ag_412_mlrc_season.eps}
        \put (16,22) {\colorbox{white}{(b)}}   
      \end{overpic}
    \end{minipage} 

    \vspace{0.3cm}

    \begin{minipage}[c]{1.0\linewidth}
      \centering
      \begin{overpic}[trim=0 0 0 0,clip,height=5.0cm]{./Figures/Par_vs_Zenith_poc_season.eps}
        \put (16,22) {\colorbox{white}{(c)}}   
      \end{overpic} 
    \end{minipage}  


    \caption{(a) Chlor-{\it a}, (b) $a_{g:mlrc}(412)$ and (c) POC versus solar zenith angle, color coded by seasons (red: spring, green: summer, fall: blue, and winter: black). \label{fig:par_vs_zenith_season} } 
\end{figure}
\autoref{fig:par_vs_zenith_detrend_season} shows the anomalies of Chlor-{\it a}, $a_{g:mlrc}(412)$ and POC, color coded by season.
%-%-%-%-%-%-%-%-%-%-%=FIGURE=%-%-%-%-%-%-%-%-%-%-%-%-%
\begin{figure}[H]
    \begin{minipage}[c]{0.49\linewidth}
      \centering
      \begin{overpic}[trim=0 0 0 00,clip,height=5.0cm]{./Figures/par_vs_Zenith_detrend_chlor_a_season.eps}
        \put (16,22) {\colorbox{white}{(a)}}   
      \end{overpic}
    \end{minipage}  
    \hfill
    \begin{minipage}[c]{0.49\linewidth}
      \centering
      \begin{overpic}[trim=0 0 0 00,clip,height=5.0cm]{./Figures/par_vs_Zenith_detrend_ag_412_mlrc_season.eps}
        \put (16,22) {\colorbox{white}{(b)}}   
      \end{overpic}
    \end{minipage}  

    \vspace{0.5cm}
 
    \begin{minipage}[c]{1.0\linewidth}
      \centering
      \begin{overpic}[trim=0 0 0 00,clip,height=5.0cm]{./Figures/par_vs_Zenith_detrend_poc_season.eps}
        \put (16,22) {\colorbox{white}{(c)}}   
      \end{overpic} 
    \end{minipage}  

    \caption{Anomalies of (a) Chlor-{\it a}, (b) $a_{g:mlrc}(412)$ and (c) POC versus solar zenith angle, color coded by seasons (red: spring, green: summer, fall: blue, and winter: black). \label{fig:par_vs_zenith_detrend_season} } 
\end{figure}
%-%-%-%-%-%-%-%-%-%-%=END FIGURE=%-%-%-%-%-%-%-%-%-%-%-%-%
The anomalies of $R_{rs}(\lambda)$ versus the solar zenith angle are shown in \autoref{fig:Rrs_vs_zenith_detrend_season} color coded by season.
%-%-%-%-%-%-%-%-%-%-%=FIGURE=%-%-%-%-%-%-%-%-%-%-%-%-%
\begin{figure}[H]
    \begin{minipage}[c]{0.49\linewidth}
      \centering
      \begin{overpic}[trim=0 0 0 00,clip,height=5.0cm]{./Figures/Rrs_vs_Zenith_detrend_412_season.eps}
        \put (16,22) {\colorbox{white}{(a)}}   
      \end{overpic}
    \end{minipage}  
    \hfill
    \begin{minipage}[c]{0.49\linewidth}
      \centering
      \begin{overpic}[trim=0 0 0 00,clip,height=5.0cm]{./Figures/Rrs_vs_Zenith_detrend_443_season.eps}
        \put (16,22) {\colorbox{white}{(b)}}   
      \end{overpic}
    \end{minipage}  

    \vspace{0.5cm}
 
    \begin{minipage}[c]{0.49\linewidth}
      \centering
      \begin{overpic}[trim=0 0 0 00,clip,height=5.0cm]{./Figures/Rrs_vs_Zenith_detrend_490_season.eps}
        \put (16,22) {\colorbox{white}{(c)}}   
      \end{overpic} 
    \end{minipage}  
    \hfill
    \begin{minipage}[c]{0.49\linewidth}
      \centering
      \begin{overpic}[trim=0 0 0 00,clip,height=5.0cm]{./Figures/Rrs_vs_Zenith_detrend_555_season.eps}
        \put (16,22) {\colorbox{white}{(d)}}   
      \end{overpic}
    \end{minipage}  

    \vspace{0.5cm}
 
    \begin{minipage}[c]{0.49\linewidth}
      \centering
      \begin{overpic}[trim=0 0 0 00,clip,height=5.0cm]{./Figures/Rrs_vs_Zenith_detrend_660_season.eps}
        \put (16,22) {\colorbox{white}{(e)}}   
      \end{overpic}
    \end{minipage}  
    \hfill
    \begin{minipage}[c]{0.49\linewidth}
      \centering
      \begin{overpic}[trim=0 0 0 00,clip,height=5.0cm]{./Figures/Rrs_vs_Zenith_detrend_680_season.eps}
        \put (16,22) {\colorbox{white}{(f)}}   
      \end{overpic} 
    \end{minipage}  

    \caption{Anomalies of $R_{rs}(\lambda)$ versus solar zenith angle, color coded by seasons (red: spring, green: summer, fall: blue, and winter: black). \label{fig:Rrs_vs_zenith_detrend_season} } 
\end{figure}
%-%-%-%-%-%-%-%-%-%-%=END FIGURE=%-%-%-%-%-%-%-%-%-%-%-%-%
%%%%%%%%%%%%%%%%%%% SECTION %%%%%%%%%%%%%%%%%%%%%%%%%%%%%%%%
% \vspace{-.4cm}
\section*{References}

%%%%%%%%%%%%%%%%%%%%%%%
%% Elsevier bibliography styles
%%%%%%%%%%%%%%%%%%%%%%%
%% To change the style, put a % in front of the second line of the current style and
%% remove the % from the second line of the style you would like to use.
%%%%%%%%%%%%%%%%%%%%%%%

%% Numbered
%\bibliographystyle{model1-num-names}

%% Numbered without titles
%\bibliographystyle{model1a-num-names}

%% Harvard
% \bibliographystyle{model2-names.bst}\biboptions{authoryear}

%% Vancouver numbered
%\usepackage{numcompress}\bibliographystyle{model3-num-names}

%% Vancouver name/year
%\usepackage{numcompress}\bibliographystyle{model4-names}\biboptions{authoryear}

%% APA style
\bibliographystyle{model5-names}\biboptions{authoryear}

%% AMA style
%\usepackage{numcompress}\bibliographystyle{model6-num-names}

%% `Elsevier LaTeX' style
% \bibliographystyle{elsarticle-num}
% \bibliographystyle{apalike}
\bibliography{/Users/jconchas/Documents/Latex/bib/javier_NASA.bib} 

% \listoffigures

% \listoftables

\end{document}

%*~*~*~*~*~*~*~*~*~*~*~*~*~*~*~*~*~*~*~*~*~*~*~*~*~*~*~*~*~*~*~*~*~*~
% Antonio's comments on 6/13/17:

% Javier,

% Results look really interesting regarding 6h and 7h time points (wonder if it’s related to stray light from sun entering FOV at end of the day).  Very good start on the text. Figures look great – may not need all the figures from Fig. 7-13 are needed – let’s discuss this.  Need more descriptive figure captions for Figs. 16-18.  Do you need Fig. 17 if include Fig. 16.  Fig. 19 is really cool that the values match well for the most part.

% Let’s discuss Fig. 20 and 21.  Not sure I agree with conclusion that results get better from blue to red.  The red looks worse than blue to me.  Can you say something about how spatially and temporally homogeneous the study region is?  The temporal homogeneity is critical for the diurnal analysis.  What about day-to-day variability for the same time of day (e.g., mean and variance of a 3-day analysis at 0h, 1h, 2h, … 7h).  If you can show that from day 0 to day 2 for each time of day that GOCI Rrs (and other products) are similar to the diurnal mean or the mid 3 time points of the day, then you can establish homogeneity.  Maybe pick a 3-day period for each season of 1 year to test?

%*~*~*~*~*~*~*~*~*~*~*~*~*~*~*~*~*~*~*~*~*~*~*~*~*~*~*~*~*~*~*~*~*~*~
% Antonio's comments on 6/15/17:

% The main assumption in your analysis and thus applicability of your results is that the study region is homogeneous through the day such that you can discern if there are higher errors at certain times of the day.  So, I’m suggesting an analysis that would further support this primary assumption.  

% 1. Start small - select one 3-day sequence of GOCI images per season in 1 year (when you have 3 consecutive days of clear-sky images).
% 2. Quantify the mean and variance for each time of day … 0h, 1h, etc.  How do the statistics compare with the diurnal analyses comparing values with diurnal average and mid-day average?
% a. The point here is to demonstrate the following:  (1) demonstrate that there is minimal day-to-day and diurnal variability for this study region, (2) are the large errors  you see at 0h, 6h and 7h due strictly to the instrument/processing/SZA and not due to variability in ocean properties.
% b. Compare means and stdev (or other variance statistic) of each time of day (0h, 1h, etc.) with each other as well as with the mid-3 time points.
% 3. If the data are supportive of your assumption and conclusions (or confuse the issue), then maybe go further and analyze additional 3-day sequences.